\documentclass[12pt,Bold,TexShade]{mcgilletdclass}
\usepackage{graphicx}
\usepackage{times}
\usepackage{subfig}
\usepackage{longtable}
\usepackage{float}
\usepackage{setspace}
\usepackage{multirow} % allows multiple rows in table
\usepackage{adjustbox}
\usepackage{enumitem}
\usepackage{rotating}
\usepackage{lipsum,lineno} % for placeholder text
\usepackage{hyperref} % for hyperlink accross the document
\usepackage{array} % for sizing of tables
\usepackage{lscape} % for landscape tables

% Bib style
% \bibliographystyle{unsrt}
%\setcitestyle{authoryear,open={((},close={))}}
  \usepackage[
    style=authoryear,
    maxnames = 2,
    backend=bibtex,natbib
  ]{biblatex}
\addbibresource{mcgilletd}
\addbibresource{software}

% remove section numbering without using section* (keeps in TOC)
\setcounter{secnumdepth}{4}

% % for PDF/A
% \usepackage{hyperref}
% \hypersetup{
%     colorlinks=true,
%     linkcolor=black,
%     filecolor=black,
%     urlcolor=blue,
%     citecolor=black
% }
% \usepackage[a-1b]{pdfx}

% Reduce number of words breaking over lines
\hyphenpenalty=1000

% Cross-referencing
\usepackage{xr}
\externaldocument{"chap_1"}
\externaldocument{"chap_2"}
\externaldocument{"appendix_1"}

% May or may not be necessary to have separate bibliographies per chapter
% McGill requirements ask for it, but others did not do so
% https://engineering.purdue.edu/~mark/puthesis/faq/chapter-bibliographies/

%%%%%%%%%%%%%%%%%%%%%%%%%%%%%%%%%%%%%%%%%%%%%%%%%%%%%
%% Have you configured your TeX system for proper  %%
%% page alignment? See the McGillETD documentation %%
%% for two methods that can be used to control     %%
%% page alignment. One method is demonstrated      %%
%% below. See documentation and the ufalign.tex    %%
%% file for instructions on how to adjust these    %%
%% parameters.                                     %%
\addtolength{\hoffset}{0pt}                        %%
\addtolength{\voffset}{0pt}                        %%
%%                                                 %%
%%%%%%%%%%%%%%%%%%%%%%%%%%%%%%%%%%%%%%%%%%%%%%%%%%%%%
%%       Define student-specific info

\SetTitle{\huge{Integrating land use and land cover change simulations and connectivity modelling: a case study in the Montérégie region in southern Quebec}}%

\SetAuthor{\large{\textit{Valentin Lucet}}}%
\SetDegreeType{Master of Science\vspace{0cm}}%
\SetDepartment{Department of Biology\vspace{-2.1cm}}%
\SetUniversity{McGill University}%
\SetUniversityAddr{Montreal, Quebec}%
\SetThesisDate{2020-08-15}%
\SetRequirements{A thesis submitted to McGill University in partial fulfillment \\of the requirements of the degree of Master of Science}%

\SetCopyright{\textcopyright Valentin Lucet, 2020 \\ All rights reserved}%

%% Input any special commands below
\listfiles
\begin{document}

\maketitle

\frontmatter

\begin{romanPagenumber}{2}%
\TOCHeading{\MakeUppercase{Table of Contents}}%
\LOTHeading{{List of Tables}}%
\LOFHeading{{List of Figures}}%
\tableofcontents %


%%%%%%%%%%%%%%%%%%%%%%%%%%%%%%%%%%%%%%%%%%%%%%%%
%%         Dedication                         %%
%%%%%%%%%%%%%%%%%%%%%%%%%%%%%%%%%%%%%%%%%%%%%%%%
\SetDedicationName{{Dedication}}%
\SetDedicationText{To be completed.}%
\Dedication%

%%%%%%%%%%%%%%%%%%%%%%%%%%%%%%%%%%%%%%%%%%%%%%%%%%%%%
%%         English Abstract                        %%
%%%%%%%%%%%%%%%%%%%%%%%%%%%%%%%%%%%%%%%%%%%%%%%%%%%%%
\SetAbstractEnName{{Abstract}}%
\SetAbstractEnText{Connectivity conservation science, whose goal is to preserve the continuity of habitat throughout a given landscape, proceeds by identifying priority areas given the current configuration of the landscape. However, current connectivity conservation planning methods suffer from at least two flaws: their prioritization process fails to take into account risks associated with future land use change, and also fails to confront the results of the prioritization with the priorities perceived by stakeholders. In this thesis, we show an attempt to remedy those two issues in an ongoing effort of connectivity conservation planning for the region of Montérégie in Southern Quebec, Canada. In the first chapter, we built on past work of connectivity modelling using circuit theory in the region and complemented it with land use change modelling that uses a combination of statistical modelling and MCMC-based simulations. Models trained on past land use data were used to project future land use changes and estimate future changes in functional connectivity for 5 different umbrella species. We derived conservation priorities for the design of a local network of connected protected areas resilient to future landscape change. In a second chapter, we compare those results with the perceived conservation priorities in the region. We conducted a day-long workshop with stakeholders involved in the ongoing connectivity conservation planning effort in the region and collected information on landscape features considered “core” and “linkages” priority areas. We discuss the importance of considering land use changes to produce a resilient network of protected areas and highlight the need for a multi- stakeholder approach in the definition of conservation priorities.}

\AbstractEn%

%%%%%%%%%%%%%%%%%%%%%%%%%%%%%%%%%%%%%%%%%%%%%%%%%%%%%
%%         French Abstract                         %%
%%%%%%%%%%%%%%%%%%%%%%%%%%%%%%%%%%%%%%%%%%%%%%%%%%%%%
\SetAbstractFrName{{Abr\'{e}g\'{e}}}%
\SetAbstractFrText{To be completed.}%
\AbstractFr%


%%%%%%%%%%%%%%%%%%%%%%%%%%%%%%%%%%%%%%%%%%%%%%%%
%%         Acknowledgements                   %%
%%%%%%%%%%%%%%%%%%%%%%%%%%%%%%%%%%%%%%%%%%%%%%%%
\SetAcknowledgeName{{Acknowledgements}}%
% Among other acknowledgements, the student is required to declare the extent to which assistance (paid or unpaid) has been given by members of staff, fellow students, research assistants, technicians, or others in the collection of materials and data, the design and construction of apparatus, the performance of experiments, the analysis of data, and the preparation of the thesis (including editorial help).
% In addition, it is appropriate to recognize the supervision and advice given by the thesis supervisor(s) and advisors.
\SetAcknowledgeText{%
\noindent To be completed.
}%
\Acknowledge%

%%%%%%%%%%%%%%%%%%%%%%%%%%%%%%%%%%%%%%%%%%%%%%%%%%%%
%%        Contribution to original knowledge      %% => NOT NEEDED FOR MSC
%%%%%%%%%%%%%%%%%%%%%%%%%%%%%%%%%%%%%%%%%%%%%%%%%%%%
%\SetContribKnowledgeName{{Contribution to Original Knowledge}}
%\SetContribKnowledgeText{\lipsum[66]
%}%
%\ContribKnowledge%

%%%%%%%%%%%%%%%%%%%%%%%%%%%%%%%%%%%%%%%%%
%%       THESIS FORMAT AND STYLE       %% => UNSURE SO FAR
%%%%%%%%%%%%%%%%%%%%%%%%%%%%%%%%%%%%%%%%%
%\SetLinkName{{Thesis Format}}%
%\SetLinkText{This thesis is written in a manuscript-based format. Throughout I use the Chicago citation style.
%\\
%\\
%\textbf{Chapter 1:} Student, A. Supervisor, B. (2013), My First Chapter Title \textit{Journal of %Ecographic Informatics} 10 (1), 878887.
%\\
%\\
%}
%\Link%

%%%%%%%%%%%%%%%%%%%%%%%%%%%%%%%%%%%%%%%%%%%%%%%%%%%%
%%        Contribution of Authors                 %%
%%%%%%%%%%%%%%%%%%%%%%%%%%%%%%%%%%%%%%%%%%%%%%%%%%%%
% A doctoral thesis must clearly state the elements of the thesis that are considered original scholarship and distinct contributions to knowledge.
% Contributions of the student to each chapter must be explicitly stated.
% Contributions of any co-authors to each chapter must be explicitly stated.
\SetContribAuthorsName{{Contribution of Authors}}
\SetContribAuthorsText{\noindent
I am the first author for all chapters and the appendix in this
thesis.
\\
\\
\textbf{Chapter 1:}
I wrote the manuscript with input from my supervisor.
\\
\\
\textbf{Chapter 2:}
I wrote the manuscript with input from my supervisor.
\\
\\
}%
\ContribAuthors%
\listoffigures %
\listoftables %

\end{romanPagenumber}

%%%%%%%%%%%%%%%%%%%%%%%%%%%%%%%%%%%%%%%%%%%%%%%%%%%%
%%        General Introduction                     %
%%%%%%%%%%%%%%%%%%%%%%%%%%%%%%%%%%%%%%%%%%%%%%%%%%%%
% Clearly state the rationale and objectives of the research.
\newcommand*{\SetIntroName}[1]{\renewcommand*{\ETDIntroName}{#1}}%
\newcommand*{\ETDIntroName}{IntroMENTS}%

\newcommand{\SetIntroText}[1]{\renewcommand*{\ETDIntroText}{#1}}%
\newcommand*{\ETDIntroText}{Intro text goes here!}%

\newcommand{\Intro}{%
    \begin{simpleenv}{}{}{}{}%
    \pagestyle{plain}%
  \setboolean{SetDSpace}{false}%
    \GoSingle%
    \begin{QZ@Cent}%
          \bfseries{\ETDIntroName}
      \end{QZ@Cent}%
    \vspace*{0.5in}
      \par%
      \GoDouble%
      \setlength{\parskip}{-2em}
      %\chapter*{\ETDIntroName}%
      \addcontentsline{toc}{extrachapter}{\ETDIntroName}%
      \ETDIntroText%
    \end{simpleenv}%
    \setboolean{SetDSpace}{true}}%
%%%%%%%%%%%%%%%%%%%%%%%%%%%%%%%%%%%%%%%%%%%%%%%%%%%%

% Introduction
%%%%%%%%%%%%%%%%%%%%%%%%%%%%%%%%%%%%%%%%%%%%%%%%%%%%
\SetIntroName{{General Introduction}} % To review

\SetIntroText{%
Space is a finite resource. How we, as a community, manage and govern space is a reflection of the trade-offs and choices made by different people and organizations at different spatial and temporal scales. Those choices determine and regulate land use: if and how the resources held on the land are exploited, transformed or conserved.  The results of those choices, referred to as land use and land cover change, is an important threat to the biodiversity and ecosystem function.\\

One example of ecosystem function affected by land use that is of crucial importance to biodiversity is ecological connectivity. Ecological connectivity is the extent to which the landscape supports the movements of organisms \citep{gonzalez_spatial_2017}, and is paramount for the resilience of both populations and ecosystem services \citep{mitchell_monteregie_2015} in heterogeneous and fragmented landscapes. Land use changes such as urban sprawl can cause deforestation, fragmenting habitats, and slowly eroding ecological connectivity. Many urban landscapes are experiencing uncontrolled urban sprawl and have suffered losses in connectivity and ecosystem services in consequence. Examples include cities like Barcelona \citep{marulli_gis_2005}, New York City \citep{mcphearson_urban_2014}, and also Montreal \citep{dupras_urban_2015}.  The forces behind those land use changes are complex and understanding them is an obstacle to conservation planning  \citep{worboys_connectivity_2010}.  Because land use change is a social process with consequences of both social and ecological nature, it is best understood within the concept of social-ecological system \citep{ostrom_general_2009}. A social-ecological system (SES) can be understood as the set of human and non-human actors, the set of natural habitats they inhabit and resources and use, and the set of interactions that are maintained between all the components of the system. SESs thus form complex and integrated aggregates of interactions \citep{hinkel_enhancing_2014}. Those interactions also impact governance, the process by which actors in power establish rules and laws \citep{bissonnette_comparing_2018}.\\

Connectivity conservation planning refers to the enterprise that engages multiple actors such as academics, NGOs, governmental bodies at different scales and in a common goal to conserve the ecological connectivity of the landscape. Connectivity conservation methods usually entails modelling connectivity of the landscape of interest and using a prioritization method to determine conservation priorities. However, current connectivity conservation planning methods have at least two major limitations: their prioritization process fails to take into account risks associated with future land use change, and also fails to confront the results of the prioritization with the priorities perceived by stakeholders. Not taking into account risks of land use change would in theory lead to ill-informed conservation planned that would be over-optimistic with regard to their probability of success. In addition, failing to integrate the perceptions of stakeholders is detrimental to conservation for two reasons: first, it most likely mean that conservation will fail to gather enough local momentum to lead to actual policy change, and second, it means that the only tool for decision making will be the model results, whereas these are incomplete representation of the landscape and would benefit from inputs from stakeholders. This is especially true in landscapes where a considerable effort of connectivity modelling has already been conducted, like in the landscape of interest in this thesis, the southern Quebec region of Montérégie.\\

Montérégie is situated southeast of the city of Montreal, and contains parts of the Greater Montreal Area (GMA). The ecological connectivity of the GMA and its benefits as a provider of ecosystem services has recently been assessed in a report to the Quebec ministry of the environment (Rayfield 2018, unpublished). This study focused on identifying regions of highest connectivity, and therefore of highest priority for the conservation of biodiversity and ecosystem services. Other work by Rayfield et al. (2019, unpublished) has extended the analysis of connectivity to the whole of the Saint Lawrence Lowlands.\\

Although the map produced by this analysis is a snapshot of the current state of connectivity in the region, methods are available for including future land use and climate change impacts \citep{albert_applying_2017}. Those methods rely on established the use of land use and land cover change models whose complexity has increased from simple probabilistic state transition models to more advanced approaches using targets, discrete events and accounting for the time elapsed since the last transition \citep{verburg_combining_2009, daniel_state-and-transition_2016}.\\

Methods are also available to include stakeholder’s input in connectivity conservation planning. Some of those methods have been developed through the methodology of participatory modelling, which can be defined as a modelling framework that can integrate knowledge from multiple sources, even if this knowledge is generated by different processes. For instance, it is possible landscape perceptions by different actors and quantitative modelling. Those methods often rely on collecting data through a community-driven process during workshops. Those methods are time consuming and require a long term engagement with a given community over many years. Other workshop-based methods are less involved, and allow researchers to simply collect data to be confronted with the results of traditional modelling techniques.\\

In this thesis, we show an attempt to remedy the two issues we identified above,  in an ongoing effort of connectivity conservation planning for the region of Montérégie in Southern Quebec, Canada. In the first chapter, we built on past work of connectivity modelling using circuit theory in the region and complemented it with land use change modelling that uses a combination of statistical modelling and MCMC-based simulations. In the second chapter, we compare those results with the perceived conservation priorities in the landscape, using data collected during a day-long workshop with stakeholders. The Montérégie is relevant for our questions given the recent political momentum gained by connectivity conservation. There is a strong political will in the region for the conservation of ecological connectivity.
}%
\Intro

%  Chapter I
% !TEX root = ./thesis.tex
\chapter{Integrating Land Use Change Modelling with Connectivity Modelling}
\begin{center}
{Valentin Lucet$^{1}$, Andrew Gonzalez$^{1}$}\\
\end{center}
\textit{Author Affiliations:}\\
\normalsize{$^{1}$Department of Biology, McGill University}\\
\section{Abstract}

Ecological connectivity, defined as the extent to which the landscape supports the movements of organisms, can be strongly affected by land use. It is an important component of the resilience of populations in heterogeneous and fragmented landscapes. Land use changes such as urban sprawl and agricultural intensification intensify habitat fragmentation and landscape homogenization, leading to the erosion of ecological connectivity. The Montérégie region in southern Quebec, where this work takes place, is experiencing urban growth and sprawl. We present a framework that integrates land-use change and connectivity modelling to forecast future changes in connectivity, using a combination of statistical modelling, MCMC-based simulations, and circuit theory. We used a hybrid modelling approach to project future land use changes using different climate scenarios, and estimate future changes in functional connectivity flow for 5 different umbrella species. We explore the flexibility of a scenario approach in forecasting the range of possible futures for ecological connectivity in the region, derive insights for the hypothetical design of a local network of connected protected areas resilient to future landscape change. In conclusion, we highlight the need for a multi stakeholder approach in the definition of scenarios and conservation priorities.\\

\section{Introduction}
\vspace{2em}

%[intro paragraph, andy said it is good]
The goal of connectivity conservation planning is to preserve the continuity of habitat throughout a given landscape, by identifying priority areas and corridors important for the preservation of ecological connectivity \citep{keeley_thirty_2019}. Ecological connectivity, defined as the extent to which the landscape facilitates or impedes the movement of organisms \citep{crooks_landscape_2006}, is a critical component of the resilience of populations in heterogeneous and fragmented landscapes \citep{gonzalez_spatial_2017}. Connectivity conservation planning methods do not typically account for risks associated with future land use and climate change, two main drivers of the erosion of ecological connectivity. Here, we ask whether those methods can be improved by taking into account those risks. To do so, we integrate a connectivity model with a land use change model including climate scenarios. 

% [CC as an adaptation measure, but CC designs are vulnerable]
Connectivity conservation has often been pointed out as a method of adaptation against both climate and land use change \citep{costanza_landscape_2019}. One recurring argument is that preserving landscape connectivity could be critical for ensuring the adequate protection of shifting species ranges \citep{krosby_ecological_2010, keeley_making_2018} due to climate change. Another is that it is more important than ever to connect habitats in landscapes increasingly hostile to species movement \citep{ellis_anthropogenic_2010}. These arguments make the case for the design of Connected Protected Area Networks (CPAN, \cite{stewart_corridors_2019}). Those networks are meant to allow the movement of species across multiple scales, which may require them to include both natural and semi-natural habitat, with different levels of protections \citep{daloia_coupled_2019}. This heterogeneity is likely to be reinforced by the multiplicity of urban and land use planning practices that exist on the landscape, making CPAN designs particularly vulnerable to changes in land use. It is therefore important to test which specific CPAN design would be most resilient to future  change.  One possible approach to exploring this resilience is to use landscape-level forecast models.

%[Land use change and climate models often  used in conservation]
Land use change models have often used in conservation planning (for example, see \cite{echeverria_spatially_2008} or \cite{lesschen_identification_2007}). They allow researchers to test the efficiency of a given conservation plan in the face of expected land use change. Similarly, climate models are often used in conservation, for example to assess whether protected areas will be able to track the current changes in species ranges due to climate change \citep{araujo_would_2004, heller_targeting_2015}. They are central in species distribution models \citep{porfirio_improving_2014}, a classic tool in conservation research. 

% [what about those models in CC context?]
Both types of models are increasingly common In approaches that focus on ecological connectivity. Connectivity researchers have long been interested how past changes in land use can explain current connectivity patterns \citep{dupras_urban_2015, henareh_khalyani_spatial_2013, vergara_deforestation_2013, patru-stupariu_using_2013} and multiple connectivity measures have been developed, one specifically in this optic \citep{saura_new_2007}. But beyond these historical analyses, land use forecast models are becoming more common \citep{correa_ayram_habitat_2015}. For example, \cite{rubio_sustaining_2012}  aimed at identifying a network of forest patches resilient to land use change under multiple scenarios in Spain. Similarly, \cite{piquer-rodriguez_future_2012} integrated land use change and connectivity models to identify under-protected areas by projecting land use change trends. \cite{huang_simulating_2018} also used land use change models to assess potential impacts on functional connectivity in China. From these examples, we can see that combining land use change modelling and connectivity modelling does not yet have a fully established methodology. In its emerging form, it simply consists of running a land use change model on a given landscape, and comparing changes in the connectivity measure of choice between time steps \citep{perkl_urban_2018}. 

% [back to what we want to do]
Climate change can interact with land use change in destructive ways for biodiversity \citep{oliver_interactions_2014} It is therefore likely that connectivity models integrating these two drivers would improve connectivity conservation methods such as CPAN design. Yet, in a recent review, \cite{costanza_landscape_2019} have found that connectivity models frameworks rarely combine  land use change and climate change. Here, we build on recent connectivity modelling work using circuit theory, by pairing it with a hybrid land use and climate change model. The model we develop uses a combination of statistical modelling and simulations based on a Markov Chain Monte Carlo (MCMC) method, combining Random forests (RF) models with a Cellular automaton (CA) model. We use this RF-CA model to project future land use changes and estimate the changes in potential functional connectivity for five different umbrella species, under different climate change scenarios. We also derive conservation insights for the design of a hypothetical regional network of connected protected areas resilient to future landscape change. In addition, we provide a cursory analysis of past land use change trends in our study region, the Monteregie region in southern Quebec. 

%[we are building on past work, this work determined that we could do 5 species instead of 14]
This work is the continuation of recent research efforts on the connectivity of the region: \cite{albert_applying_2017} and \citep{rayfield_priorisation_2018}. In a seminal paper, \cite{albert_applying_2017} laid out some of the methodological steps we follow such as umbrella species selection and habitat suitability analysis. They also included a simple land use change model that was parameterized to replicate plausible change in the region. \cite{meurant_selecting_2018} developed the criteria for reducing the number of focal species from fourteen to five. They showed that because species had redundant connectivity needs, modelling the needs for 5 species resulted in qualitatively similar results than when modelling the needs of all 14 species, such as in Albert et al. (\citeyear{albert_applying_2017}). While Rayfield et al. (\citeyear{rayfield_priorisation_2018}) extended the spatial scale of the analysis to include the Saint Lawrence lowlands ecoregion, they demonstrated that we could exploit this redundancy to reduce the computing time needed for the connectivity analysis. This is an important development because land use change simulations are also computationally intensive.

% \subsubsection*{\textit{Modelling potential functional connectivity}}
%[description of connectivity]
Connectivity is a dynamic property of the landscape: it changes across time and space \citep{beyer_functional_2013}. It is also scale and species specific, varying according to the needs of each species and because those needs also vary depending on the time of the year, life history, and on the species’ habitat range and preference \citep{anderson_scale-dependent_2005}. It is important to distinguish structural connectivity from functional connectivity \citep{kindlmann_connectivity_2008}. While structural connectivity strictly refers to how animal movement is mediated by the configuration of habitat features in the landscape, functional connectivity is based on species’ preferences for such features and how they move among them. Functional connectivity modelling therefore attempts to model the extent to which the landscape meets the need of a specific species or set of species.

%[functional versus PFC]
There has been a trend in the recent literature to focus on functional connectivity, as it is regarded as a more realistic representation of  the landscape’s capacity to facilitate species movement than structural connectivity. However, functional connectivity analyses require direct animal movement data or genetic data for model fitting and for validation (for example, see \cite{beyer_functional_2013, milanesi_three-dimensional_2017}). This type of data is costly to acquire, especially when needed for many species. For this reason, much of current functional connectivity modelling, including the methods of this chapter are in fact modelling “potential functional connectivity”, or PFC. PFC models do not necessarily rely on actual movement data, but are based on other kinds of knowledge of species habitat preferences and movement behaviors, for instance, from expert knowledge and literature reviews. This makes PFC well suited to the task of this paper. 

%[we combine the PFC and redundancy concepts]
Here we model  PFC for the 5 focal species identified by  \cite{meurant_selecting_2018} and use the workflow of \citep{rayfield_priorisation_2018} for habitat suitability and connectivity analysis. We complement this framework with the RF-CA land use model. It is important to note that the primary goal of this chapter is not to explain the drivers of land use change in the region, but to provide a sufficiently realistic simulation to replicate the trends in land use change that have been observed historically, and to project those trends forward into the future.

%\subsubsection*{\textit{A hybrid land use change model}}
%[what is land use change modelling]
A classic spatially defined land use change model is what can be called a “state change model”, where the landscape is divided in a grid and each grid element (pixel) is assigned a state that changes in discrete time \citep{daniel_state-and-transition_2016}. The difficulty of land use change modelling resides in setting the transition probabilities that define the probability of a pixel transitioning to another state (or to remain as is), given its current state and given a set of conditions both intrinsic and extrinsic to that pixel.

%[phenomenological models]
A simple land use change model will assume that the rules of state changes obey markovian laws: i.e. that the next state will always only depend on the current timestep and not previous timesteps. This is a simplification: in real landscapes, the history of the pixel (beyond its present state) can play a large role in the rules governing state change. Probably due to their simplicity, markovian models such as cellular automata have a long history in land use change models \citep{agarwal_review_2002}. They have been shown to deliver accurate results, notably in urban spread modelling \citep{soares-filho_dinamicastochastic_2002, jokar_arsanjani_integration_2013, iacono_markov_2015}. They are yet another application of markov chains, and are fairly easy to set up and to run. Markov chain models represent a phenomenological approach to land use change, because the model applies the rules of transition without encoding the mechanisms underlying the  change. For example, the model defines that the state “forest” has a probability of 0.23 to transition into urban if it contains at least 2 urban pixels in a 4 pixel radius (this is an adjacency rule), but it is oblivious to the drivers of land use change.

%[mechanistic models]
In comparison to the phenomenological approach of cellular automata, a mechanistic approach to land use change modelling would fully consider that land change happens because of processes of multi-scale decision making, and will attempt to model those processes directly. Agent-Based Models (ABMs) are probably the best example of a mechanistic approach to land use change modelling. ABMs simulate the behavior of individuals in a network of decision processes whose effects are designed to sum up to the observed change \citep{parker_agent-based_2002, filatova_spatial_2013}. Those models are promising for the future of land use change modelling, but tend to be applied at smaller spatial scales, and tend to rely on expert knowledge or on a rather involved data collection procedure (such as interviews with actors, \cite{taylor_agent-based_2016}). This makes the the modelling process more difficult (albeit rewarding) as it requires a deeper understanding of the specific drivers of change in the region.

%[no guidelines, choice of model]
In the absence of a solid body of methodological comparisons between statistical frameworks in land use change modelling, it is difficult to decide on the best framework for raster-based (i.e. based on grids) models. The RF-CA model presented in this chapter is a  hybrid approach in which a statistical approach is taken (statistical modelling through random forest) and combined with a more mechanistic approach (a cellular automaton with transition targets and spatial dependency rules, CA). The RF-CA model is similar to other popular approaches like the CLUE family of models (Conversion of Land Use and its Effects \cite{verburg_modeling_2002, verburg_combining_2009}), which are some of the most commonly used models in land system science. However In CLUE, the allocation is determined by the local and regional “demand” for each land use. This demand is a latent variable for the complex and multi-scalar social-environmental processes that determine land use change, and which more mechanistic models attempt to simulate \citep{verburg_combining_2009}.

% [2 steps but there are other better integrated alternatives]
We choose here to present the two steps of probability modelling and allocation as separate because they are methodologically distinct in our model. But it is worthwhile to note that recent efforts in land use change modelling have managed to integrate those steps. Those recent methods can integrate suitability and allocation because the model is capable of encoding both processes at the same time. We can mention important work on modelling land use change with machine learning techniques such as Neural Networks (NNs) \citep{tayyebi_simulating_2013} or exploiting the potential of non frequentist statistics in the case of Belief Networks (BBNs, see \cite{celio_modeling_2014} for a great example). BBNs, like ABMs are promising methods but tend to be applied at smaller spatial scales.

%\subsubsection*{\textit{A scenario-based approach}}
% [scenarios]
 Scenarios are a flexible approach to forecasting the range of possible futures in landscapes \citep{peterson_scenario_2003}. We integrate land use change and connectivity modelling under two “warming” scenarios (Baseline and RCP 8.5) and one “control” scenario (Historic). 
Depending on the scenario, the model generates different forest change dynamics. The Historic scenario assumes no change in forest composition and stands as our control. While the baseline scenario assumes no further warming and status quo forest change dynamics, the RCP 8.5 scenario is one of the most severe climate change scenarios, and the trends observed the in the baseline scenario are exacerbated \cite{IPCC}. We also derive two simple land use change scenarios: Business as usual (BAU) and Reforestation. Those two scenarios are meant to represent two extremes in land use change, and it is likely that the most probable trajectories for our landscape will be within those two boundaries. In combination with the climate scenarios, this brings our total scenario number to 6.
What predictions can be made as to how connectivity will be affected under our model? There has been a significant research effort to study and predict forest dynamics under climate change in Quebec. It has been shown that the distributions of many of the most abundant trees occurring in the region are out of equilibrium with climate at the margin of their range \citep{talluto_extinction_2017}. Some papers points to existing and expected lags in the response of forest to climate change \citep{savage_elevational_2015,}, and others have commented on how disturbances and forest management practices are likely to fasten and facilitate range shifts \citep{leithead_northward_2010, boulanger_climate_2019, vieira_paying_2020}. A major prediction that can be made is that although boreal type (coniferous) forest is likely to respond more slowly to climate range than deciduous type, both coniferous and mixed population types in southern Quebec are likely to become more rarer, and be replaced by deciduous stands. 
We can therefore derive different predictions as to how connectivity will be affected for each species under our model. Three main predictions can be made. First, under the BAU scenario, we can expect that our land use change model will simulate an decrease in connectivity change for our focal species. We can also expect connectivity to maintain itself under the Reforestation scenario. Finally. we can expect climate change to impact more strongly species that prefer coniferous  and mixed forest types. \\

\section{Methods}

Our workflow is divided into two major steps: land use change modelling and connectivity modelling. Two workflow figures describe the steps and methods involved (figure \ref{fig:workflow1} and \ref{fig:workflow2}). \\

\subsubsection*{\textit{Software tools and reproducibility}}

All work was conducted in the R statistical software version 3.6.2 "Dark and Stormy Night" \citep[see][]{R}). The data and code for this analysis are available at \href{https://github.com/VLucet/landchange-connectivity-Montérégie}{this GitHub repository}.

We used ST-Sim version 2.2.10, scripted in R with the help of the rsyncrosim package version 1.2.0. ST-Sim ran on Linux (Ubuntu 18.04) via Mono 6.4. Many steps of data preparation were conducted in GRASS GIS 7.8.

A note should be added on data availability and reproducibility. A few datasets were not obtained from open sources. These datasets are marked in red in figure \ref{fig:workflow1}. The Canadian census data was obtained through the University of Toronto's CHASS (Computing in the Humanities and Social Sciences) Data Centre, which requires institutional access. This makes the Random Forest analysis not reproducible. In addition, data on protected areas in Montérégie was obtained  via the RMN (Réseaux des milieux naturels protégés du Québec) under a limited license, which makes the Land use change modelling via ST-Sim not fully reproducible. However, the results of those steps are made open, and the subsequent steps (habitat modelling, connectivity analysis) are reproducible via the use of various software ( i.e. docker, renv, and more). Please consult the appendix on reproducibility for the steps to take to reproduce those results. \\ %TODO write appendix on reproducibility

\subsection{Land Use Change Modelling}
Land use change modelling is a prolific subfield of land systems science which has spurred a broad diversity of approaches \citep{dang_review_2016, noszczyk_review_2018}. Beyond the many methods published, there is a large number of applications and software tools available for research - although their openness and accessibility varies \citep{moulds_open_2015}. Papers that compare frameworks and results across tools and methods are rare \citep{pontius_comparing_2008, pontius_comparison_2005, sun_comparison_2018}. Although a full review of land use change modelling methods is beyond the scope of this chapter and this thesis, I will introduce the key concepts.

Land use change models can be classified according to their level of focus on land use change drivers, from phenomenological (“top down”) to more mechanistic approaches (“bottom up”). Given the simplicity and ease of use of phenomenological models, and the extra demands of mechanistic models, most approaches to land use change modelling fall between these two extremes, and a certain number of “hybrid” approaches have been developed \citep{sun_comparison_2018, jokar_arsanjani_integration_2013}. The approach we take in this chapter is not strictly phenomenological and introduces some mechanistic understanding of land use change, and can therefore be qualified as a hybrid method.

Our land use change model is set up in two steps: a land use suitability analysis and an allocation algorithm. In a nutshell, the suitability analysis step determines how likely each pixel is to acquire a given state by producing a probability surface, predicted based on spatial data, and the allocation step simulates change over that surface. The allocation is a Markov process informed by the results of the first step. This Markov process is embellished to include conditions such as neighboring rules and transition size distributions.

The two steps framework we just outlined allows us to combine the power of markov chains with the flexibility of statistics. Any method can in theory be used to determine the probability surface (although linear and logistic regression techniques are most common), and complex algorithms can be written to determine rules of allocation. The simplicity and flexibility of this framework explains the popularity of common land use change modelling frameworks like CLUE  (Conversion of Land Use and its Effects \cite{verburg_modeling_2002, verburg_combining_2009}). \\

\subsubsection{The RF-CA modelling framework}

Our choice of statistical framework was done considering at least three factors: computing time, prediction power and capacity to deal with spatial processes. Computing time is an important factor because fitting models to large datasets can be time consuming. Capacity for dealing with spatial processes varies across methods. Finally, statistical frameworks sit along a continuum that emphasises more or less inference and prediction. A Bayesian model will be more easily thought of as a tool for statistical inference, whereas a neural network will be used more with predictive power in mind. 

We chose Random Forests (RF) as a statistical framework. RFs are a “tree-based machine learning algorithm that generates a “forest” of randomized independent to each other and identically distributed decision trees”. RFs can be easily “grown’ in parallel, which reduces computation time. RFs possess a strong predictive power, but can still provide metrics of variable importance, providing an interesting balance between inference and predictive power.

For the allocation step, we use an advanced cellular automata (CA) modelling tool (markov-chain based) called ST-Sim, a package of the free software SyncroSim. ST-Sim allows for more complex simulation than a classic CA as it allows for complex neighbouring rules with state and age cell tracking. It is also built to handle large datasets and can be scripted to run in parallel. It is important to note that although it is the first time RF is used with ST-Sim (to our knowledge), this hybrid modelling approach (coined “RF-CA”) has been used successfully in other land use change modelling projects \citep{kamusoko_simulating_2015, gounaridis_random_2019}. ST-Sim can be customized to function under multiple different types of inputs. The two main inputs are transition probability and transition targets. Readers should refer to Daniel el al. (\citeyear{daniel_state-and-transition_2016}) for a more in-depth overview of how ST-Sim and SyncroSim functions. \\

\subsubsection{Data sources and preparation}

\subsubsection*{\textit{Land use change data}}

For most modelling frameworks, the raw source of land use change data is remote sensing data. Because producing land use change data from remotely sensed imagery was beyond the scope of this thesis, we looked for a curated dataset. Agriculture and Agri-Food Canada (AAFC) produces a dataset of land use in Canada at a 10 years interval and at the resolution of 30 meters (the resolution is the dimension of a pixel in a spatial grid, or raster data, and is an important property of land use datasets). This dataset provides land use data for the years of 1990, 2000, 2010. We found this data product, referred to hereafter as the AAFC data  to be the right fit for our approach: the data production method of the AAFC  data is consistent, meaning that similar methods have been used to produce all three maps at different time points. Consistency in data production methods is key as it makes the computation of change between time points more reliable.

This data is not without limitations: compared to other data products - such as the Quebec Ministry of the Environment land use dataset or the Statistique Quebec land accounting dataset - the AAFC dataset is not very detailed: only a few land use categories are identified, and important information for suitability analysis such as forest age and forest density are missing.

Another limitation is the coordinate reference system (CRS) in which this dataset is available: it is projected in a UTM projection system, which is not the standard CRS used by other data products that usually covers Quebec and emanates from open data portals in the province. This makes direct comparison (pixel by pixel) to those others datasets difficult. Finally, the AAFC data does not differentiate between road types, which has implications for habitat suitability analysis (see the habitat suitability analysis section).\\

\subsubsection*{\textit{Forest types and dynamics under climate change}}

The AAFC data does not differentiate between different forest types. However, our connectivity analysis relies on an analysis of habitat suitability, which is reliant on knowing the preferences of different species for different forest types (see section of habitat suitability analysis for more details). In order to integrate dynamics of forest types into our simulations, we relied on the results of a separate set of simulations that were run in LANDIS, a free and open source source designed to model forest growth. Those simulations were run by Larocque and Rayfield (unpublished report to the Quebec Ministry of the Environment) for the extent of the Saint-Lawrence lowlands ecoregion, in the context of a research contract with the MELCC (Ministère de l'Environnement et de la Lutte contre les changements climatiques). The LANDIS simulations do not include land use change and therefore can not be directly merged into our simulations. Instead, we parameterized the changes in forest types from the results of the LANDIS simulations. A more detailed methodological description of the LANDIS simulations is to be found in the appendix ([add ref to appendix]). %TODO write this appendix with the help of Yan and Guillaume

We source two different sets of "parameters" from these simulations: first, forest type data in 2010 was used as a starting point for the forecast scenarios. Second, a set of transition multipliers were extracted from the simulations and was used to reflect different forest dynamics under different climate change scenarios. These are described in more details in the "Model Execution" section.\\

\subsubsection*{\textit{Explanatory variables}}

A diversity of drivers of land-use change can be found in the literature. Some general group of variables can be identified: an important distinction can be made between physical and social-economical variables. Proxy variables for land use change drivers used in this chapter were selected among commonly recognized predictors of land use change.

A digital elevation model was used to derive elevation data (continuous variable). We used the data product “SRTM 30m” available throughout the Google Earth Engine data portal to generate this variable. This data was already available at a 30 m resolution.

We used the Canadian Census for the years of 1991, 2001 and 2011, provided by Statistics Canada, to extract two variables: population change (1991 -2001) and average income. The data was not available in a spatial format and had to be turned into a spatial object in R before being rasterized. The data was collected at the lowest aggregation level available: the denomination area (DAs) or the Enumeration area (EAs) depending on the census year (see table \ref{tab:variables} for the full description of variables and data sources).\\

\subsubsection*{\textit{Data preparation}}

The first step of preparing the data from the model consisted in reclassifying the AAFC land use dataset into only 5 categories: the 3 categories of our land use change model and supplemental categories of roads and wetlands. Then, all the pixels classified as forest were reclassified as null, and the resulting maps were patched with the forest cover from the LANDIS simulations' starting conditions (state of forest types in 2010). At the end of this process, some cells remained null as the forest cover in 1990 was larger than that of 2010. We filled those NULL values using modal imputation under two different neighbourhood window sizes (13 and 25 pixels). Next, a 5km buffer was added around Montérégie so as to improve the Circuitscape results later down the pipeline (see the Connectivity Analysis section). Not all NA cells were due to the difference in forest cover between 1990 and 2010, some were due to the fact that the LANDIS run did not cover some parts of the 5km buffer. Those cells were filled with the value corresponding to the forest type Mixed Forest of Medium Age. We decided on this imputed value because it is among the most suitable for all species and therefore unlikely to cause significant changes in the Circuitscape runs, even if it does not correspond to the mode or the mean for the region. In addition, because no change takes place in the buffer, those values can be equivalent to a "padding" around the landscape.

A second step was the preparation of the explanatory variables. Vector data was rasterized when needed. A significant effort was invested in working with the Canadian Census data, as Statistics Canada does not make available a spatial dataset for their census data. Therefore, the geometries of DAs and EAs had to be matched with the corresponding line in the Census dataset, based on the unique EA and DA IDs. Some of these (between 2 and 5 \% of the total number of DAs or EAs, depending of the time steps) have remained unmatched due to discrepancies between the (spatial) boundaries datasets of EAs/DAs available and the EAs/DAs listed in the StatsCan dataset. This led to areas with NAs values. These areas could not be ignored in our random forest model and had to be imputed. We used mean imputation for lack of a better method but are aware of the large number of issues associated with this practice \citep[see][]{lodder_impute_2014}.

In order to reduce run time as well as memory requirements, we ran the entire simulation at a 90m resolution, which is an aggregation (resample) of all the data in the rasters by a factor of 3, using a majority rule (changing the resolution of our data to 90m). Finally, all variables were extracted and turned into tabular data. Only relevant pixels (i.e. rows in the dataset) were kept to model each transition. For instance, only forest and agricultural land pixels were kept to model urbanization. The variables were standardized prior to running the model. We used the \verb|tidymodels| framework (version 0.1.0) in R to pre-process our tabular data and run our random forest models. \\

\subsubsection{Preliminary analysis of past land use change}

To better understand land use patterns in the region, we ran some preliminary analysis on land use change data at the municipality level. Montérégie is made up of 177 different municipalities. We generated land use matrices (amount of each land use category) and land cover transition matrices (1990 to 2010) for each municipality and for the land use categories corresponding to our model. We then performed multivariate analyses by running a first ordination using PCA (Principal Component Analysis), and then using Ward clustering % TODO add ref to legendre.
Because municipalities in Montérégie have variable size, the amount of change in each municipality needs to be standardized. The data was normalized prior to the ordination to reflect relative amounts of change in each municipality. We used the vegan package in R to produce such plots.\\

\subsubsection{Model Execution}

\subsubsection*{Random forest}

\subsubsection*{\textit{Calibration \& validation}}

Land use change data has a tendency to be highly unbalanced, because over a certain time period, the amount of pixels that have transitioned into a new state is usually small compared to the amount that did not transition into a new state. In the case of Random forest models, there are a few ways to deal with such imbalance. In this work, we simply down-sampled our dataset to reach a ratio of 2:1 (i.e. 2 pixels that did not transition for each pixel that did transition).

The RF model was calibrated on a training partition (70\% of the down-sampled dataset) for the timestep of 1990 to 2000. The model was then spatially validated on a test partition (30\% of the dataset), and then temporally validated for the timestep of 2000 to 2010 (100\% of the dataset for this timestep). \\

\subsubsection*{\textit{Performance evaluation}}

There are a number of ways to evaluate model performance in the context of land use change predictions. We used three different approaches:
\begin{itemize}
 \item We report the $R^{2}$ resulting from the random forest model, as well as variable importance (evaluated as gini impurity score).
 \item We used a ROC curve with Area Under the Curve (AUC) to compare true observed change with the predicted probabilities of transition. This method is widely used in land use change modelling. %TODO add ref (ANDY)
 \item We used a Conventional Cross Validation (CCV) method for the reporting of AUC values, with 10 folds, for which we also produce a measure of AUC with ROC curves.\\
\end{itemize}

\subsubsection*{ST-Sim}

ST-Sim \citep{daniel_state-and-transition_2016}  is a State and Transition model (STM) and can therefore be provided with probabilities of land use change. However in order to integrate the results of our random forest models, we parameterized ST-Sim differently than a classic STM. All probabilities are set to 1, and we leverage the spatial multiplier feature of the software. We provide our fitted probability surfaces (outputs of the RF models) as spatial multipliers. Therefore, each pixel's probability of change is provided by that surface (multiplied by 1).  This is not enough to reproduce past trends in land use change: we need to provide transition targets to the allocation algorithm. These targets are directly based on the amount of land change between 1990 and 2010 for time steps 1990 - 2010.\\

\subsubsection*{\textit{Model definition}}

The ST-Sim model uses three states: Forest, Urban and Agriculture. Forest was further subdivided into classes of age and type, derived from the LANDIS outputs: three forest types (\textbf{decididuous}, \textbf{mixt}, and \textbf{coniferous}) and three age classes (\textbf{young}, \textbf{medium}, and \textbf{old}) for a total of 9 classes.
The model allows for 4 groups of transitions:
\begin{itemize}
\item{\textbf{Urbanisation}}: groups \textbf{deforestation} (Forest $\Rightarrow$ Urban) and \textbf{agricultural loss} (Agriculture $\Rightarrow$ Urban).
\item{\textbf{Agricultural expansion}} (Forest $\Rightarrow$ Urban).
\item{\textbf{Forest Internals}}: groups the 72 combinations of forest change between the 9 forest internal states.
\end{itemize}
The model takes in 3 different strata. A stratum in ST-Sim refers to a subset of the entire landscape being considered for analysis. Strata are typically used to divide the overall landscape into regions or zones, each of which can have different model parameters specified. The three stratum are:
\begin{itemize}
\item{Primary stratum:} delineates Montérégie from the  surrounding buffer and highlights protected areas.
\item{Secondary stratum:} delineates municipalities of the region, to allow to set targets specific to each municipality. 
\item{Tertiary stratum:} delineates LANDIS land types, to allow to set transition multipliers for forest types that match the LANDIS simulations. \\
\end{itemize}

\subsubsection*{\textit{Spatial dependency}}

Under the parameters defined so far, ST-Sim takes care of transitioning the amount of land corresponding to the targets it is fed, and will do it probabilistically across the landscape according to the probability surface (spatial multipliers). There is an additional way to restrict this change via Adjacency parameters, which are essentially neighbourhood rules. Adjacency rules were defined from basic assumptions about transition spread. The list of neighbourhood rules given for each transition type (or group) is provided in the appendix table \ref{tab:neigh_rules}.

We added two more constraints for the Forest Internals group: the Time since Transition parameter which sets a minimum time of 20 years (2 timesteps) in between transitions between Forest  types states. In addition, pathway autocorrelation settings allows to further isolate the characterization of transition rules to the specific land type to which a pixel belongs.\\

\subsubsection*{\textit{Scenario definition}}

In the subsequent description, we make the distinction between:
\begin{itemize}
\item{\textit{historic runs}}, parameterized to reproduce the changes that happened between 1990 and 2010 in the region, with a different set of transition targets for 1990-2000 and 2000-2010.
\item{\textit{forecast runs}}, parameterized to project forward an average of the trends observed in the two decades, with a single set of transition targets (average of the two 1990-2000 and 2000-2010 targets) and for which only certain set of spatial multipliers change depending on climatic scenarios.
\end{itemize}
We defined two land use change scenarios, crossed with two different climate change scenarios, plus a control climate scenario. Those six scenarios in total are designed to  encompass the realm of possible future habitat changes for our 5 focal species.
The two land use change scenarios are:
\begin{itemize}
\item{\textbf{Business As Usual (BAU)}}: the land use change trends are projected forward into the future with no alteration to those trends.
\item {\textbf{Business As Usual + Reforestation (BAU-R)}}: the land use change trends are projected forward into the future, and reforestation is randomly occurring, converting to forest as much land as is lost through urbanization, for every timestep.
\end{itemize}
The climate scenarios influence the probability of transitions between forest types (i.e. succession and disturbances) they subsequently influence habitat suitability and connectivity. The three climate scenarios are:
\begin{itemize}
\item{\textbf{Historic (HIST)}}: No forest change: forest remains as it was in 2010. This is equivalent to a control treatment.
\item{\textbf{Baseline (BASE)}}: Forest change continues as it does in today's climate.
\item{\textbf{RCP 8.5 (RCP8)}}: The forest changes according to the emissions scenario RCP 8.5, characterised by increasing green-house gas emissions, high rates of population growth, modest GDP growth and low rates of technological development and uptake. It is the most severe of the future scenarios \citep{ipcc_summary_2013}.\\
\end{itemize}

\subsubsection*{\textit{Details of scenario parameters}}

We ran a total of seven scenarios (one historic run and six forecast runs). Two behaviours remain constant in all seven scenarios: no land use change  takes place in the 5km buffer. In addition, new forest laid by the reforestation transition does not change state for the remainder of the simulation (this is because LANDIS does not provide land type data for new pixels). Therefore, reforestation always produces a forest of the class "Medium Deciduous". In all scenarios described, ST-Sim with decadal time steps, for 10 (5?) iterations (10 MCMC realization).

The final list of scenarios is the following:
\begin{itemize}
\item Historic run - 1990 to 2010 \textbf{(HIST)}. In this historic run, the forest type for all three time steps was taken from the LANDIS inputs (forest type data 2010). This is of course not accurate, but because no change is allowed to take place, this is not a problem and it is only there to allow consistency and comparison with the other scenarios.
\item Forecast land use BAU - Historic climate - 2010 to 2100 \textbf{(BAU-HIST)}
\item Forecast land use BAU - Baseline climate - 2010 to 2100 \textbf{(BAU-BASE)}
\item Forecast land use BAU - RCP 8.5 climate - 2010 to 2100 \textbf{(BAU-RCP8)}
\item Forecast land use reforestation - Historic climate - 2010 to 2100 \textbf{(BAU-R-HIST)}
\item Forecast land use reforestation - Baseline climate - 2010 to 2100 \textbf{(BAU-R-BASE)}
\item Forecast land use reforestation - RCP 8.5 climate - 2010 to 2100 \textbf{(BAU-R-RCP8)}\\
\end{itemize}

\subsection{Habitat suitability and connectivity modelling}

Connectivity modelling is similar to land use change modelling inasmuch as the diversity of methods at the researcher’s disposal is large \citep{calabrese_comparison_2004}. As mentioned in the intro, we implemented an analysis of Potential Functional Connectivity, or PFC. A typical PFC modelling workflow can be broken down to 3 steps: species selection, Habitat suitability modelling and connectivity analysis. The three steps are described below, and the methodology partially follows the methodology of Rayfield et al. This part of the methods is summarized in figure \ref{fig:workflow2}.\\

\subsubsection{Species Selection}

Species selection is unchanged from Rayfield et al. The five species chosen are presented in the appendix table \ref{tab:species}. It was demonstrated that modelling the connectivity needs of those 5 species was equivalent to modelling the needs of 14 species in the study region \citep{meurant_selecting_2018, albert_applying_2017}\\

\subsubsection{Habitat Suitability}

Habitat suitability analysis consists in reclassifying land use data into a resistance surface which is then used to model connectivity. For each of the maps obtained from the land use change analysis (for each of the timestep in every iteration and for each of the 5 species), we followed the following steps to reclass land use into resistance:
\begin{enumerate}
\item Classify the landscape as \textbf{forest} and \textbf{non forest}.
\item Reclassify non-forest pixels into resistance following the supplementary tables in  \citeauthor{rayfield_priorisation_2018}, with only a few alterations. See table \ref{tab:key_non_forest} for the reclassification key.
\item Clump the forest landscape into contiguous \textbf{patches} using queen case neighbourhood (includes diagonal).
\item Reclassify pixels in all patches as suitable (value = 1) unsuitable (value = 0) or moderately suitable (value = 0.5), based on forest age and type, and following guidelines in the supplementary tables in  \citeauthor{rayfield_priorisation_2018}.  The key for such reclassification is recorded in table \ref{tab:suit_pixls}.
\item  Compute the mean patch value based on the clumped outputs.
\item Classify each patch as \textbf{possibly suitable} or \textbf{unsuitable}, with a cut-off value of 0.5.
\item Among the suitable patches, calculate their size.
\item Classify \textbf{possibly suitable} patches into \textbf{fully suitable} patches or \textbf{too small} patches based on their size, using the values provided by \citeauthor{rayfield_priorisation_2018} (see table \ref{tab:patch_size}).
\item Reclassify patches into resistance, once again following the guidelines in \citeauthor{rayfield_priorisation_2018}'s supplementary tables. See table \ref{tab:hab_or_not} for the reclassification key.
\item Patch the re-classified non-forest map with the re-classified patches map.  
\end{enumerate}

Although these steps build on the work of \cite{albert_applying_2017} and \cite{rayfield_priorisation_2018}, we apply a simpler habitat suitability modelling workflow. There are a number of reasons for why we were not able to apply the exact same workflow as  in \citeauthor{rayfield_priorisation_2018}. First, the resolution of the land use change data differs: we used a 90m resolution versus a 30m resolution for \citeauthor{rayfield_priorisation_2018}. This has implications for the minimum patch size required for the two species (\textit{Blarina} and \textit{Plethodon}) which at the resolution of 90m becomes impossible to differentiate. In addition, we did not include a range of variables that were used in \citeauthor{rayfield_priorisation_2018}: forest density, soil drainage, distance from minor versus major roads,   \\

\subsubsection{Connectivity analysis}

Methods of connectivity modelling are grounded in important concepts, the most important of which is graph theory \citep{dale_graphs_2010}. Graph theory is the study of networks, and describes the properties of the elements (edges and vertices) that make up a network . Applying graph theory to a landscape comes down to collapsing that landscape into a network, from which metrics can be computed, and models can be fitted - for instance dispersal models, which makes metapopulation theory another important concept in connectivity\citep{hanski_habitat_1999}. Metapopulation theory describes how populations of a given species in a landscape persist due to dispersal and gene flow.

In this paper, we analyse connectivity using circuit theory-based software called Circuitscape. Circuit theory is used to simulate the movement of propagules. The software “borrows algorithms from electronic circuit theory to predict connectivity in heterogeneous landscapes''. It is a free and open software under MIT license developed originally in Python, and now in its 5th version, in Julia \citep{circuitjulia}.

For each of the maps produced in the suitability analysis, Circuitscape was run in two directions (“wall to wall” run, \cite{mcrae_conserving_2016}): east to west and north to south. This method allowed us to model omnidirectional animal movements, and required that we added the resulting North/South and East/West flow map. The results for each map were therefore added to produce a final set of flow maps for analysis. From these flow maps, we extracted the mean flow for each of the municipalities and for the entire flow map. \\

\section{Results}

We first describe the results of the preliminary analysis on land use change dynamics, then go on to describe the results of the hybrid land use change and connectivity model.\\

\subsection{Analysis of past land use changes in Montérégie}

The clustering and subsequent ordination of the land use change matrices of the 177 municipalities revealed that Montérégie has 5 profiles and land compositions (see figures \ref{fig:clustervals}, \ref{fig:PCAvals}, and \ref{fig:mapvals}):
\renewcommand{\labelitemi}{$\textendash$}
\begin{itemize}[leftmargin=0.5cm]
  \item \textbf{Forest - Dominant}: have the lowest level of fragmentation and are dominated by forest
  \item \textbf{Forest - Agriculture}: still have a healthy amount of forested areas but fragmentation is much more pronounced.
  \item \textbf{Agriculture - Dominant}: forested habitat is scarce and most of the remaining forest is classified as “Trees” in the AAFC dataset (forest fragment of less than 1 hectare)
  \item \textbf{Urban - Medium density}: correspond to the front of the wave of urban sprawling
  \item \textbf{Urban - High density}: urban cores make up most of the municipality
\end{itemize} 

The clustering and subsequent ordination of land use change profiles showed that Montérégie has 4 different profiles (see figures \ref{fig:clustertrans}, \ref{fig:PCAtrans}, and \ref{fig:maptrans}):
\begin{itemize}[leftmargin=0.5cm]
  \item \textbf{Urban Spread / Deforestation}: forest fragmentation is progressing mainly via the growth of urban land (in the west) or villegiatives pressures (in the east)
  \item \textbf{Urban Spread / Agricultural loss}: agriculture is losing ground to urban land
  \item \textbf{Agricultural Expansion / Fragmentation}: forest is losing ground to agriculture in those municipalities where forest is still quite present
  \item \textbf{Agricultural Expansion / Deforestation}: forest is already scare and is being replaced by agricultural lands
These results show interesting regional trends with a front line of fragmentation and deforestation on each side of the region and along the Richelieu river. \\ % (Fig. \ref{fig:map}).
\end{itemize}

\subsection{Random forest Models}

\subsubsection{Variable importance}

As previously mentioned, the primary goal of this chapter is not to draw strong inference with regards to land use change drivers in the region. It is still worth noting that  the most important variables to predict urbanization and agricultural expansion in our model are the distance from urban land and size of forest patches, respectively (see \ref{tab:varimp}). This demonstrates that small patches close to urban areas are the most under threat of conversion in the region.  \\

\subsubsection{Model performance}

The reported adjusted  $R^{2}$ for our urbanization and agricultural expansion random forest models are $0.57$ and $0.56$ respectively. The values for AUC are comparatively higher: varying between $0.841$ and $0.931$ (depending of the validation set) for Agricultural Expansion and between $0.921$ and $0.939$ for Urbanisation (see figures \ref{fig:roc_agex} and \ref{fig:roc_urb} for the spatial validation ROC curves, and see summary table \ref{tab:R_squares_AUC}). The resampled AUC (10-folds CCV) is equally as high with means of $0.929 \pm .002$ and $0.938 \pm 0.002$ respectively (see figure \ref{fig:roc_rs} for the 10-fold ROC curves).\\ 

\subsection{Land Use Change Model}

%% Descrive results of the model with figures that shows the changes in forest compositions and land use composition.  
\subsubsection{Historic run}


\subsubsection{Scenario runs}

\subsection{Connectivity Modelling}

%% Show results of historic versus scenarios (make different figure for only this chapter)
\subsubsection{Historic run}

\subsubsection{Scenario runs}

%---------------------------------------------------------------------------------------------------------------------------------------------------
\section{Discussion}

% discuss 1) what methodological integration was achieved 2) the limitations of the method and 3) compare and contrast scenarios

\section{Conclusion}

%---------------------------------------------------------------------------------------------------------------------------------------------------

\newpage
\begin{center}
\section*{Figures \& Tables}
\end{center}

%---------------------------------------------------------------------------------------------------------------------------------------------------
% methods figures %TODO better caption describing colors

\begin{figure}[h]
\makebox[\textwidth]{
\includegraphics[width=1.3\textwidth]{figures/Chapter1_flowchart.png}
}
\caption{Workflow for data preparation, statistical modelling and land use change modelling.}
\label{fig:workflow1}
\end{figure}
\clearpage

\begin{figure}[h!]
\makebox[\textwidth]{
\includegraphics[width=1.3\textwidth]{figures/Chapter1_flowchart2.png}
}
\caption{Workflow for the habitat suitability and connectivity analyses.} 
\label{fig:workflow2}
\end{figure}
\clearpage

%---------------------------------------------------------------------------------------------------------------------------------------------------
% Results figures

% Figures: values

% Clustering moved to appendix

% PCA
\begin{figure}[h!]
  \centering
    \includegraphics[width=0.9\textwidth]{figures/PCA_data_profiles.png}
  \caption{Ordination of land use data (proportions) for municipalities. Groups are derived from clustering in \ref{fig:clustervals}}
  \label{fig:PCAvals}
\end{figure}

% MAP
\begin{figure}[h!]
  \centering
    \includegraphics[width=\textwidth]{figures/profiles_land_use.png}
  \caption{Geographical distribution of the 5 profiles identified in \ref{fig:clustervals} and \ref{fig:PCAvals}.}
  \label{fig:mapvals}
\end{figure}

% Figures: Transitions

% PCA
\begin{figure}[h!]
  \centering
    \includegraphics[width=0.9\textwidth]{figures/PCA_trans_profiles.png}
  \caption{Ordination of land use transition data for municipalities. Groups are derived from clustering \ref{fig:clustertrans}}
  \label{fig:PCAtrans}
\end{figure}

%MAP
\begin{figure}[h!]
  \centering
    \includegraphics[width=\textwidth]{figures/transition_prof_map.png}
  \caption{Geographical distribution of the 4 change profiles identified in \ref{fig:clustertrans} and \ref{fig:PCAtrans}.}
  \label{fig:maptrans}
\end{figure}

\clearpage

% Other results figures
%--------------------------------
%  Resample, figure and table
\begin{figure}[h!]
\makebox[\textwidth]{
  \includegraphics[width=1.3\textwidth]{figures/double_roc_resample.png}
}
 \caption{ROC Curves for re-samples}
 \label{fig:roc_rs}
\end{figure}

\begin{figure}[h!]
\makebox[\textwidth]{
  \includegraphics[width=\textwidth]{figures/rf_ratio_2_agex_roc.png}
}
 \caption{ROC Curve for Agricultural Expansion}
 \label{fig:roc_agex}
\end{figure}

\begin{figure}[h!]
\makebox[\textwidth]{
  \includegraphics[width=\textwidth]{figures/rf_ratio_2_urb_roc.png}
}
 \caption{ROC Curve for urbanisation}
 \label{fig:roc_urb}
\end{figure}

%---------------------------------------------------------------------------------------------------------------------------------------------------
% Other Tables

% Model validation
\begin{table}[!htbp] \centering 
  \caption{$R^{2}$ values for both models and both validation sets.} 
  \label{tab:R_squares_AUC} 
\begin{tabular}{@{\extracolsep{5pt}} ccccc} 
\\[-1.8ex]\hline 
\hline \\[-1.8ex] 
Model & Validation set & $R^{2}$ & AUC & Average Precision \\ 
\hline \\[-1.8ex] 
Agricultural Expansion & Spatial & $0.566$ & $0.931$ & $0.652$ \\ 
Agricultural Expansion &  Temporal & $0.566$ & $0.841$ & $0.023$ \\ 
Urbanisation & Spatial & $0.573$ & $0.939$ & $0.234$ \\ 
Urbanisation & Temporal & $0.573$ & $0.921$ & $0.250$ \\ 
\hline \\[-1.8ex] 
\end{tabular} 
\end{table} 

% RF Variable importance
\begin{table}[h!]
\centering
\caption{Variable importance (gini impurity index) for both models - non-categorical variables only}
\label{tab:varimp}
\begin{tabular}{lcc}
\hline
\multicolumn{1}{c}{\multirow{2}{*}{\textbf{Variable}}} & \multicolumn{2}{c}{\textbf{Model}} \\ \cline{2-3} 
\multicolumn{1}{c}{} & \multicolumn{1}{l}{\textbf{Urbanisation}} & \multicolumn{1}{l}{\textbf{Agricultural expansion}} \\ \hline
Distance from urban land & 756.8292 & 1787.357 \\
Size of forest patch & 548.5826 & 5373.270 \\
Elevation & 580.9726 & 1823.918 \\
Population change & 447.1120 & 1138.319 \\
Income & 390.5668 & 1108.612 \\ 
\hline
\end{tabular}
\end{table}

\clearpage

\SetLinkName{Linking Statement}
\SetLinkText{In Chapter I, we integrated a land use and climate change model with a habitat suitability and connectivity model under two different scenarios, and showed how different species might react to changes in land use and climate by analyzing the resulting flow surfaces. We provide  a reproducible yet perfectible methodology to start to take into account important drivers like land use and climate into regional connectivity conservation planning, using the case study of the Montérégie region. We demonstrate the power of simple scenarios to draw the boundaries of the prediction space for potential functional connectivity. However, we recognize the need for scenarios that take into account the needs and perceptions of stakeholders of the region. In Chapter 2, we take a first step toward deriving the basis for such community-driven scenarios by presenting the results of a community workshop organized in Montérégie between multiple stakeholders. We use those results to derive simple land use change scenarios and use those scenarios with the integrated model from Chapter 1. These scenarios can serve as a basis on which to iterate a community-driven process of conservation scenario building. We derive the same analysis for our species of interest than in Chapter 1 and compare and contrast the results between scenarios.}
\Link

%  Chapter 2
% !TEX root = ./thesis.tex
\chapter{Integrating stakeholders perceptions of connectivity conservation priorities into a spatially explicit land use and connectivity change model}
\begin{center}
{Valentin Lucet$^{1}$, Andrew Gonzalez$^{1}$}\\
\end{center}
\textit{Author Affiliations:}\\
\normalsize{$^{1}$Department of Biology, McGill University}\\

\section{Abstract}

\section{Introduction}
%\textit{Problem statement}: Connectivity conservation planning methods do not account for stakeholder perceptions. This can potentially lead to ill-informed conservation plans with low chances of success.
%
%\textit{Research questions}: \textbf{How do different stakeholders perceive connectivity conservation priorities, given the obstacles and opportunities for land use planning apparent in the region, and which conservation scenarios can be derived from these perceptions?} \\

%-----------------------------------------------
Connectivity conservation methods usually entails modelling connectivity of the landscape of interest and using a prioritization method to determine conservation priorities. However, current connectivity conservation planning methods fail to confront the results of the prioritization with the priorities perceived by stakeholders.

Failing to integrate the perceptions of stakeholders is detrimental to conservation for two reasons: first, it most likely mean that conservation will fail to gather enough local momentum to lead to actual policy change, and second, it means that the only tool for decision making will be the modelisations, whereas these are incomplete representation of the landscape and would benefit from inputs from stakeholders. This is especially true in landscapes where a considerable effort of connectivity modelling has already been conducted, like in the landscape of interest in this thesis, the southern Quebec region of Montérégie.

Connectivity conservation is often faced with the issue of understanding the processes driving land use change (Worboys et al. 2010). Because land use change is a social process with consequences of both social and ecological nature, it is best understood within the concept of social-ecological system. A social-ecological system (SES) can be understood as the set of human and non-human actors, the set of natural habitats they inhabit and resources and use, and the set of interactions that are maintained between all the components of the system (Ostrom 2009). SESs thus form complex and integrated aggregates of interactions (Hinkel et al. 2014). Those interactions also impact governance, the process by which actors in power establish rules and laws (Bissonnette et al. 2018).

Our understanding of SESs often lacks two important elements: social realism and spatial explicitness.   We need to build more realistic models for social-ecological systems by being more spatially explicit about the obstacles and opportunities presented by conservation as a land use type. Understanding how these obstacles and opportunities can influence management decisions is crucial for our understanding of connectivity conservation planning, where land-use conflicts can hinder the protection and restoration of connectivity. Although it is relatively easy to identify where land-use changes might conflict with connectivity conservation, evaluating to which extent these conflicts matter in a local conservation context is more difficult (Mitchell et al. 2015).

This study aims to start filling the gap identified by conducting participatory research in Montérégie, in south Quebec, where connectivity conservation has become an important stake. This project has been developed in collaboration with the non-profit NAQ (Nature action Quebec). NAQ invited the QCBS to contribute to their connectivity conservation project (called PADF, or Plan d’Aménagement Durable des Forêts) by co-developing the research described here.

The goal of this chapter is therefore twofold: gain a better understanding of conservation priorities in the region and then compare them to the results of chapter 1. We use a workshop and GIS as the main methodological tool.\\
%-----------------------------------------------

\section{Methods}

In order to understand what stakeholders perceive as a conservation priority, and derive conservation scenarios from those perceptions, we conducted a community workshop. The workshop was approved by McGill University’s Research Ethics Board (see appendix {x} for the approval letter). The workshop was held on January 22nd 2020 in the public library of Saint-Jean sur Richelieu. \\

\subsection{Community workshop}

The workshop aimed to gather stakeholders from multiple groups:
\begin{enumerate}
  \item Representatives of Non-Governmental Organizations that are involved in the conservation of ecological connectivity within the study extent (i.e. Montérégie).
  \item Land use planners (“amenagistes”) of the administrative regions covered by the study extent (the 15 MRCs in Montérégie).
  \item Representatives of Ministries involved in conservation (MFFP, MELCC).
  \item Representatives of the UPA (“Union des producteurs agricoles”) in Montérégie.
  \item Representatives of the private forestry industry unions (“producteurs forestiers”)
\end{enumerate}
Of all the groups, only the last group was not represented. \\

\subsubsection{Consensual mapping}

We employed a method coined “consensual participatory mapping in geographically structured focus groups”. Participants whose organisations operate in the same region (the same MRC) were seated at the same table. See table \ref{tab:workshoptables} for the breakdown of each MRC by table. In addition, for participants whose zone of influence or zone of action covered the whole region, two “transversal tables” were created. In the subsequent sections, the workshop results are divided between the regional and transversal tables.
The general method proceeded in 3 exercises, each step involving the same focus groups.

\begin{itemize}
  \item Exercise 1: mapping of forested cores of importance for ecological connectivity.
  \item Exercise 2: mapping of obstacles (for example: specific land use types, development projects in progress, local legislation) and opportunities for habitat connectivity in terms of social-economic activity and land use.
  \item Exercise 3: mapping of links (already recognized or potential) of importance between forested cores of importance, taking into account obstacles and opportunities on the landscape.
\end{itemize}

Each of these three exercises was conducted in 4 to 5 steps. Here we use exercise 1 to describe the method in more detail at each step.

\begin{enumerate}
\item Individual reflection - each participant thought on their own about the problem at hand. For instance, participants took time to think about what forested cores are important to connect in the landscape
\item Group discussion, at each table (i.e. in each focus group) each participant contributed their answer to the question posed by the exercise. For instance, participants shared what forested cores they found to be important.
\item Group discussion on the criteria that each participant used to answer the question. For instance, participants said why they thought they chose these forested cores.
\item Consensus building, participants voted for the most important criteria. For instance, participants used stickers that identify their group affiliation to vote for the criteria they found most important.
\item Room discussion: all participants exchanged on their decisions by sharing the results of their table’s work to the rest of the room.
\end{enumerate}

Each participant received a unique and neutral identifier of the form Affiliation-Geography. For example, a (hypothetical) land use planner from the table that brought together participants from the Maskoutains region received the code [A-1-1], and the UPA representatives from the Richelieu region received the codes [C-3 -1] and [C-3-2]. In addition, for certain activities, the affiliation of the participants is color-coded. For instance, blue for the land use planners and purple for the ministry representatives.

The coding system manifests itself in multiple ways, depending on the activity:
\begin{itemize}
\item When participants engage in activities involving drawing, the materials on which they draw will bear the aforementioned coding (A-1 etc..).
\item When participants engaged in activities involving post-its, the color of their post-it represented their affiliation or bore the aforementioned coding, depending on the activity.
\item When participants engaged in activities involving a weighting of choices with stickers, the color of their post-it or sticker represented their affiliation or had the aforementioned coding, depending on the activity.
\end{itemize}

To ensure that this system is used consistently throughout the workshop, each participant was given a personal folder with their own color-coded stickers and post-its. Participants were instructed to only use the stickers and material that have been personally handed to them. All participants were given a visual aid to remind them of the agenda of the workshop. Facilitators were also present at almost every table and helped facilitate the discussion. They were given a document to help them in this role. All workshop materials, including the ethics approval document for this research project can be found in Appendix 2.\\

\subsubsection*{Weighting system}

 The participants were given a total of 6 stickers to cast their vote for the opportunities and obstacles that they thought were most important. They were given a total of 6 stickers: 2 blue (important), 2 yellow (very important) and 2 red (of the first importance).\\

\subsubsection{Data processing and analysis}

\subsubsection*{Voting: opportunity and challenges}

The data from the opportunities and challenges activity (with post-its), and the weighting of these elements were treated with the following steps:
\begin{enumerate}
  \item Each post-it (opportunity or challenge) received a unique ID.
  \item A first score is calculated by summing the points for each sticker (each vote)
  \begin{itemize}
      \item 1 point for a blue vote
      \item 2 points for a yellow vote
      \item 3 points for a red vote
  \end{itemize}
  \item This score was then weighted by multiplying it by the post-it’s diversity score which served as a measure of consensus
\begin{itemize}
\item This diversity score is the inverse simpson index (R package Vegan). The more diverse the group the higher this index is.
\end{itemize}
\end{enumerate}
These steps were carried out for each table and the results were treated separately for each table because not all tables had the same potential for diversity. For each table, the 10 post-its with the highest scores were retained, 5 among the post-its that had been placed on a specific area of the map (spatialized) and 5 for those that were not. The list of post-it contents, along with the scores, can be found in the appendix tables \ref{tab:opp_chall_ns} and \ref{tab:opp_chall_s}. In addition, a map synthetic map of the spatialized post-its was produced (see figures \ref{fig:reg_AC} and \ref{fig:trans_AC}).\\

\subsubsection*{Digitization of results}

The priority areas and links were digitized by hand in QGIS. The smooth tool was used to generalize the trace of each area and corridors (see figures \ref{fig:reg}  and \ref{fig:trans} for the final results). It is important to note that those maps are an approximate representation of reality, as the drawings done during the activity were an approximation of what the participant had in mind.

In order to integrate the results with the land use change model, we added a 5 km buffer around the linkages to simulate a corridor effect. This corridor width size is arbitrary and is in most cases  an unrealistic expectation forhow large a wildlife corridor can or should be. \\

\subsubsection{Conservation and land use change scenarios}

The conservation scenarios were derived with the goal to demonstrate how results of this type of community workshop can be integrated with other planning methods for connectivity conservation, such as land use change and connectivity modelling. They are not meant to be directly used as conservation guidelines. They are unrealistic by definition and are meant to reflect the most extreme bounds of connectivity change and/or protection measures imaginable for the region. More realistic scenarios featuring targeted action would need to be devised in future community workshops. Although there exists a tremendous amount of work on scenario-building for conservation, specific examples for connectivity conservation are rare. The scenarios presented here are a “proof of concept” for the integration of qualitative results into a quantitative model. 

We integrated the results of the workshop with the land use change and connectivity model devised in Chapter 1 by changing two attributes of the model: \textit{conservation status} and \textit{reforestation location}. These additional conservation scenarios are crossed with the climate change scenarios.

Conservation status determines whether or not forest can be cut down, and whether pixels can transition into urban and agricultural land. In Chapter 1, we gave this status to areas that were previously known as protected, but assumed no more forest would be added to this set of protected areas. In this chapter, a protection status is given to all pixels within the identified priority areas and within a 5km buffer around the identified priority linkage areas. All pixels under this status cannot transition into urban land and remain in their state. This represents an unrealistic proportion of the region under protection (about 33% of the region)

The reforestation status modified what zones can be targeted for reforestation. In reforestation scenarios in Chapter 1, reforestation happened randomly (barring some neighboring rules). In this chapter, reforestation is targeted to the same areas identified in the workshop (all pixels within the identified priority areas and within a 5km buffer around the identified priority linkage areas). As in the model described in Chapter 1, all new reforested area takes the value of medium aged (30-50 yr) deciduous forest and remain in this state for the rest of the simulation. As previously mentioned, this assumption is necessary because LANDIS surface types are not available for newly forested areas. For the same reason, no forest transitions are allowed for this forest type. These two assumptions are conservative:  we do not impose succession or disturbance on these newly forested ares and assume medium age as an average age class.

The three modified conservation and land use change scenarios are therefore:
\begin{enumerate}
  \item Business as usual land use change + corridor protection \textbf{(BAU-Corr)}
 \item Business as usual land use change + corridor protection + reforestation \textbf{(BAU-R-Corr)}
 \item Business as usual land use change + corridor protection + targeted reforestation \textbf{(BAU-R(T)-Corr)} \\
\end{enumerate}

\subsubsection{Connectivity analyses}

In order to be able to compare the results of these new scenarios with those developed in Chapter 1, we conducted the same connectivity analyses using the Circuitscape software. The software “borrows algorithms from electronic circuit theory to predict connectivity in heterogeneous landscapes''. It is a free and open software under MIT license developed originally in Python, and now in its 5th version, in Julia \citep{circuitjulia}.

For each of the maps produced in the suitability analysis, Circuitscape was run in two directions (“wall to wall” run, \cite{mcrae_conserving_2016}): east to west and north to south. This method allowed us to model omnidirectional animal movements, and required that we added the resulting North/South and East/West flow map. The results for each map were therefore added to produce a final set of flow maps for analysis. From these flow maps, we extracted the mean flow for each of the municipalities and for the entire flow map. 

In addition, similarly to in Chapter 1 we extracted the distribution (logged) of pixel flow values under the form of histograms. We also ran the SURF analysis in the same way and counted up the number detected feature for each surface, in order to derive a measure of flow complexity.\\

\section{Results}

\subsection{Community workshop}

\subsubsection{Forested cores of importance}  

All tables were able to quickly build good consensus as to which forested cores was to be linked in priority. The participants at regional (figure \ref{fig:reg_AC}) and transversal (figure \ref{fig:trans_AC}) tables, identified redundant groups of forest patches. Because transversal tables covered the whole region, they often identified cores that engulfed the cores identified by the regional table, giving us an idea of what could constitute regional versus sub-regional conservation priorities.

Among the cores identified, a lot of the currently protected areas in the region were circled, for example the protected areas of the Green Mountains in the MRC of Brome-Missisquoi, or parts of the Monteregian Hills such as the Mount Saint-Hilaire. %TODO maybe calculate the exact % of overlap?

\subsubsection{Obstacles and opportunities}

The tables identified a total of 63 items as opportunities or assets, and 56 as constraints or obstacles. Constraints scored higher, on average, than assets, indicating a focus of participants on constraints. Tables \ref{tab:opp_chall_s} and \ref{tab:opp_chall_ns} displays the 5 assets or obstacles that scored the highest for each table. We divided the results into elements that had been spatialized (i.e. placed on a specific location of the map during the exercise), and non spatialized (i.e. not placed on a specific location ). Mapped elements should reflect a localized realities for connectivity conservation, whereas non mapped elements should speak to more general and large scale aspects. The elements in the spatialized set scored higher on average than in the non spatialized set, indicating that participants tended to focus on more tangible and local issues ans assets.

Among non-spatialized elements, the assets that scored the highest varied according to the sub-region under consideration. In the Centre table participants emphasized the strong presence of environmental NGOs, whereas the North and East tables emphasized elements linked to forest management. Interestingly, the presence of a strong agricultural matrix in the region scored high as an asset at the Transversal tables, and very high as a constraint on a regional table, highlighting the polarizing status of the agricultural land use type in the region. Another constraint linked to agriculture is the CPTAQ - an agency responsible for the protection of agricultural lands often involved in discussions around land use planning - mentioned for the Centre sub-region. Similarly, the East table mentions the issues that farmers have to make their farm profitable as another constraint. Besides agriculture, participants emphasized other constraints, such as economic realities making conservation a less likely investment, given the cost of implementing connectivity conservation measures - also mentioned as a perceived constraint. Finally, the fact that the overwhelming majority of forests are situated on private lands, along with the existing road infrastructure, deserved to be mentioned.

The spatialized elements are summarized in maps \ref{fig:trans_AC} and \ref{fig:reg_AC}, which shows the single highest scoring element (asset or constraints) for each sub region. In terms of assets, the Centre, East and North tables mentioned positive local inclinations toward conservation. Similarly, the West and Transversal tables mentioned local legislation and the active mobilization of stakeholders, highlighting that connectivity conservation is a concern for many stakeholders in the region. Concerning constraints, participants emphasized once again elements related to the predominance of agriculture in the region. The lack of engagement from farmers in mentioned in the Centre and West tables, along with land use "pressures" coming both from agriculture and urban spread. The East table emphasized on the similar "villegiative" pressures, along with more local concern such as the Highway 10 and the intensity of local agriculture practices. The North table also mentions its highways. The transversal tables mentions the price of agricultural lands and the need for compensating the loss of agricultural production due to conservation. \\

\subsubsection{Links of importance at the regional scale}

\subsubsection*{Factors of Priority}

Among the factors that were designated as a priority for the identification of links, elements that came up the most were tied to the characteristics of the natural environments that make up the nodes that are to be connected. The "quality" of these environments and of the nodes themselves seemed important to workshop participants: for instance, the protection status of habitats, their diversity (variety of habitats), their surface area, isolation or proximity, and the pre-existence of connectivity potential. This highlights that stakeholders agree on the value of high quality habitats and on the need to protect them.

In addition, there was much discussion on the dual role of both "facilitating" and "constraining" factors in deciding whether a link is a priority, and as to which type should be given more weigth. Facilitating factors include the feasibility of establishing the link, social facilitators such as political and social buy-in, and territorial facilitators, such as appropriate of land use. There were several types of constraints: financial, social, territorial and physical, and must be considered in relation to the socio-political and spatial planning context in which they emerge. % example?

The presence of species at risk was identified as an important factor, along with other elements such as social factors (available economic means and the level of local interest in conservation) and characteristics of the link itself, including the functionality of the corridor. Consideration of threats such as the level of fragmentation was also noted. Finally, the fact that the environment is to be protected or restored was also deemed important. Interestingly, regulations are among the factors that were cited less often, such as Quebec's PRMHH ("Plans régionaux des milieux humides et hydriques", or regional wetlands conservation plans), and migration routes. \\ 

\subsubsection*{Priority link mapping}

As for the forested cores, there was overall good consensus on what links should be prioritized for all groups. The consensual mapping process identified important links between the key areas identified in the previous steps at both regional (figure \ref{fig:reg}) and transversal tables (figure \ref{fig:trans}). At regional tables, a first comment can be made about the choices in corridor locations. Montérégie is a heavily fragmented region, with lots of small patches. This leads to what could be called "corridor opportunism". Although it was made clear that the goal was not to design a precise path for each corridor, we see that participants wanted to link areas by linking patches they judged to be organized in a linear or corridor-like fashion. This demonstrates that participants saw small patches as important for corridor design and perceived corridors as an important tool to conserve and restore the potential of fragmented patches. However, it also means that designing corridors in such a way  might mean that an otherwise important patch could be ignored because its location does not fit a linear corridor design.

A second comment can be made about corridor redundancy. Participants were ask to limit themselves in the amount of links they were to prioritize, but were not given an exact number. Only one table (North) felt that corridor redundancy was important enough to include redundant design in their priorities. It can be clearly seen in figure \ref{fig:reg} most participants decided to sacrifice redundancy for parsimony, as only the northern part of the region displays redundant (yet arguably still parsimonious), corridor design.

As for the forested cores, there was a lot of redundancy between the regional and the transversal tables concerning the links identified, there was also some differences and complementarity (see figure \ref{fig:trans}). Because the transversal tables covered the entire region, they were able to identify links at a larger scale. This allowed the participants to designated\ a larger scale link along the Saint Lawrence River, as well as transversal links between the lake Champlain region and the easternmost parts of the region, and between the Saint-Francois wildlife area and mount Rigaud, mentioning that this last link would need to be implemented in collaboration with Ontario. It was clear that participants at transversal tables felt that a broader discussion of links with outside of the region was necessary. \\

\subsection{Conservation scenarios}
% Similar to Chapter 1 results section

\subsubsection{Land Use Change Model}
% Go throught each

\subsubsection{Connectivity Modelling}

\vspace{1em}

\subparagraph*{\textit{Current flow}} 

\vspace{1em}

\subparagraph*{\textit{Flow distributions}}

\vspace{1em}

\subparagraph*{\textit{Feature detection}} 

\section{Discussion}

%TBC

\section{Conclusion}

% To rewrite

% At this step of completion of the project, there remains many steps with regard to data processing, analysis and presentation that still needs to be answered. Concerning data processing and model proofing, chapter 1 remains very weak in its capacity to evaluate how much the Random Forest - Cellular Automata model can be trusted to provide a good predictor of land use change. This is due in part to the model fitting process which so far relies on practices for which we should perhaps seek an alternative such as mean imputation and raster aggregation. Concerning data analysis and presentation, the main issues are in chapter 2, where the methods of analysis of the workshop data remain to be fully determined. The method for comparing  connectivity outputs with community-produced corridors also remains to be decided. That being said, we are on our way to provide an answer to the problems identified in the problem statement of each chapter.\\

%---------------------------------------------------------------------------------------------------------------------------------------------------

\newpage
\begin{center}
\section*{Figures \& Tables}
\end{center}

%---------------------------------------------------------------------------------------------------------------------------------------------------
% tables (moved to appendix)

%---------------------------------------------------------------------------------------------------------------------------------------------------
% figures

% AC

\begin{figure}[h!]
\makebox[\textwidth]{
\includegraphics[width=1.3\textwidth]{figures/Regional_AC.png}
}
\caption{Cartography of opportunity and challenges  with the highest importance and diversity metric (Regional tables).}
\label{fig:reg_AC}
\end{figure}
\clearpage

\begin{figure}[h!]
\makebox[\textwidth]{
\includegraphics[width=1.3\textwidth]{figures/Transversal_AC.png}
}
\caption{Cartography of opportunity and challenges  with the highest importance and diversity metric (Transversal tables).}
\label{fig:trans_AC}
\end{figure}
\clearpage

%---------------------------------------------------------------------------------------------------------------------------------------------------
% Links and areas

\begin{figure}[h!]
\makebox[\textwidth]{
\includegraphics[width=1.3\textwidth]{figures/Regional.png}
}
\caption{Cartography of priority areas and links (Regional tables).}
\label{fig:reg}
\end{figure}
\clearpage

\begin{figure}[h!]
\makebox[\textwidth]{
\includegraphics[width=1.3\textwidth]{figures/Transversal.png}
}
\caption{Cartography of priority areas and links (Transversal tables).}
\label{fig:trans}
\end{figure}
\clearpage

%---------------------------------------------------------------------------------------------------------------------------------------------------
% Flow

% Linear
\begin{figure}[h!]
\makebox[\textwidth]{
  \includegraphics[width=\textwidth]{figures/connectivity_decrease_x5species_chap2.png}
}
 \caption{Change in mean flow (in \% of the the 2010 flow) between 2010 and 2100, contrasting BAU scenario (solid line) with conservation scenarios.}
 \label{fig:flow_linear_2}
\end{figure}

% Radar
\begin{figure}[h!]
\makebox[\textwidth]{
  \includegraphics[width=\textwidth]{figures/radar_ggradar_chap2.png}
}
 \caption{Change in mean flow (in \% of the the 2010 flow) between 2010 and 2100, contrasting BAU scenario (solid line) with conservation scenarios.}
 \label{fig:flow_radar_2}
\end{figure}

% Radar All
\begin{figure}[h!]
\makebox[\textwidth]{
  \includegraphics[width=\textwidth]{figures/radar_ggradar_both.png}
}
 \caption{Change in mean flow (in \% of the the 2010 flow) between 2010 and 2100, contrasting BAU scenario (solid line) with both other land use change and conservation scenarios.}
 \label{fig:flow_radar_both}
\end{figure}

% Histograms
\begin{figure}[h!]
\makebox[\textwidth]{
  \includegraphics[width=\textwidth]{figures/hist_chap2.png}
}
 \caption{Histograms of flow values change.}
 \label{fig:hist_2}
\end{figure}

%---------------------------------------------------------------------------------------------------------------------------------------------------
% SURF

% Linear
\begin{figure}[h!]
\makebox[\textwidth]{
  \includegraphics[width=\textwidth]{figures/surf_chap2.png}
}
 \caption{Change in identified features (partly pinch-points.}
 \label{fig:surf_linear_2}
\end{figure}

% Radar
\begin{figure}[h!]
\makebox[\textwidth]{
  \includegraphics[width=\textwidth]{figures/surf_radar_chap2.png}
}
 \caption{Change in identified features (partly pinch-points.}
 \label{fig:surf_radar_2}
\end{figure}


% General Discussion
% \include{discussion}
\SetLinkName{{General Discussion \& Conclusion}}%
% In case of a manuscript-based thesis the comprehensive discussion should encompass all of the chapters of the thesis and should not be a repetition of the individual chapters.
\SetLinkText{To be completed.
}%
\Link%

% Appendix
%\ETDAppendix{Chapter I Supplementary Material}{
%% !TEX root = ./thesis.tex

\chapter*{\textbf{Chapter I Supplementary Material \\ \hspace{1em}}}
\addcontentsline{toc}{chapter}{Chapter I Supplementary Material}

\setcounter{chapter}{3}
\setcounter{table}{0}
\setcounter{figure}{0}

%---------------------------------------------------------------------------------------------------------------------------------------------------
% METHODS CHAP1

%\begin{landscape}
\begin{longtable}[c]{|p{5cm}|p{11cm}|}
%\centering
\caption{Description of the 5 species used in the connectivity model (taken from Rayfield et al. 2018)}
\label{tab:species} \\
%\begin{tabular}{|p{5cm}|p{10cm}|}
\hline
\hline
\textbf{Species} & \textbf{Description} \\ \hline
Northern short-tailed Shrew \newline \textit{Blarina brevicauda} & Abundant small fossorial mammal. This highly active species can live in a diversity of habitats (grasslands, old fields, marshy areas, gardens, and some developed areas) but is mainly found in deciduous and mixed old forests with thick understories that provide good cover for hiding from predators. It feeds primarily on earthworms found in areas with moist soils. It has a high reproductive rate and is generally a poor disperser although it can cross gaps of 50-100m. \\ \hline
American Marten \newline \textit{Martes americana} & Small vagile carnivorous predator. Found in core areas of dense (\textgreater{}60\% cover) and old (\textgreater{}70 yr) coniferous or mixed forests with complex vertical and horizontal structure. It requires large home ranges (above hundreds of ha). It generally avoids large openings and clearings (above few hundred meters wide) but crosses roads and frozen rivers easily. Deep persistent snow pack is a habitat critical element as it excludes predators (Canis latrans) and competitors (Martes pennanti) and provides good hunting conditions. This forest specialist is particularly sensitive to human activities. Juveniles are able to cover tens of kilometers when dispersing, more than what would be expected from body mass-based estimates. Trapped for its fur, this species has patrimonial and economical importance. \\ \hline
Red-Backed Salamander \newline \textit{Plethodon cinereus} & Terrestrial salamander. This sedentary and territorial forest-dwelling species lives under the leaf litter or coarse woody debris in mature and moist deciduous and mixed forests. It is a poor disperser that uses tens of square meters as a home range and rarely ventures more than 50 m in open fields. Roads and edges (up to 20-30 m) have a negative effect on populations’ densities and reduce individual movements. \\ \hline
Wood Frog \newline \textit{Rana sylvatica} & Forest-specialist amphibian. This species prefers mixed and coniferous stands with closed canopy (\textgreater 40\%) and moist soil covered with woody debris (for egg deposition) but can adapt to other closed habitats. Both aquatic (palustrine, fish-free wetlands, not open and permanent ones) and terrestrial habitats are essential, and distance between both should not exceed ca. 600 m. It is sensitive to forest edge (25-35 m), gaps, high intensity agriculture, human developments and recent clearcuts that can act as barriers to movement. This poor disperser is particularly sensitive to roads and habitat loss (fidelity to first breeding pond). \\ \hline
Black Bear \newline \textit{Ursus americanus} & Large opportunistic omnivorous mammal. This species likes deciduous and mixed mature forest with dense cover interspersed with small clearings and early-successional stages of forest that are rich in berry production (depends on main soil surface deposit). It uses broad territories (tens to hundreds of square km) to follow fruiting season by going upslope with the season. It is an effective seed disperser because it can travel long distances (up to 390 km), in particular male juveniles. It clearly avoids human activity (up to 5 km) and roads (up to 800 m), in particular highways, and is likely to take a detour instead of crossing a 60 m gap. \\
\hline
\hline
%\end{tabular}
\end{longtable}
%\end{landscape}

%---------------------------------------------------------------------------------------------------------------------------------------------------
% RF

% RF variables
\begin{table}[h!]
\centering
\caption{Description and data sources for all variables used in the RF-CA model}
\label{tab:variables}
\begin{tabular}{lll}
\hline
\textbf{Variable} & \textbf{Format} & \textbf{Source}  \\ 
\hline
Distance from urban land & \multirow{3}{*}{Raster} & \multirow{2}{*}{Generated from land cover data} \\ \cline{1-1}
Size of forest patch &  &  \\ \cline{1-1} \cline{3-3} 
Elevation &  & \begin{tabular}[c]{@{}l@{}}SRTM 30m from \\ Google Earth Engine \\ data library\end{tabular} \\ \hline
Population change & \multirow{2}{*}{\begin{tabular}[c]{@{}l@{}}Tabular data joined to vector \\ data and rasterized\end{tabular}} & \multirow{2}{*}{\begin{tabular}[c]{@{}l@{}}Canadian Census for \\ 1991, 2001 and 2011\end{tabular}} \\ \cline{1-1}
Income &  &  \\ 
\hline
\end{tabular}
\end{table}

%---------------------------------------------------------------------------------------------------------------------------------------------------
% STSIM

% Neighborhood rules

\begin{table}[h!]
\centering
\caption{Neighborhood rules for ST-Sim transitions}
\label{tab:neigh_rules}
\begin{tabular}{llcc}
\hline
\textbf{Transition} & \textbf{State counted} & \textbf{Neighborhood (m)} & \textbf{Minimum Proportion} \\ \hline
Urbanization & Urban & 500 & 0.7 \\
Agricultural Expansion & Agriculture & 250 & 0.9 \\
Reforestation & Forest (any type) & 250 & 0.9 \\
Forest Internals & Forest (specific type) & 225 & 0.15 \\ \hline
\end{tabular}
\end{table}

%---------------------------------------------------------------------------------------------------------------------------------------------------
% Connectivity analyses

% Habitat suitability 

\begin{table}[h!]
\centering
\caption{Pixel suitability based on forest types and age.}
\label{tab:suit_pixls}
\begin{tabular}{lccccccccc}
\cline{2-10}
 & \multicolumn{9}{c}{\textbf{Forest}} \\ \cline{2-10} 
\textbf{} & \multicolumn{3}{c}{\textbf{Deciduous}} & \multicolumn{3}{c}{\textbf{Mixt}} & \multicolumn{3}{c}{\textbf{Coniferous}} \\ \hline
\textbf{Species} & \textbf{Young} & \textbf{Medium} & \textbf{Old} & \textbf{Young} & \textbf{Medium} & \textbf{Old} & \textbf{Young} & \textbf{Medium} & \textbf{Old} \\ \hline
\textit{\begin{tabular}[c]{@{}l@{}}Blarina \\ brevicauda\end{tabular}} & 0 & 0.5 & 1 & 0 & 0.5 & 1 & 0 & 0 & 0 \\ \hline
\textit{\begin{tabular}[c]{@{}l@{}}Martes \\ americana\end{tabular}} & 0 & 0 & 0 & 0 & 1 & 1 & 1 & 1 & 1 \\ \hline
\textit{\begin{tabular}[c]{@{}l@{}}Plethodon \\ cinereus\end{tabular}} & 1 & 1 & 1 & 1 & 0.5 & 1 & 0 & 0 & 0 \\ \hline
\textit{\begin{tabular}[c]{@{}l@{}}Rana \\ sylvatica\end{tabular}} & 1 & 1 & 1 & 1 & 1 & 1 & 1 & 1 & 1 \\ \hline
\textit{\begin{tabular}[c]{@{}l@{}}Ursus\\ americanus\end{tabular}} & 1 & 1 & 1 & 0.5 & 0.5 & 0.5 & 1 & 1 & 1 \\ \hline
\end{tabular}
\end{table}

% Reclassification into resistance

\begin{table}[h!]
\centering
\caption{Resistance reclassification key for non-forest pixels.}
\label{tab:key_non_forest}
\begin{tabular}{lcccccc}
\hline
\textbf{Species} & \textbf{Urban land} & \textbf{Roads} & \textbf{Agricultural land} & \textbf{Wetlands} & \textbf{Water} & \textbf{Other} \\ \hline 
\textit{\begin{tabular}[c]{@{}l@{}}Blarina \\ brevicauda\end{tabular}} 	&  32 & 32 &	8    & 8 & 16 & 8\\ \hline
\textit{\begin{tabular}[c]{@{}l@{}}Martes \\ americana\end{tabular}} 	&  32 & 32 &	16  & 8 &	 16 & 8\\ \hline
\textit{\begin{tabular}[c]{@{}l@{}}Plethodon \\ cinereus\end{tabular}} 	&  32 & 32 &	8    & 8 & 32 & 8	\\ \hline
\textit{\begin{tabular}[c]{@{}l@{}}Rana \\ sylvatica\end{tabular}} 			&  32 & 32 &	8    & 2 & 8   &	 8	\\ \hline
\textit{\begin{tabular}[c]{@{}l@{}}Ursus\\ americanus\end{tabular}} 		&  32 & 32 &	16  & 2 &	 16 & 8\\ \hline
\end{tabular}
\end{table}

\clearpage

% Minimum patch

\begin{table}[h!]
\centering
\caption{Minimum patch size for suitability of a patch in the habitat suitability analysis.}
\label{tab:patch_size}
\begin{tabular}{lc}
\hline
\textbf{Species} 																									& \textbf{Minimum patch size} (hectares) \\ \hline 
\textit{\begin{tabular}[c]{@{}l@{}}Blarina \\ brevicauda\end{tabular}} 	&  1		\\ \hline
\textit{\begin{tabular}[c]{@{}l@{}}Martes \\ americana\end{tabular}} 	&  150		\\ \hline
\textit{\begin{tabular}[c]{@{}l@{}}Plethodon \\ cinereus\end{tabular}} 	&  1		\\ \hline
\textit{\begin{tabular}[c]{@{}l@{}}Rana \\ sylvatica\end{tabular}} 			&  1		\\ \hline
\textit{\begin{tabular}[c]{@{}l@{}}Ursus\\ americanus\end{tabular}} 		&  1200		\\ \hline
\end{tabular}
\end{table}

% Key patches type

\begin{table}[h!]
\centering
\caption{Resistance reclassification key for different forest patches types.}
\label{tab:hab_or_not}
\begin{tabular}{lccc}
\hline
\textbf{Species} & \textbf{Habitat patch} & \textbf{Habitat patch - too small} & \textbf{Non-habitat patch} \\ \hline 
\textit{\begin{tabular}[c]{@{}l@{}}Blarina \\ brevicauda\end{tabular}} 	&	1	&	2	&	4	\\ \hline
\textit{\begin{tabular}[c]{@{}l@{}}Martes \\ americana\end{tabular}} 	&  1	&	4	&	8	\\ \hline
\textit{\begin{tabular}[c]{@{}l@{}}Plethodon \\ cinereus\end{tabular}} 	&  1	&	2	&	4	\\ \hline
\textit{\begin{tabular}[c]{@{}l@{}}Rana \\ sylvatica\end{tabular}} 			&  1	&	2	&	4	\\ \hline
\textit{\begin{tabular}[c]{@{}l@{}}Ursus\\ americanus\end{tabular}} 		&  1	&	4	&	16	\\ \hline
\end{tabular}
\end{table}

\newpage

%---------------------------------------------------------------------------------------------------------------------------------------------------
% RESULTS CHAP1

% VALUES
% Clustering
\begin{figure}[h!]
\centering
 \includegraphics[width=\textwidth]{figures/clustering_values.png}
 \caption{Results of Ward clustering for land use for municipalities (cut at 5 groups)}
 \label{fig:clustervals}
\end{figure}

% TRANSITIONS

% Clustering
\begin{figure}[h!]
  \centering
    \includegraphics[width=\textwidth]{figures/clustering_trans.png}
  \caption{Results of Ward clustering for transition data for municipalities (cut at 4 groups)}
  \label{fig:clustertrans}
\end{figure}

%---------------------------------------------------------------------------------------------------------------------------------------------------
% METHODS CHAP2

% Workshop tables

\chapter*{\textbf{Chapter II Supplementary Material \\ \hspace{1em}}}
\addcontentsline{toc}{chapter}{Chapter II Supplementary Material}

\begin{table}[h!]
\centering
\caption{Opportunity and challenge table (Non Spatial).}
\label{tab:opp_chall_ns}
\begin{tabular}{m{0.15\textwidth}lm{0.5\textwidth}l}
\hline
\textbf{Table} &
  \textbf{\begin{tabular}[c]{@{}l@{}}Atout/Contrainte \\ (Opportunity/Challenge)\end{tabular}} &
  \textbf{\begin{tabular}[c]{@{}l@{}}Contenu\\ (Content)\end{tabular}} &
  \textbf{Score} \\ \hline
\multirow{5}{*}{Centre} &
  \multirow{3}{*}{Atouts} &
  Présence d'organisme environnementaux &
  8.00 \\ \cline{3-4} 
                       &                              & Basses terres: lien prioritaire l’échelle nationale & 4.00  \\ \cline{3-4} 
                       &                              & Programme ALUS                                      & 1.00  \\ \cline{2-4} 
                       & \multirow{2}{*}{Contraintes} & CPTAQ                                               & 18.00 \\ \cline{3-4} 
                       &                              & Besoin de rentabilité des entreprises agricoles     & 5.00  \\ \hline
\multirow{3}{*}{Est}   & Atouts                       & Vocation forestiere existante                       & 8.00  \\ \cline{2-4} 
                       & \multirow{2}{*}{Contraintes} & Tenure privée des terres                            & 2.00  \\ \cline{3-4} 
                       &                              & Plusieurs territoires couverts                      & 1.00  \\ \hline
\multirow{5}{*}{Nord} &
  \multirow{2}{*}{Atouts} &
  Réglementation favorable maintien couvert bois et corridor métropolitain &
  16.20 \\ \cline{3-4} 
                       &                              & PRMHH                                               & 10.80 \\ \cline{2-4} 
                       & \multirow{3}{*}{Contraintes} & Usage agricole prédominant                          & 44.33 \\ \cline{3-4} 
                       &                              & Compréhension et participation citoyenne            & 14.40 \\ \cline{3-4} 
                       &                              & Coûts pour faire de la connectivité                 & 10.80 \\ \hline
\multirow{6}{*}{Ouest} & \multirow{2}{*}{Atouts}      & Routes cours d'eau                                  & 3.00  \\ \cline{3-4} 
                       &                              & Presence de sols pauvre                             & 3.00  \\ \cline{2-4} 
                       & \multirow{4}{*}{Contraintes} & Les réalités économiques                            & 50.00 \\ \cline{3-4} 
                       &                              & Terres privées                                      & 12.00 \\ \cline{3-4} 
                       &                              & LPTAA                                               & 7.20  \\ \cline{3-4} 
                       &                              & Le monde politique                                  & 4.00  \\ \hline
\multirow{6}{*}{Montérégie} &
  \multirow{4}{*}{Atouts} &
  Milieu agricole: potentiel de faire des corridors si un levier est trouvé &
  12.60 \\ \cline{3-4} 
                       &                              & Présence d'acteurs locaux et education              & 8.00  \\ \cline{3-4} 
                       &                              & PRMHH                                               & 3.00  \\ \cline{3-4} 
                       &                              & Permettre l'aménagement forestier                   & 3.00  \\ \cline{2-4} 
                       & \multirow{2}{*}{Contraintes} & Routes et autoroutes                                & 4.00  \\ \cline{3-4} 
                       &                              & Terres privées, présence prédominantes              & 3.00  \\ \hline
\end{tabular}
\end{table}

%---------------------------------------------------------------------------------------------------------------------------------------------------
% RESULTS CHAP2

% Opportunity /challenges

\begin{table}[]
\centering
\caption{Opportunity and challenge table (Spatial).}
\label{tab:opp_chall_s}
\begin{tabular}{m{0.15\textwidth}lm{0.5\textwidth}l}
\hline
\textbf{Table} &
  \textbf{\begin{tabular}[c]{@{}l@{}}Atout/Contrainte \\ (Opportunity/Challenge)\end{tabular}} &
  \textbf{\begin{tabular}[c]{@{}l@{}}Contenu\\ (Content)\end{tabular}} &
  \textbf{Score} \\ \hline
\multirow{5}{*}{Centre} & Atouts                       & Leaders politiques positifs                                  & 33.33          \\ \cline{2-4} 
                        & Les deux                     & Propriétés protégées                                         & 26.67          \\ \cline{2-4} 
                        & \multirow{3}{*}{Contraintes} & Pressions urbaines                                           & 42.86          \\ \cline{3-4} 
                        &                              & Manque d'adhésion des agriculteurs                           & 39.29          \\ \cline{3-4} 
                        &                              & Pressions agricoles                                          & 16.20          \\ \hline
\multirow{5}{*}{Est}          & Atouts                       & Mobilisation projet corridor bleu vert fondation séthy                         & 18.00 \\ \cline{2-4} 
                        & \multirow{4}{*}{Contraintes} & Autoroute 10                                                 & 26.67          \\ \cline{3-4} 
                        &                              & Pressions villegiatives                                      & 22.2           \\ \cline{3-4} 
                        &                              & Fragmentation des habitats liées au dev                      & 18.00          \\ \cline{3-4} 
                        &                              & Activites agricoles intensives                               & 16.2           \\ \hline
\multirow{4}{*}{Nord}         & \multirow{2}{*}{Atouts}      & Municipalite pro-protection des monteregiennes ex. st bruno mont saint hilaire & 8.00  \\ \cline{3-4} 
                        &                              & Comité municipal travail MRC MDY                             & 1.00           \\ \cline{2-4} 
                        & \multirow{2}{*}{Contraintes} & Autoroute 10 20 30 et autres                                 & 1.00           \\ \cline{3-4} 
                        &                              & Gestion de l'application des bandes riveraines               & 1.00           \\ \hline
\multirow{6}{*}{Ouest}  & \multirow{4}{*}{Atouts}      & Proximité des milieux                                        & 24.00          \\ \cline{3-4} 
                        &                              & Article 50.3 du règlement des exploitations agricoles        & 19.20          \\ \cline{3-4} 
                        &                              & Mobilisation des acteurs du milieux                          & 12.00          \\ \cline{3-4} 
                        &                              & Bande riveraine potentielle                                  & 10.00          \\ \cline{2-4} 
                        & \multirow{2}{*}{Contraintes} & Canal beauharnois isole                                      & 24.00          \\ \cline{3-4} 
                        &                              & Réticences de certains producteurs                           & 10.00          \\ \hline
\multirow{5}{*}{Montérégie 1} & \multirow{3}{*}{Atouts}      & Réseaux de sites avec couvert forestier                                        & 14.4  \\ \cline{3-4} 
                        &                              & Grande volonté d'action locale pour créer de la connectivité & 12.60          \\ \cline{3-4} 
                        &                              & Rétrécissement du fleuve                                     & 7.20           \\ \cline{2-4} 
                              & \multirow{2}{*}{Contraintes} & Agriculture intensive compenser la production                                  & 16.20 \\ \cline{3-4} 
                        &                              & Développement urbain                                         & 12.00          \\ \hline
\multirow{5}{*}{Montérégie 2} & \multirow{2}{*}{Atouts}      & Mobilisation sociale organisme conservation sensibilisation                    & 12.00 \\ \cline{3-4} 
                        &                              & Usage des sols favorable                                     & 10.00          \\ \cline{2-4} 
                        & \multirow{3}{*}{Contraintes} & Prix des terres agricoles                                    & 12.00          \\ \cline{3-4} 
                        &                              & Étalement urbain deuxième couronne                           & 10.00          \\ \cline{3-4} 
                        &                              & Pole logistique de transport                                 & 10.00          \\ \hline
\end{tabular}
\end{table}

%}
% !TEX root = ./thesis.tex

\chapter*{\textbf{Chapter I Supplementary Material \\ \hspace{1em}}}
\addcontentsline{toc}{chapter}{Chapter I Supplementary Material}

\setcounter{chapter}{3}
\setcounter{table}{0}
\setcounter{figure}{0}

%---------------------------------------------------------------------------------------------------------------------------------------------------
% METHODS CHAP1

%\begin{landscape}
\begin{longtable}[c]{|p{5cm}|p{11cm}|}
%\centering
\caption{Description of the 5 species used in the connectivity model (taken from Rayfield et al. 2018)}
\label{tab:species} \\
%\begin{tabular}{|p{5cm}|p{10cm}|}
\hline
\hline
\textbf{Species} & \textbf{Description} \\ \hline
Northern short-tailed Shrew \newline \textit{Blarina brevicauda} & Abundant small fossorial mammal. This highly active species can live in a diversity of habitats (grasslands, old fields, marshy areas, gardens, and some developed areas) but is mainly found in deciduous and mixed old forests with thick understories that provide good cover for hiding from predators. It feeds primarily on earthworms found in areas with moist soils. It has a high reproductive rate and is generally a poor disperser although it can cross gaps of 50-100m. \\ \hline
American Marten \newline \textit{Martes americana} & Small vagile carnivorous predator. Found in core areas of dense (\textgreater{}60\% cover) and old (\textgreater{}70 yr) coniferous or mixed forests with complex vertical and horizontal structure. It requires large home ranges (above hundreds of ha). It generally avoids large openings and clearings (above few hundred meters wide) but crosses roads and frozen rivers easily. Deep persistent snow pack is a habitat critical element as it excludes predators (Canis latrans) and competitors (Martes pennanti) and provides good hunting conditions. This forest specialist is particularly sensitive to human activities. Juveniles are able to cover tens of kilometers when dispersing, more than what would be expected from body mass-based estimates. Trapped for its fur, this species has patrimonial and economical importance. \\ \hline
Red-Backed Salamander \newline \textit{Plethodon cinereus} & Terrestrial salamander. This sedentary and territorial forest-dwelling species lives under the leaf litter or coarse woody debris in mature and moist deciduous and mixed forests. It is a poor disperser that uses tens of square meters as a home range and rarely ventures more than 50 m in open fields. Roads and edges (up to 20-30 m) have a negative effect on populations’ densities and reduce individual movements. \\ \hline
Wood Frog \newline \textit{Rana sylvatica} & Forest-specialist amphibian. This species prefers mixed and coniferous stands with closed canopy (\textgreater 40\%) and moist soil covered with woody debris (for egg deposition) but can adapt to other closed habitats. Both aquatic (palustrine, fish-free wetlands, not open and permanent ones) and terrestrial habitats are essential, and distance between both should not exceed ca. 600 m. It is sensitive to forest edge (25-35 m), gaps, high intensity agriculture, human developments and recent clearcuts that can act as barriers to movement. This poor disperser is particularly sensitive to roads and habitat loss (fidelity to first breeding pond). \\ \hline
Black Bear \newline \textit{Ursus americanus} & Large opportunistic omnivorous mammal. This species likes deciduous and mixed mature forest with dense cover interspersed with small clearings and early-successional stages of forest that are rich in berry production (depends on main soil surface deposit). It uses broad territories (tens to hundreds of square km) to follow fruiting season by going upslope with the season. It is an effective seed disperser because it can travel long distances (up to 390 km), in particular male juveniles. It clearly avoids human activity (up to 5 km) and roads (up to 800 m), in particular highways, and is likely to take a detour instead of crossing a 60 m gap. \\
\hline
\hline
%\end{tabular}
\end{longtable}
%\end{landscape}

%---------------------------------------------------------------------------------------------------------------------------------------------------
% RF

% RF variables
\begin{table}[h!]
\centering
\caption{Description and data sources for all variables used in the RF-CA model}
\label{tab:variables}
\begin{tabular}{lll}
\hline
\textbf{Variable} & \textbf{Format} & \textbf{Source}  \\ 
\hline
Distance from urban land & \multirow{3}{*}{Raster} & \multirow{2}{*}{Generated from land cover data} \\ \cline{1-1}
Size of forest patch &  &  \\ \cline{1-1} \cline{3-3} 
Elevation &  & \begin{tabular}[c]{@{}l@{}}SRTM 30m from \\ Google Earth Engine \\ data library\end{tabular} \\ \hline
Population change & \multirow{2}{*}{\begin{tabular}[c]{@{}l@{}}Tabular data joined to vector \\ data and rasterized\end{tabular}} & \multirow{2}{*}{\begin{tabular}[c]{@{}l@{}}Canadian Census for \\ 1991, 2001 and 2011\end{tabular}} \\ \cline{1-1}
Income &  &  \\ 
\hline
\end{tabular}
\end{table}

%---------------------------------------------------------------------------------------------------------------------------------------------------
% STSIM

% Neighborhood rules

\begin{table}[h!]
\centering
\caption{Neighborhood rules for ST-Sim transitions}
\label{tab:neigh_rules}
\begin{tabular}{llcc}
\hline
\textbf{Transition} & \textbf{State counted} & \textbf{Neighborhood (m)} & \textbf{Minimum Proportion} \\ \hline
Urbanization & Urban & 500 & 0.7 \\
Agricultural Expansion & Agriculture & 250 & 0.9 \\
Reforestation & Forest (any type) & 250 & 0.9 \\
Forest Internals & Forest (specific type) & 225 & 0.15 \\ \hline
\end{tabular}
\end{table}

%---------------------------------------------------------------------------------------------------------------------------------------------------
% Connectivity analyses

% Habitat suitability 

\begin{table}[h!]
\centering
\caption{Pixel suitability based on forest types and age.}
\label{tab:suit_pixls}
\begin{tabular}{lccccccccc}
\cline{2-10}
 & \multicolumn{9}{c}{\textbf{Forest}} \\ \cline{2-10} 
\textbf{} & \multicolumn{3}{c}{\textbf{Deciduous}} & \multicolumn{3}{c}{\textbf{Mixt}} & \multicolumn{3}{c}{\textbf{Coniferous}} \\ \hline
\textbf{Species} & \textbf{Young} & \textbf{Medium} & \textbf{Old} & \textbf{Young} & \textbf{Medium} & \textbf{Old} & \textbf{Young} & \textbf{Medium} & \textbf{Old} \\ \hline
\textit{\begin{tabular}[c]{@{}l@{}}Blarina \\ brevicauda\end{tabular}} & 0 & 0.5 & 1 & 0 & 0.5 & 1 & 0 & 0 & 0 \\ \hline
\textit{\begin{tabular}[c]{@{}l@{}}Martes \\ americana\end{tabular}} & 0 & 0 & 0 & 0 & 1 & 1 & 1 & 1 & 1 \\ \hline
\textit{\begin{tabular}[c]{@{}l@{}}Plethodon \\ cinereus\end{tabular}} & 1 & 1 & 1 & 1 & 0.5 & 1 & 0 & 0 & 0 \\ \hline
\textit{\begin{tabular}[c]{@{}l@{}}Rana \\ sylvatica\end{tabular}} & 1 & 1 & 1 & 1 & 1 & 1 & 1 & 1 & 1 \\ \hline
\textit{\begin{tabular}[c]{@{}l@{}}Ursus\\ americanus\end{tabular}} & 1 & 1 & 1 & 0.5 & 0.5 & 0.5 & 1 & 1 & 1 \\ \hline
\end{tabular}
\end{table}

% Reclassification into resistance

\begin{table}[h!]
\centering
\caption{Resistance reclassification key for non-forest pixels.}
\label{tab:key_non_forest}
\begin{tabular}{lcccccc}
\hline
\textbf{Species} & \textbf{Urban land} & \textbf{Roads} & \textbf{Agricultural land} & \textbf{Wetlands} & \textbf{Water} & \textbf{Other} \\ \hline 
\textit{\begin{tabular}[c]{@{}l@{}}Blarina \\ brevicauda\end{tabular}} 	&  32 & 32 &	8    & 8 & 16 & 8\\ \hline
\textit{\begin{tabular}[c]{@{}l@{}}Martes \\ americana\end{tabular}} 	&  32 & 32 &	16  & 8 &	 16 & 8\\ \hline
\textit{\begin{tabular}[c]{@{}l@{}}Plethodon \\ cinereus\end{tabular}} 	&  32 & 32 &	8    & 8 & 32 & 8	\\ \hline
\textit{\begin{tabular}[c]{@{}l@{}}Rana \\ sylvatica\end{tabular}} 			&  32 & 32 &	8    & 2 & 8   &	 8	\\ \hline
\textit{\begin{tabular}[c]{@{}l@{}}Ursus\\ americanus\end{tabular}} 		&  32 & 32 &	16  & 2 &	 16 & 8\\ \hline
\end{tabular}
\end{table}

\clearpage

% Minimum patch

\begin{table}[h!]
\centering
\caption{Minimum patch size for suitability of a patch in the habitat suitability analysis.}
\label{tab:patch_size}
\begin{tabular}{lc}
\hline
\textbf{Species} 																									& \textbf{Minimum patch size} (hectares) \\ \hline 
\textit{\begin{tabular}[c]{@{}l@{}}Blarina \\ brevicauda\end{tabular}} 	&  1		\\ \hline
\textit{\begin{tabular}[c]{@{}l@{}}Martes \\ americana\end{tabular}} 	&  150		\\ \hline
\textit{\begin{tabular}[c]{@{}l@{}}Plethodon \\ cinereus\end{tabular}} 	&  1		\\ \hline
\textit{\begin{tabular}[c]{@{}l@{}}Rana \\ sylvatica\end{tabular}} 			&  1		\\ \hline
\textit{\begin{tabular}[c]{@{}l@{}}Ursus\\ americanus\end{tabular}} 		&  1200		\\ \hline
\end{tabular}
\end{table}

% Key patches type

\begin{table}[h!]
\centering
\caption{Resistance reclassification key for different forest patches types.}
\label{tab:hab_or_not}
\begin{tabular}{lccc}
\hline
\textbf{Species} & \textbf{Habitat patch} & \textbf{Habitat patch - too small} & \textbf{Non-habitat patch} \\ \hline 
\textit{\begin{tabular}[c]{@{}l@{}}Blarina \\ brevicauda\end{tabular}} 	&	1	&	2	&	4	\\ \hline
\textit{\begin{tabular}[c]{@{}l@{}}Martes \\ americana\end{tabular}} 	&  1	&	4	&	8	\\ \hline
\textit{\begin{tabular}[c]{@{}l@{}}Plethodon \\ cinereus\end{tabular}} 	&  1	&	2	&	4	\\ \hline
\textit{\begin{tabular}[c]{@{}l@{}}Rana \\ sylvatica\end{tabular}} 			&  1	&	2	&	4	\\ \hline
\textit{\begin{tabular}[c]{@{}l@{}}Ursus\\ americanus\end{tabular}} 		&  1	&	4	&	16	\\ \hline
\end{tabular}
\end{table}

\newpage

%---------------------------------------------------------------------------------------------------------------------------------------------------
% RESULTS CHAP1

% VALUES
% Clustering
\begin{figure}[h!]
\centering
 \includegraphics[width=\textwidth]{figures/clustering_values.png}
 \caption{Results of Ward clustering for land use for municipalities (cut at 5 groups)}
 \label{fig:clustervals}
\end{figure}

% TRANSITIONS

% Clustering
\begin{figure}[h!]
  \centering
    \includegraphics[width=\textwidth]{figures/clustering_trans.png}
  \caption{Results of Ward clustering for transition data for municipalities (cut at 4 groups)}
  \label{fig:clustertrans}
\end{figure}

%---------------------------------------------------------------------------------------------------------------------------------------------------
% METHODS CHAP2

% Workshop tables

\chapter*{\textbf{Chapter II Supplementary Material \\ \hspace{1em}}}
\addcontentsline{toc}{chapter}{Chapter II Supplementary Material}

\begin{table}[h!]
\centering
\caption{Opportunity and challenge table (Non Spatial).}
\label{tab:opp_chall_ns}
\begin{tabular}{m{0.15\textwidth}lm{0.5\textwidth}l}
\hline
\textbf{Table} &
  \textbf{\begin{tabular}[c]{@{}l@{}}Atout/Contrainte \\ (Opportunity/Challenge)\end{tabular}} &
  \textbf{\begin{tabular}[c]{@{}l@{}}Contenu\\ (Content)\end{tabular}} &
  \textbf{Score} \\ \hline
\multirow{5}{*}{Centre} &
  \multirow{3}{*}{Atouts} &
  Présence d'organisme environnementaux &
  8.00 \\ \cline{3-4} 
                       &                              & Basses terres: lien prioritaire l’échelle nationale & 4.00  \\ \cline{3-4} 
                       &                              & Programme ALUS                                      & 1.00  \\ \cline{2-4} 
                       & \multirow{2}{*}{Contraintes} & CPTAQ                                               & 18.00 \\ \cline{3-4} 
                       &                              & Besoin de rentabilité des entreprises agricoles     & 5.00  \\ \hline
\multirow{3}{*}{Est}   & Atouts                       & Vocation forestiere existante                       & 8.00  \\ \cline{2-4} 
                       & \multirow{2}{*}{Contraintes} & Tenure privée des terres                            & 2.00  \\ \cline{3-4} 
                       &                              & Plusieurs territoires couverts                      & 1.00  \\ \hline
\multirow{5}{*}{Nord} &
  \multirow{2}{*}{Atouts} &
  Réglementation favorable maintien couvert bois et corridor métropolitain &
  16.20 \\ \cline{3-4} 
                       &                              & PRMHH                                               & 10.80 \\ \cline{2-4} 
                       & \multirow{3}{*}{Contraintes} & Usage agricole prédominant                          & 44.33 \\ \cline{3-4} 
                       &                              & Compréhension et participation citoyenne            & 14.40 \\ \cline{3-4} 
                       &                              & Coûts pour faire de la connectivité                 & 10.80 \\ \hline
\multirow{6}{*}{Ouest} & \multirow{2}{*}{Atouts}      & Routes cours d'eau                                  & 3.00  \\ \cline{3-4} 
                       &                              & Presence de sols pauvre                             & 3.00  \\ \cline{2-4} 
                       & \multirow{4}{*}{Contraintes} & Les réalités économiques                            & 50.00 \\ \cline{3-4} 
                       &                              & Terres privées                                      & 12.00 \\ \cline{3-4} 
                       &                              & LPTAA                                               & 7.20  \\ \cline{3-4} 
                       &                              & Le monde politique                                  & 4.00  \\ \hline
\multirow{6}{*}{Montérégie} &
  \multirow{4}{*}{Atouts} &
  Milieu agricole: potentiel de faire des corridors si un levier est trouvé &
  12.60 \\ \cline{3-4} 
                       &                              & Présence d'acteurs locaux et education              & 8.00  \\ \cline{3-4} 
                       &                              & PRMHH                                               & 3.00  \\ \cline{3-4} 
                       &                              & Permettre l'aménagement forestier                   & 3.00  \\ \cline{2-4} 
                       & \multirow{2}{*}{Contraintes} & Routes et autoroutes                                & 4.00  \\ \cline{3-4} 
                       &                              & Terres privées, présence prédominantes              & 3.00  \\ \hline
\end{tabular}
\end{table}

%---------------------------------------------------------------------------------------------------------------------------------------------------
% RESULTS CHAP2

% Opportunity /challenges

\begin{table}[]
\centering
\caption{Opportunity and challenge table (Spatial).}
\label{tab:opp_chall_s}
\begin{tabular}{m{0.15\textwidth}lm{0.5\textwidth}l}
\hline
\textbf{Table} &
  \textbf{\begin{tabular}[c]{@{}l@{}}Atout/Contrainte \\ (Opportunity/Challenge)\end{tabular}} &
  \textbf{\begin{tabular}[c]{@{}l@{}}Contenu\\ (Content)\end{tabular}} &
  \textbf{Score} \\ \hline
\multirow{5}{*}{Centre} & Atouts                       & Leaders politiques positifs                                  & 33.33          \\ \cline{2-4} 
                        & Les deux                     & Propriétés protégées                                         & 26.67          \\ \cline{2-4} 
                        & \multirow{3}{*}{Contraintes} & Pressions urbaines                                           & 42.86          \\ \cline{3-4} 
                        &                              & Manque d'adhésion des agriculteurs                           & 39.29          \\ \cline{3-4} 
                        &                              & Pressions agricoles                                          & 16.20          \\ \hline
\multirow{5}{*}{Est}          & Atouts                       & Mobilisation projet corridor bleu vert fondation séthy                         & 18.00 \\ \cline{2-4} 
                        & \multirow{4}{*}{Contraintes} & Autoroute 10                                                 & 26.67          \\ \cline{3-4} 
                        &                              & Pressions villegiatives                                      & 22.2           \\ \cline{3-4} 
                        &                              & Fragmentation des habitats liées au dev                      & 18.00          \\ \cline{3-4} 
                        &                              & Activites agricoles intensives                               & 16.2           \\ \hline
\multirow{4}{*}{Nord}         & \multirow{2}{*}{Atouts}      & Municipalite pro-protection des monteregiennes ex. st bruno mont saint hilaire & 8.00  \\ \cline{3-4} 
                        &                              & Comité municipal travail MRC MDY                             & 1.00           \\ \cline{2-4} 
                        & \multirow{2}{*}{Contraintes} & Autoroute 10 20 30 et autres                                 & 1.00           \\ \cline{3-4} 
                        &                              & Gestion de l'application des bandes riveraines               & 1.00           \\ \hline
\multirow{6}{*}{Ouest}  & \multirow{4}{*}{Atouts}      & Proximité des milieux                                        & 24.00          \\ \cline{3-4} 
                        &                              & Article 50.3 du règlement des exploitations agricoles        & 19.20          \\ \cline{3-4} 
                        &                              & Mobilisation des acteurs du milieux                          & 12.00          \\ \cline{3-4} 
                        &                              & Bande riveraine potentielle                                  & 10.00          \\ \cline{2-4} 
                        & \multirow{2}{*}{Contraintes} & Canal beauharnois isole                                      & 24.00          \\ \cline{3-4} 
                        &                              & Réticences de certains producteurs                           & 10.00          \\ \hline
\multirow{5}{*}{Montérégie 1} & \multirow{3}{*}{Atouts}      & Réseaux de sites avec couvert forestier                                        & 14.4  \\ \cline{3-4} 
                        &                              & Grande volonté d'action locale pour créer de la connectivité & 12.60          \\ \cline{3-4} 
                        &                              & Rétrécissement du fleuve                                     & 7.20           \\ \cline{2-4} 
                              & \multirow{2}{*}{Contraintes} & Agriculture intensive compenser la production                                  & 16.20 \\ \cline{3-4} 
                        &                              & Développement urbain                                         & 12.00          \\ \hline
\multirow{5}{*}{Montérégie 2} & \multirow{2}{*}{Atouts}      & Mobilisation sociale organisme conservation sensibilisation                    & 12.00 \\ \cline{3-4} 
                        &                              & Usage des sols favorable                                     & 10.00          \\ \cline{2-4} 
                        & \multirow{3}{*}{Contraintes} & Prix des terres agricoles                                    & 12.00          \\ \cline{3-4} 
                        &                              & Étalement urbain deuxième couronne                           & 10.00          \\ \cline{3-4} 
                        &                              & Pole logistique de transport                                 & 10.00          \\ \hline
\end{tabular}
\end{table}


%\ETDAppendix{References}{
%
%\printbibliography
%}
\printbibliography[heading=bibintoc, title={Bibliography \hspace{1em}}]

\end{document}
