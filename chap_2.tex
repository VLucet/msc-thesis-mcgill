% !TEX root = ./thesis.tex
\chapter{Integrating stakeholders perceptions of connectivity conservation priorities into a spatially explicit land use and connectivity change model}
\begin{center}
{Valentin Lucet$^{1}$, Andrew Gonzalez$^{1}$}\\
\end{center}
\textit{Author Affiliations:}\\
\normalsize{$^{1}$Department of Biology, McGill University}\\

\section{Abstract}

\section{Introduction}
%\textit{Problem statement}: Connectivity conservation planning methods do not account for stakeholder perceptions. This can potentially lead to ill-informed conservation plans with low chances of success.
%
%\textit{Research questions}: \textbf{How do different stakeholders perceive connectivity conservation priorities, given the obstacles and opportunities for land use planning apparent in the region, and which conservation scenarios can be derived from these perceptions?} \\

%-----------------------------------------------
Connectivity conservation methods usually entails modelling connectivity of the landscape of interest and using a prioritisation method to determine conservation priorities. However, current connectivity conservation planning methods fail to confront the results of the prioritisation with the priorities perceived by stakeholders.

Failing to integrate the perceptions of stakeholders is detrimental to conservation for two reasons: first, it most likely mean that conservation will fail to gather enough local momentum to lead to actual policy change, and second, it means that the only tool for decision making will be the modelisations, whereas these are incomplete representation of the landscape and would benefit from inputs from stakeholders. This is especially true in landscapes where a considerable effort of connectivity modelling has already been conducted, like in the landscape of interest in this thesis, the southern Quebec region of Montérégie.

Connectivity conservation is often faced with the issue of understanding the processes driving land use change (Worboys et al. 2010). Because land use change is a social process with consequences of both social and ecological nature, it is best understood within the concept of social-ecological system. A social-ecological system (SES) can be understood as the set of human and non-human actors, the set of natural habitats they inhabit and resources and use, and the set of interactions that are maintained between all the components of the system (Ostrom 2009). SESs thus form complex and integrated aggregates of interactions (Hinkel et al. 2014). Those interactions also impact governance, the process by which actors in power establish rules and laws (Bissonnette et al. 2018).

Our understanding of SESs often lacks two important elements: social realism and spatial explicitness.   We need to build more realistic models for social-ecological systems by being more spatially explicit about the obstacles and opportunities presented by conservation as a land use type. Understanding how these obstacles and opportunities can influence management decisions is crucial for our understanding of connectivity conservation planning, where land-use conflicts can hinder the protection and restoration of connectivity. Although it is relatively easy to identify where land-use changes might conflict with connectivity conservation, evaluating to which extent these conflicts matter in a local conservation context is more difficult (Mitchell et al. 2015).

This study aims to start filling the gap identified by conducting participatory research in Montérégie, in south Quebec, where connectivity conservation has become an important stake. This project has been developed in collaboration with the non-profit NAQ (Nature action Quebec). NAQ invited the QCBS to contribute to their connectivity conservation project (called PADF, or Plan d’Aménagement Durable des Forêts) by co-developing the research described here.

The goal of this chapter is therefore twofold: gain a better understanding of conservation priorities in the region and then compare them to the results of chapter 1. We use a workshop and GIS as the main methodological tool.\\
%-----------------------------------------------

\section{Methods}

In order to understand what stakeholders perceive as a conservation priority, and derive conservation scenarios from those perceptions, we conducted a community workshop. The workshop was approved by McGill University’s Research Ethics Board (see appendix {x} for the approval letter). The workshop was held on January 22nd 2020 in the public library of Saint-Jean sur Richelieu. \\

\subsection{Community Workshop}

The workshop aimed to gather stakeholders from multiple groups:
\begin{enumerate}
  \item Representatives of Non-Governmental Organizations that are involved in the conservation of ecological connectivity within the study extent (i.e. Montérégie).
  \item Land use planners (“amenagistes”) of the administrative regions covered by the study extent (the 15 MRCs in Montérégie).
  \item Representatives of Ministries involved in conservation (MFFP, MELCC).
  \item Representatives of the UPA (“Union des producteurs agricoles”) in Montérégie.
  \item Representatives of the private forestry industry unions (“producteurs forestiers”)
\end{enumerate}
Of all the groups, only the last group was not represented. \\

\subsubsection{Consensual mapping}

We employed a method coined “consensual participatory mapping in geographically structured focus groups”. Participants whose organisations operate in the same region (the same MRC) were seated at the same table. See table \ref{tab:workshoptables} for the breakdown of each MRC by table. In addition, for participants whose zone of influence or zone of action covered the whole region, two “transversal tables” were created. In the subsequent sections, the workshop results are divided between the regional and transversal tables.
The general method proceeded in 3 exercises, each step involving the same focus groups.

\begin{itemize}
  \item Exercise 1: mapping of forested cores of importance for ecological connectivity.
  \item Exercise 2: mapping of obstacles (for example: specific land use types, development projects in progress, local legislation) and opportunities for habitat connectivity in terms of social-economic activity and land use.
  \item Exercise 3: mapping of links (already recognized or potential) of importance between forested cores of importance, taking into account obstacles and opportunities on the landscape.
\end{itemize}

Each of these three exercises was conducted in 4 to 5 steps. Here we use exercise 1 to describe the method in more detail at each step.

\begin{enumerate}
\item Individual reflection - each participant thought on their own about the problem at hand. For instance, participants took time to think about what forested cores are important to connect in the landscape
\item Group discussion, at each table (i.e. in each focus group) each participant contributed their answer to the question posed by the exercise. For instance, participants shared what forested cores they found to be important.
\item Group discussion on the criteria that each participant used to answer the question. For instance, participants said why they thought they chose these forested cores.
\item Consensus building, participants voted for the most important criteria. For instance, participants used stickers that identify their group affiliation to vote for the criteria they found most important.
\item Room discussion: all participants exchanged on their decisions by sharing the results of their table’s work to the rest of the room.
\end{enumerate}

Each participant received a unique and neutral identifier of the form Affiliation-Geography. For example, a (hypothetical) land use planner from the table that brought together participants from the Maskoutains region received the code [A-1-1], and the UPA representatives from the Richelieu region received the codes [C-3 -1] and [C-3-2]. In addition, for certain activities, the affiliation of the participants is color-coded. For instance, blue for the land use planners and purple for the ministry representatives.

The coding system manifests itself in multiple ways, depending on the activity:
\begin{itemize}
\item When participants engage in activities involving drawing, the materials on which they draw will bear the aforementioned coding (A-1 etc..).
\item When participants engaged in activities involving post-its, the color of their post-it represented their affiliation or bore the aforementioned coding, depending on the activity.
\item When participants engaged in activities involving a weighting of choices with stickers, the color of their post-it or sticker represented their affiliation or had the aforementioned coding, depending on the activity.
\end{itemize}

To ensure that this system is used consistently throughout the workshop, each participant was given a personal folder with their own color-coded stickers and post-its. Participants were instructed to only use the stickers and material that have been personally handed to them.\\

\subsubsection*{Weighting system}

 The participants were given a total of 6 stickers to cast their vote for the opportunities and obstacles that they thought were most important. They were given a total of 6 stickers: 2 blue (important), 2 yellow (very important) and 2 red (of the first importance).\\

\subsubsection{Data processing and analysis}

\subsubsection*{Voting: opportunity and challenges}

The data from the opportunities and challenges activity (with post-its), and the weighting of these elements were treated with the following steps:
\begin{enumerate}
  \item Each post-it (opportunity or challenge) received a unique ID.
  \item A first score is calculated by summing the points for each sticker (each vote)
  \begin{itemize}
      \item 1 point for a blue vote
      \item 2 points for a yellow vote
      \item 3 points for a red vote
  \end{itemize}
  \item This score was then weighted by multiplying it by the post-it’s diversity score which served as a measure of consensus
\begin{itemize}
\item This diversity score is the inverse simpson index (R package Vegan). The more diverse the group the higher this index is.
\end{itemize}
\end{enumerate}
These steps were carried out for each table and the results were treated separately for each table because not all tables had the same potential for diversity. For each table, the 10 post-its with the highest scores were retained, 5 among the post-its that had been placed on a specific area of the map (spatialized) and 5 for those that were not. The list of post-it contents, along with the scores, can be found in the appendix tables \ref{tab:opp_chall_ns} and \ref{tab:opp_chall_s}. In addition, a map synthetic map of the spatialized post-its was produced (see figures \ref{fig:reg_AC} and \ref{fig:trans_AC}).\\

\subsubsection*{Digitization of results}

The priority areas and links were digitized by hand in QGIS. The smooth tool was used to generalize the trace of each area and corridors (see figures \ref{fig:reg}  and \ref{fig:trans} for the final results). It is important to note that those maps are an approximate representation of reality, as the drawings done during the activity were an approximation of what the participant had in mind.

In order to integrate the results with the land use change model, we added a 5 km buffer around the linkages to simulate a corridor effect. This corridor width size is arbitrary and is in most cases  an unrealistic expectation forhow large a wildlife corridor can or should be. \\

\subsubsection{Conservation and land use change scenarios}

The conservation scenarios were derived with the goal to demonstrate how results of this type of community workshop can be integrated with other planning methods for connectivity conservation, such as land use change and connectivity modelling. They are not meant to be directly used as conservation guidelines. They are unrealistic by definition and are meant to reflect the most extreme bounds of connectivity change and/or protection measures imaginable for the region. More realistic scenarios featuring targeted action would need to be devised in future community workshops. Although there exists a tremendous amount of work on scenario-building for conservation, specific examples for connectivity conservation are rare. The scenarios presented here are a “proof of concept” for the integration of qualitative results into a quantitative model. 

We integrated the results of the workshop with the land use change and connectivity model devised in Chapter 1 by changing two attributes of the model: \textit{conservation status} and \textit{reforestation location}. These additional conservation scenarios are crossed with the climate change scenarios.

Conservation status determines whether or not forest can be cut down, and whether pixels can transition into urban and agricultural land. In Chapter 1, we gave this status to areas that were previously known as protected, but assumed no more forest would be added to this set of protected areas. In this chapter, a protection status is given to all pixels within the identified priority areas and within a 5km buffer around the identified priority linkage areas. All pixels under this status cannot transition into urban land and remain in their state. This represents an unrealistic proportion of the region under protection (about 33% of the region)

The reforestation status modified what zones can be targeted for reforestation. In reforestation scenarios in Chapter 1, reforestation happened randomly (barring some neighboring rules). In this chapter, reforestation is targeted to the same areas identified in the workshop (all pixels within the identified priority areas and within a 5km buffer around the identified priority linkage areas). As in the model described in Chapter 1, all new reforested area takes the value of medium aged (30-50 yr) deciduous forest and remain in this state for the rest of the simulation. As previously mentioned, this assumption is necessary because LANDIS surface types are not available for newly forested areas. For the same reason, no forest transitions are allowed for this forest type. These two assumptions are conservative:  we do not impose succession or disturbance on these newly forested ares and assume medium age as an average age class.

The three modified conservation and land use change scenarios are therefore:
\begin{enumerate}
  \item Business as usual land use change + corridor protection \textbf{(BAU-Corr)}
 \item Business as usual land use change + corridor protection + reforestation \textbf{(BAU-R-Corr)}
 \item Business as usual land use change + corridor protection + targeted reforestation \textbf{(BAU-R(T)-Corr)} \\
\end{enumerate}

\subsubsection{Connectivity analyses}

In order to be able to compare the results of these new scenarios with those developed in Chapter 1, we conducted the same connectivity analyses using the Circuitscape software. The software “borrows algorithms from electronic circuit theory to predict connectivity in heterogeneous landscapes''. It is a free and open software under MIT license developed originally in Python, and now in its 5th version, in Julia \citep{circuitjulia}.

For each of the maps produced in the suitability analysis, Circuitscape was run in two directions (“wall to wall” run, \cite{mcrae_conserving_2016}): east to west and north to south. This method allowed us to model omnidirectional animal movements, and required that we added the resulting North/South and East/West flow map. The results for each map were therefore added to produce a final set of flow maps for analysis. From these flow maps, we extracted the mean flow for each of the municipalities and for the entire flow map. \\

\section{Results}

%TBC 

\section{Discussion}

%TBC

\section{Conclusion}

% To rewrite

% At this step of completion of the project, there remains many steps with regard to data processing, analysis and presentation that still needs to be answered. Concerning data processing and model proofing, chapter 1 remains very weak in its capacity to evaluate how much the Random Forest - Cellular Automata model can be trusted to provide a good predictor of land use change. This is due in part to the model fitting process which so far relies on practices for which we should perhaps seek an alternative such as mean imputation and raster aggregation. Concerning data analysis and presentation, the main issues are in chapter 2, where the methods of analysis of the workshop data remain to be fully determined. The method for comparing  connectivity outputs with community-produced corridors also remains to be decided. That being said, we are on our way to provide an answer to the problems identified in the problem statement of each chapter.\\

%---------------------------------------------------------------------------------------------------------------------------------------------------

\newpage
\begin{center}
\section*{Figures \& Tables}
\end{center}

%---------------------------------------------------------------------------------------------------------------------------------------------------
% tables

\begin{table}[h!]
\centering
\caption{Breakdown of the area covered by each table in the workshop}
\label{tab:workshoptables}
\begin{tabular}{ll}
\hline
\textbf{Table} & \textbf{MRCs} \\ \hline
Ouest (West) & Vaudreuil, Haut SL, Beauharnois \\
Centre (Center) & Jardins, Haut Richelieu, Rouville, Roussillon \\
Nord (North) & \begin{tabular}[c]{@{}l@{}}Longueuil, Marguerite d'Youville, Vallée du richelieu, \\ Pierre de Saurel, Les Maskoutains\end{tabular} \\
Est & Brome-Missisquoi, Haute Yamaska, Acton \\
Transversal (x2) & Toute la Montérégie (All of  Montérégie)\\ \hline
\end{tabular}
\end{table}

\clearpage

%---------------------------------------------------------------------------------------------------------------------------------------------------
% figures

% AC

\begin{figure}[h!]
\makebox[\textwidth]{
\includegraphics[width=1.3\textwidth]{figures/Regional_AC.png}
}
\caption{Cartography of opportunity and challenges  with the highest importance and diversity metric (Regional tables).}
\label{fig:reg_AC}
\end{figure}
\clearpage

\begin{figure}[h!]
\makebox[\textwidth]{
\includegraphics[width=1.3\textwidth]{figures/Transversal_AC.png}
}
\caption{Cartography of opportunity and challenges  with the highest importance and diversity metric (Transversal tables).}
\label{fig:trans_AC}
\end{figure}
\clearpage

%---------------------------------------------------------------------------------------------------------------------------------------------------
% Links and areas

\begin{figure}[h!]
\makebox[\textwidth]{
\includegraphics[width=1.3\textwidth]{figures/Regional.png}
}
\caption{Cartography of priority areas and links (Regional tables).}
\label{fig:reg}
\end{figure}
\clearpage

\begin{figure}[h!]
\makebox[\textwidth]{
\includegraphics[width=1.3\textwidth]{figures/Transversal.png}
}
\caption{Cartography of priority areas and links (Transversal tables).}
\label{fig:trans}
\end{figure}
\clearpage

%---------------------------------------------------------------------------------------------------------------------------------------------------
% Flow

% Linear
\begin{figure}[h!]
\makebox[\textwidth]{
  \includegraphics[width=\textwidth]{figures/connectivity_decrease_x5species_chap2.png}
}
 \caption{Change in mean flow (in \% of the the 2010 flow) between 2010 and 2100, contrasting BAU scenario (solid line) with conservation scenarios.}
 \label{fig:flow_linear_2}
\end{figure}

% Radar
\begin{figure}[h!]
\makebox[\textwidth]{
  \includegraphics[width=\textwidth]{figures/radar_ggradar_chap2.png}
}
 \caption{Change in mean flow (in \% of the the 2010 flow) between 2010 and 2100, contrasting BAU scenario (solid line) with conservation scenarios.}
 \label{fig:flow_radar_2}
\end{figure}

% Radar All
\begin{figure}[h!]
\makebox[\textwidth]{
  \includegraphics[width=\textwidth]{figures/radar_ggradar_both.png}
}
 \caption{Change in mean flow (in \% of the the 2010 flow) between 2010 and 2100, contrasting BAU scenario (solid line) with both other land use change and conservation scenarios.}
 \label{fig:flow_radar_both}
\end{figure}

% Histograms
\begin{figure}[h!]
\makebox[\textwidth]{
  \includegraphics[width=\textwidth]{figures/hist_chap2.png}
}
 \caption{Histograms of flow values change.}
 \label{fig:hist_2}
\end{figure}

%---------------------------------------------------------------------------------------------------------------------------------------------------
% SURF

% Linear
\begin{figure}[h!]
\makebox[\textwidth]{
  \includegraphics[width=\textwidth]{figures/surf_chap2.png}
}
 \caption{Change in identified features (partly pinch-points.}
 \label{fig:surf_linear_2}
\end{figure}

% Radar
\begin{figure}[h!]
\makebox[\textwidth]{
  \includegraphics[width=\textwidth]{figures/surf_radar_chap2.png}
}
 \caption{Change in identified features (partly pinch-points.}
 \label{fig:surf_radar_2}
\end{figure}
