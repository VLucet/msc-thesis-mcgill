% !TEX root = ./thesis.tex
\chapter{Comparing model-driven connectivity conservation priorities with perceptions of priorities by local stakeholders}
\begin{center}
{Valentin Lucet$^{1}$, Andrew Gonzalez$^{1}$}\\
\end{center}
\textit{Author Affiliations:}\\
\normalsize{$^{1}$Department of Biology, McGill University}\\

\section{Abstract}

\section{Introduction}

\textit{Problem statement}: Current connectivity conservation science  fails to confront the results of the prioritisation with the priorities perceived by stakeholders, and therefore fails to integrate them.

\textit{Research questions}: \textbf{How do different stakeholders perceive connectivity conservation priorities, given the obstacles and opportunities for land use planning apparent in the region? How do community perception of connectivity priority areas align with modelling outputs?} \\

Connectivity conservation methods usually entails modelling connectivity of the landscape of interest and using a prioritisation method to determine conservation priorities. However, current connectivity conservation planning methods fail to confront the results of the prioritisation with the priorities perceived by stakeholders.

Failing to integrate the perceptions of stakeholders is detrimental to conservation for two reasons: first, it most likely mean that conservation will fail to gather enough local momentum to lead to actual policy change, and second, it means that the only tool for decision making will be the modelisations, whereas these are incomplete representation of the landscape and would benefit from inputs from stakeholders. This is especially true in landscapes where a considerable effort of connectivity modelling has already been conducted, like in the landscape of interest in this thesis, the southern Quebec region of Montérégie.

Connectivity conservation is often faced with the issue of understanding the processes driving land use change (Worboys et al. 2010). Because land use change is a social process with consequences of both social and ecological nature, it is best understood within the concept of social-ecological system. A social-ecological system (SES) can be understood as the set of human and non-human actors, the set of natural habitats they inhabit and resources and use, and the set of interactions that are maintained between all the components of the system (Ostrom 2009). SESs thus form complex and integrated aggregates of interactions (Hinkel et al. 2014). Those interactions also impact governance, the process by which actors in power establish rules and laws (Bissonnette et al. 2018).

Our understanding of SESs often lacks two important elements: social realism and spatial explicitness.   We need to build more realistic models for social-ecological systems by being more spatially explicit about the obstacles and opportunities presented by conservation as a land use type. Understanding how these obstacles and opportunities can influence management decisions is crucial for our understanding of connectivity conservation planning, where land-use conflicts can hinder the protection and restoration of connectivity. Although it is relatively easy to identify where land-use changes might conflict with connectivity conservation, evaluating to which extent these conflicts matter in a local conservation context is more difficult (Mitchell et al. 2015).

This study aims to start filling the gap identified by conducting participatory research in Montérégie, in south Quebec, where connectivity conservation has become an important stake. This project has been developed in collaboration with the non-profit NAQ (Nature action Quebec). NAQ invited the QCBS to contribute to their connectivity conservation project (called PADF, or Plan d’Aménagement Durable des Forêts) by co-developing the research described here.

The goal of this chapter is therefore twofold: gain a better understanding of conservation priorities in the region and then compare them to the results of chapter 1. We use a workshop and GIS as the main methodological tool.\\

\section{Methods}

\subsection{workshop}

\subsubsection{Target}

The workshop was organised on January 22nd 2020, and aimed to gather stakeholders from multiple groups (of all the groups, only the last group was not represented):
\begin{enumerate}
  \item Representatives of Non-Governmental Organizations that are involved in the conservation of ecological connectivity within the study extent (i.e. Montérégie).
  \item Land use planners (“amenagistes”) of the administrative regions covered by the study extent (the 15 MRCs in Montérégie).
  \item Representatives of Ministries involved in conservation (MFFP, MELCC).
  \item Representatives of the UPA (“Union des producteurs agricoles”) in Montérégie.
  \item Representatives of the private forestry industry unions (“producteurs forestiers”) \\
\end{enumerate}

\subsubsection{Consensual mapping}

We employed a method that could be coined “consensual participatory mapping in geographically structured focus groups”. This means that participants whose organisations operate in the same region (the same MRC) are seated at the same table. See table 4 for the breakdown of each MRC by table, of which there were 4. In addition, for participants whose zone of influence/action covered the whole region, two “transversal tables” were created. In the subsequent section, the results are divided between the regional and transversal tables.
The general method proceeds in 3 exercises, each step involving the same focus groups.

\begin{itemize}
  \item Exercise 1: mapping of forested cores of importance for ecological connectivity.
  \item Exercise 2: mapping of spatially explicit obstacles and opportunities for habitat connectivity in terms of social-economic activity and land use.
  \item Exercise 3: mapping of links of importance between forested cores of importance, that takes into account obstacles and opportunities on the landscape.
\end{itemize}

Each of these three exercises is conducted in 4 to 5 steps. Here we use exercise 1 to describe the method in more detail at each step.

\begin{enumerate}
\item Individual reflection - each participant thinks on their own about the problem at hand. For instance, participants take time to think about what forested cores are important to connect in the landscape
\item Group discussion, at each table (i.e. in each focus group) each participant contributes their answer to the question posed by the exercise. For instance, participants share what forested cores they found to be important.
\item Group discussion on the criteria that each participant used to answer the question. For instance, participants say why they think they choose these forested cores.
\item Consensus building, participants vote for the most important criteria. For instance, participants use stickers that identify their group affiliation to vote for the criteria they found most important.
\item Room discussion: all participants exchange on their decisions by sharing the results of their table’s work to the rest of the room.
\end{enumerate}

Each participant receives a unique and neutral identifier of the form Affiliation-Geography. For example, a (hypothetical) land use planner from the table that brings together participants from the Maskoutains region will receive the code [A-1-1], and the UPA representatives from the Richelieu region will receive the codes [C-3 -1] and [C-3-2]. In addition, for certain activities, the affiliation of the participants is color-coded. For instance, blue for the land use planners and purple for the ministry representatives.

The coding system manifests itself in multiple ways, depending on the activity:
\begin{itemize}
\item When participants engage in activities involving drawing, the materials on which they draw will bear the aforementioned coding (A-1 etc..).
\item When participants engage in activities involving post-its, the color of their post-it represents their affiliation or bear the aforementioned coding, depending on the activity.
\item When participants engage in activities involving a weighting of choices with stickers, the color of their post-it or sticker represents their affiliation or bear the aforementioned coding, depending on the activity.
\end{itemize}

To ensure that this system is used consistently throughout the workshop, each participant is given a personal folder with their own color-coded stickers and post-its. Participants are instructed to only use the stickers and material that have been personally handed to them.\\

\subsubsection{Weighting system}

 The participants were given a total of 6 stickers to cast their vote for the opportunities and obstacles that they thought were most important. They were given a total of 6 stickers: 2 blue (important), 2 yellow (very important) and 2 red (of the first importance).

\subsubsection{Data processing and analysis}

\subsubsection*{Voting: opportunity and challenges}

\subsubsection*{Priority areas and links}

\subsubsection{Comparison with models}

\section{Results}
\lipsum[66]\\
\section{Discussion}
\lipsum[66]\\

\newpage
\section*{Figures \& Tables}
