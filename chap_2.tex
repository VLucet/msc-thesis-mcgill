% !TEX root = ./thesis.tex
\chapter{Comparing model-driven connectivity conservation priorities with perceptions of priorities by local stakeholders}
\begin{center}
{Valentin Lucet$^{1}$, Andrew Gonzalez$^{1}$}\\
\end{center}
\textit{Author Affiliations:}\\
\normalsize{$^{1}$Department of Biology, McGill University}\\

\section{Abstract}

\section{Introduction}

\textit{Problem statement}: Current connectivity conservation science  fails to confront the results of the prioritisation with the priorities perceived by stakeholders, and therefore fails to integrate them.

\textit{Research questions}: \textbf{How do different stakeholders perceive connectivity conservation priorities, given the obstacles and opportunities for land use planning apparent in the region? How do community perception of connectivity priority areas align with modelling outputs?} \\

Connectivity conservation methods usually entails modelling connectivity of the landscape of interest and using a prioritisation method to determine conservation priorities. However, current connectivity conservation planning methods fail to confront the results of the prioritisation with the priorities perceived by stakeholders.

Failing to integrate the perceptions of stakeholders is detrimental to conservation for two reasons: first, it most likely mean that conservation will fail to gather enough local momentum to lead to actual policy change, and second, it means that the only tool for decision making will be the modelisations, whereas these are incomplete representation of the landscape and would benefit from inputs from stakeholders. This is especially true in landscapes where a considerable effort of connectivity modelling has already been conducted, like in the landscape of interest in this thesis, the southern Quebec region of Montérégie.

Connectivity conservation is often faced with the issue of understanding the processes driving land use change (Worboys et al. 2010). Because land use change is a social process with consequences of both social and ecological nature, it is best understood within the concept of social-ecological system. A social-ecological system (SES) can be understood as the set of human and non-human actors, the set of natural habitats they inhabit and resources and use, and the set of interactions that are maintained between all the components of the system (Ostrom 2009). SESs thus form complex and integrated aggregates of interactions (Hinkel et al. 2014). Those interactions also impact governance, the process by which actors in power establish rules and laws (Bissonnette et al. 2018).

Our understanding of SESs often lacks two important elements: social realism and spatial explicitness.   We need to build more realistic models for social-ecological systems by being more spatially explicit about the obstacles and opportunities presented by conservation as a land use type. Understanding how these obstacles and opportunities can influence management decisions is crucial for our understanding of connectivity conservation planning, where land-use conflicts can hinder the protection and restoration of connectivity. Although it is relatively easy to identify where land-use changes might conflict with connectivity conservation, evaluating to which extent these conflicts matter in a local conservation context is more difficult (Mitchell et al. 2015).

This study aims to start filling the gap identified by conducting participatory research in Montérégie, in south Quebec, where connectivity conservation has become an important stake. This project has been developed in collaboration with the non-profit NAQ (Nature action Quebec). NAQ invited the QCBS to contribute to their connectivity conservation project (called PADF, or Plan d’Aménagement Durable des Forêts) by co-developing the research described here.

The goal of this chapter is therefore twofold: gain a better understanding of conservation priorities in the region and then compare them to the results of chapter 1. We use a workshop and GIS as the main methodological tool.\\

\section{Methods}

\subsection*{Study Site}
\lipsum[66]\\
\subsection*{Sampling}
\lipsum[66]\\
\subsection*{Analyses}
\lipsum[66]\\
\section{Results}
\lipsum[66]\\
\section{Discussion}
\lipsum[66]\\

\newpage
\section*{Figures \& Tables}
