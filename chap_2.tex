% !TEX root = ./thesis.tex
\chapter{Integrating stakeholders perceptions of connectivity conservation priorities into a spatially explicit land use and connectivity change model}
\begin{center}
{Valentin Lucet$^{1}$, Andrew Gonzalez$^{1}$}\\
\end{center}
\textit{Author Affiliations:}\\
\normalsize{$^{1}$Department of Biology, McGill University}\\

\section{Abstract}

\section{Introduction}
%\textit{Problem statement}: Connectivity conservation planning methods do not account for stakeholder perceptions. This can potentially lead to ill-informed conservation plans with low chances of success.
%
%\textit{Research questions}: \textbf{How do different stakeholders perceive connectivity conservation priorities, given the obstacles and opportunities for land use planning apparent in the region, and which conservation scenarios can be derived from these perceptions?} \\

%-----------------------------------------------
Connectivity conservation methods usually entails modelling connectivity of the landscape of interest and using a prioritization method to determine conservation priorities. However, current connectivity conservation planning methods fail to confront the results of the prioritization with the priorities perceived by stakeholders.

Failing to integrate the perceptions of stakeholders is detrimental to conservation for two reasons: first, it most likely mean that conservation will fail to gather enough local momentum to lead to actual policy change, and second, it means that the only tool for decision making will be the modelisations, whereas these are incomplete representation of the landscape and would benefit from inputs from stakeholders. This is especially true in landscapes where a considerable effort of connectivity modelling has already been conducted, like in the landscape of interest in this thesis, the southern Quebec region of Montérégie.

Connectivity conservation is often faced with the issue of understanding the processes driving land use change (Worboys et al. 2010). Because land use change is a social process with consequences of both social and ecological nature, it is best understood within the concept of social-ecological system. A social-ecological system (SES) can be understood as the set of human and non-human actors, the set of natural habitats they inhabit and resources and use, and the set of interactions that are maintained between all the components of the system (Ostrom 2009). SESs thus form complex and integrated aggregates of interactions (Hinkel et al. 2014). Those interactions also impact governance, the process by which actors in power establish rules and laws (Bissonnette et al. 2018).

Our understanding of SESs often lacks two important elements: social realism and spatial explicitness.   We need to build more realistic models for social-ecological systems by being more spatially explicit about the obstacles and opportunities presented by conservation as a land use type. Understanding how these obstacles and opportunities can influence management decisions is crucial for our understanding of connectivity conservation planning, where land-use conflicts can hinder the protection and restoration of connectivity. Although it is relatively easy to identify where land-use changes might conflict with connectivity conservation, evaluating to which extent these conflicts matter in a local conservation context is more difficult (Mitchell et al. 2015).

This study aims to start filling the gap identified by conducting participatory research in Montérégie, in south Quebec, where connectivity conservation has become an important stake. This project has been developed in collaboration with the non-profit NAQ (Nature action Quebec). NAQ invited the QCBS to contribute to their connectivity conservation project (called PADF, or Plan d’Aménagement Durable des Forêts) by co-developing the research described here.

The goal of this chapter is therefore twofold: gain a better understanding of conservation priorities in the region and then compare them to the results of chapter 1. We use a workshop and GIS as the main methodological tool.\\
%-----------------------------------------------

\section{Methods}

In order to understand what stakeholders perceive as a conservation priority, and derive conservation scenarios from those perceptions, we conducted a community workshop. The workshop was approved by McGill University’s Research Ethics Board (see appendix {x} for the approval letter). The workshop was held on January 22nd 2020 in the public library of Saint-Jean sur Richelieu. \\

\subsection{Community workshop}

The workshop aimed to gather stakeholders from multiple groups:
\begin{enumerate}
  \item Representatives of Non-Governmental Organizations that are involved in the conservation of ecological connectivity within the study extent (i.e. Montérégie).
  \item Land use planners (“amenagistes”) of the administrative regions covered by the study extent (the 15 MRCs in Montérégie).
  \item Representatives of Ministries involved in conservation (MFFP, MELCC).
  \item Representatives of the UPA (“Union des producteurs agricoles”) in Montérégie.
  \item Representatives of the private forestry industry unions (“producteurs forestiers”)
\end{enumerate}
Of all the groups, only the last group was not represented. \\

\subsubsection{Consensual mapping}

We employed a method coined “consensual participatory mapping in geographically structured focus groups”. Participants whose organisations operate in the same region (the same MRC) were seated at the same table. See table \ref{tab:workshoptables} for the breakdown of each MRC by table. In addition, for participants whose zone of influence or zone of action covered the whole region, two “transversal tables” were created. In the subsequent sections, the workshop results are divided between the regional and transversal tables.
The general method proceeded in 3 exercises, each step involving the same focus groups.

\begin{itemize}
  \item Exercise 1: mapping of forested cores of importance for ecological connectivity.
  \item Exercise 2: mapping of obstacles (for example: specific land use types, development projects in progress, local legislation) and opportunities for habitat connectivity in terms of social-economic activity and land use.
  \item Exercise 3: mapping of links (already recognized or potential) of importance between forested cores of importance, taking into account obstacles and opportunities on the landscape.
\end{itemize}

Each of these three exercises was conducted in 4 to 5 steps. Here we use exercise 1 to describe the method in more detail at each step.

\begin{enumerate}
\item Individual reflection - each participant thought on their own about the problem at hand. For instance, participants took time to think about what forested cores are important to connect in the landscape
\item Group discussion, at each table (i.e. in each focus group) each participant contributed their answer to the question posed by the exercise. For instance, participants shared what forested cores they found to be important.
\item Group discussion on the criteria that each participant used to answer the question. For instance, participants said why they thought they chose these forested cores.
\item Consensus building, participants voted for the most important criteria. For instance, participants used stickers that identify their group affiliation to vote for the criteria they found most important.
\item Room discussion: all participants exchanged on their decisions by sharing the results of their table’s work to the rest of the room.
\end{enumerate}

Each participant received a unique and neutral identifier of the form Affiliation-Geography. For example, a (hypothetical) land use planner from the table that brought together participants from the Maskoutains region received the code [A-1-1], and the UPA representatives from the Richelieu region received the codes [C-3 -1] and [C-3-2]. In addition, for certain activities, the affiliation of the participants is color-coded. For instance, blue for the land use planners and purple for the ministry representatives.

The coding system manifests itself in multiple ways, depending on the activity:
\begin{itemize}
\item When participants engage in activities involving drawing, the materials on which they draw will bear the aforementioned coding (A-1 etc..).
\item When participants engaged in activities involving post-its, the color of their post-it represented their affiliation or bore the aforementioned coding, depending on the activity.
\item When participants engaged in activities involving a weighting of choices with stickers, the color of their post-it or sticker represented their affiliation or had the aforementioned coding, depending on the activity.
\end{itemize}

To ensure that this system is used consistently throughout the workshop, each participant was given a personal folder with their own color-coded stickers and post-its. Participants were instructed to only use the stickers and material that have been personally handed to them. All participants were given a visual aid to remind them of the agenda of the workshop. Facilitators were also present at almost every table and helped facilitate the discussion. They were given a document to help them in this role. All workshop materials, including the ethics approval document for this research project can be found in Appendix 2.\\

\subsubsection*{Weighting system}

 The participants were given a total of 6 stickers to cast their vote for the opportunities and obstacles that they thought were most important. They were given a total of 6 stickers: 2 blue (important), 2 yellow (very important) and 2 red (of the first importance).\\

\subsubsection{Data processing and analysis}

\subsubsection*{Voting: opportunity and challenges}

The data from the opportunities and challenges activity (with post-its), and the weighting of these elements were treated with the following steps:
\begin{enumerate}
  \item Each post-it (opportunity or challenge) received a unique ID.
  \item A first score is calculated by summing the points for each sticker (each vote)
  \begin{itemize}
      \item 1 point for a blue vote
      \item 2 points for a yellow vote
      \item 3 points for a red vote
  \end{itemize}
  \item This score was then weighted by multiplying it by the post-it’s diversity score which served as a measure of consensus
\begin{itemize}
\item This diversity score is the inverse simpson index (R package Vegan). The more diverse the group the higher this index is.
\end{itemize}
\end{enumerate}
These steps were carried out for each table and the results were treated separately for each table because not all tables had the same potential for diversity. For each table, the 10 post-its with the highest scores were retained, 5 among the post-its that had been placed on a specific area of the map (spatialized) and 5 for those that were not. The list of post-it contents, along with the scores, can be found in the appendix tables \ref{tab:opp_chall_ns} and \ref{tab:opp_chall_s}. In addition, a map synthetic map of the spatialized post-its was produced (see figures \ref{fig:reg_AC} and \ref{fig:trans_AC}).\\

\subsubsection*{Digitization of results}

The priority areas and links were digitized by hand in QGIS. The smooth tool was used to generalize the trace of each area and corridors (see figures \ref{fig:reg}  and \ref{fig:trans} for the final results). It is important to note that those maps are an approximate representation of reality, as the drawings done during the activity were an approximation of what the participant had in mind.

In order to integrate the results with the land use change model, we added a 5 km buffer around the linkages to simulate a corridor effect. This corridor width size is arbitrary and is in most cases  an unrealistic expectation forhow large a wildlife corridor can or should be. \\

\subsubsection{Conservation and land use change scenarios}

The conservation scenarios were derived with the goal to demonstrate how results of this type of community workshop can be integrated with other planning methods for connectivity conservation, such as land use change and connectivity modelling. They are not meant to be directly used as conservation guidelines. They are unrealistic by definition and are meant to reflect the most extreme bounds of connectivity change and/or protection measures imaginable for the region. More realistic scenarios featuring targeted action would need to be devised in future community workshops. Although there exists a tremendous amount of work on scenario-building for conservation, specific examples for connectivity conservation are rare. The scenarios presented here are a “proof of concept” for the integration of qualitative results into a quantitative model. 

We integrated the results of the workshop with the land use change and connectivity model devised in Chapter 1 by changing two attributes of the model: \textit{conservation status} and \textit{reforestation location}. These additional conservation scenarios are crossed with the climate change scenarios.

Conservation status determines whether or not forest can be cut down, and whether pixels can transition into urban and agricultural land. In Chapter 1, we gave this status to areas that were previously known as protected, but assumed no more forest would be added to this set of protected areas. In this chapter, a protection status is given to all pixels within the identified priority areas and within a 5km buffer around the identified priority linkage areas. All pixels under this status cannot transition into urban land and remain in their state. This represents an unrealistic proportion of the region under protection (about 33% of the region)

The reforestation status modified what zones can be targeted for reforestation. In reforestation scenarios in Chapter 1, reforestation happened randomly (barring some neighboring rules). In this chapter, reforestation is targeted to the same areas identified in the workshop (all pixels within the identified priority areas and within a 5km buffer around the identified priority linkage areas). As in the model described in Chapter 1, all new reforested area takes the value of medium aged (30-50 yr) deciduous forest and remain in this state for the rest of the simulation. As previously mentioned, this assumption is necessary because LANDIS surface types are not available for newly forested areas. For the same reason, no forest transitions are allowed for this forest type. These two assumptions are conservative:  we do not impose succession or disturbance on these newly forested ares and assume medium age as an average age class.

The three modified conservation and land use change scenarios are therefore:
\begin{enumerate}
  \item Business as usual land use change + corridor protection \textbf{(BAU-Corr)}
 \item Business as usual land use change + corridor protection + reforestation \textbf{(BAU-R-Corr)}
 \item Business as usual land use change + corridor protection + targeted reforestation \textbf{(BAU-R(T)-Corr)} \\
\end{enumerate}

\subsubsection{Connectivity analyses}

In order to be able to compare the results of these new scenarios with those developed in Chapter 1, we conducted the same connectivity analyses using the Circuitscape software. The software “borrows algorithms from electronic circuit theory to predict connectivity in heterogeneous landscapes''. It is a free and open software under MIT license developed originally in Python, and now in its 5th version, in Julia \citep{circuitjulia}.

For each of the maps produced in the suitability analysis, Circuitscape was run in two directions (“wall to wall” run, \cite{mcrae_conserving_2016}): east to west and north to south. This method allowed us to model omnidirectional animal movements, and required that we added the resulting North/South and East/West flow map. The results for each map were therefore added to produce a final set of flow maps for analysis. From these flow maps, we extracted the mean flow for each of the municipalities and for the entire flow map. 

In addition, similarly to in Chapter 1 we extracted the distribution (logged) of pixel flow values under the form of histograms. We also ran the SURF analysis in the same way and counted up the number detected feature for each surface, in order to derive a measure of flow complexity.\\

\section{Results}

\subsection{Community workshop}

\subsubsection{Forested cores of importance}  

All tables were able to quickly build good consensus as to which forested cores was to be linked in priority. The participants at regional (figure \ref{fig:reg_AC}) and transversal (figure \ref{fig:trans_AC}) tables, identified redundant groups of forest patches. Because transversal tables covered the whole region, they often identified cores that engulfed the cores identified by the regional table, giving us an idea of what could constitute regional versus sub-regional conservation priorities.

Among the cores identified, a lot of the currently protected areas in the region were circled, for example the protected areas of the Green Mountains in the MRC of Brome-Missisquoi, or parts of the Monteregian Hills such as the Mount Saint-Hilaire. %TODO maybe calculating the exact % of overlap would be worth it here?

\subsubsection{Obstacles and opportunities}

The tables identified a total of 63 items as opportunities or assets, and 56 as constraints or obstacles. Constraints scored higher, on average, than assets, indicating a focus of participants on constraints. Tables \ref{tab:opp_chall_s} and \ref{tab:opp_chall_ns} displays the 5 assets or obstacles that scored the highest for each table. We divided the results into elements that had been spatialized (i.e. placed on a specific location of the map during the exercise), and non spatialized (i.e. not placed on a specific location ). Mapped elements should reflect a localized realities for connectivity conservation, whereas non mapped elements should speak to more general and large scale aspects. The elements in the spatialized set scored higher on average than in the non spatialized set, indicating that participants tended to focus on more tangible and local issues ans assets.

Among non-spatialized elements, the assets that scored the highest varied according to the sub-region under consideration. In the Centre table participants emphasized the strong presence of environmental NGOs, whereas the North and East tables emphasized elements linked to forest management. Interestingly, the presence of a strong agricultural matrix in the region scored high as an asset at the Transversal tables, and very high as a constraint on a regional table, highlighting the polarizing status of the agricultural land use type in the region. Another constraint linked to agriculture is the CPTAQ - an agency responsible for the protection of agricultural lands often involved in discussions around land use planning - mentioned for the Centre sub-region. Similarly, the East table mentions the issues that farmers have to make their farm profitable as another constraint. Besides agriculture, participants emphasized other constraints, such as economic realities making conservation a less likely investment, given the cost of implementing connectivity conservation measures - also mentioned as a perceived constraint. Finally, the fact that the overwhelming majority of forests are situated on private lands, along with the existing road infrastructure, deserved to be mentioned.

The spatialized elements are summarized in maps \ref{fig:trans_AC} and \ref{fig:reg_AC}, which shows the single highest scoring element (asset or constraints) for each sub region. In terms of assets, the Centre, East and North tables mentioned positive local inclinations toward conservation. Similarly, the West and Transversal tables mentioned local legislation and the active mobilization of stakeholders, highlighting that connectivity conservation is a concern for many stakeholders in the region. Concerning constraints, participants emphasized once again elements related to the predominance of agriculture in the region. The lack of engagement from farmers in mentioned in the Centre and West tables, along with land use "pressures" coming both from agriculture and urban spread. The East table emphasized on the similar "villegiative" pressures, along with more local concern such as the Highway 10 and the intensity of local agriculture practices. The North table also mentions its highways. The transversal tables mentions the price of agricultural lands and the need for compensating the loss of agricultural production due to conservation. \\

\subsubsection{Links of importance at the regional scale}

\subsubsection*{Factors of Priority}

Among the factors that were designated as a priority for the identification of links, elements that came up the most were tied to the characteristics of the natural environments that make up the nodes that are to be connected. The "quality" of these environments and of the nodes themselves seemed important to workshop participants: for instance, the protection status of habitats, their diversity (variety of habitats), their surface area, isolation or proximity, and the pre-existence of connectivity potential. This highlights that stakeholders agree on the value of high quality habitats and on the need to protect them.

In addition, there was much discussion on the dual role of both "facilitating" and "constraining" factors in deciding whether a link is a priority, and as to which type should be given more weigth. Facilitating factors include the feasibility of establishing the link, social facilitators such as political and social buy-in, and territorial facilitators, such as appropriate of land use. There were several types of constraints: financial, social, territorial and physical, and must be considered in relation to the socio-political and spatial planning context in which they emerge. % example?

The presence of species at risk was identified as an important factor, along with other elements such as social factors (available economic means and the level of local interest in conservation) and characteristics of the link itself, including the functionality of the corridor. Consideration of threats such as the level of fragmentation was also noted. Finally, the fact that the environment is to be protected or restored was also deemed important. Interestingly, regulations are among the factors that were cited less often, such as Quebec's PRMHH ("Plans régionaux des milieux humides et hydriques", or regional wetlands conservation plans), and migration routes. \\ 

\subsubsection*{Priority link mapping}

As for the forested cores, there was overall good consensus on what links should be prioritized for all groups. The consensual mapping process identified important links between the key areas identified in the previous steps at both regional (figure \ref{fig:reg}) and transversal tables (figure \ref{fig:trans}). At regional tables, a first comment can be made about the choices in corridor locations. Montérégie is a heavily fragmented region, with lots of small patches. This leads to what could be called "corridor opportunism". Although it was made clear that the goal was not to design a precise path for each corridor, we see that participants wanted to link areas by linking patches they judged to be organized in a linear or corridor-like fashion. This demonstrates that participants saw small patches as important for corridor design and perceived corridors as an important tool to conserve and restore the potential of fragmented patches. However, it also means that designing corridors in such a way  might mean that an otherwise important patch could be ignored because its location does not fit a linear corridor design.

A second comment can be made about corridor redundancy. Participants were ask to limit themselves in the amount of links they were to prioritize, but were not given an exact number. Only one table (North) felt that corridor redundancy was important enough to include redundant design in their priorities. It can be clearly seen in figure \ref{fig:reg} most participants decided to sacrifice redundancy for parsimony, as only the northern part of the region displays redundant (yet arguably still parsimonious), corridor design.

As for the forested cores, there was a lot of redundancy between the regional and the transversal tables concerning the links identified, there was also some differences and complementarity (see figure \ref{fig:trans}). Because the transversal tables covered the entire region, they were able to identify links at a larger scale. This allowed the participants to designated\ a larger scale link along the Saint Lawrence River, as well as transversal links between the lake Champlain region and the easternmost parts of the region, and between the Saint-Francois wildlife area and mount Rigaud, mentioning that this last link would need to be implemented in collaboration with Ontario. It was clear that participants at transversal tables felt that a broader discussion of links with outside of the region was necessary. \\

\subsection{Conservation scenarios}

\subsubsection{Land Use Change Model}

The projected land use maps for each of the conservation scenarios allows us to visualize some of the possible futures for the region, if connectivity conservation, as it is conceived by stakeholders, were to become a priority of regional land use planning. Figures \ref{fig:Corr_compare}, \ref{fig: CorrRef_compare} and \ref{fig: CorrRefT_compare} compares land use maps from 2010 with the projected maps for 2100 for each of the conservation scenarios. 

The information contained in those maps is summarized under the form of a bar chart showing chnages in Agricultural, Urban and Forested land use area for 2010 and 2100 (see figure \ref{fig:bar_chap2}). For all scenarios, the urban area is allowed to almost double. This is well seen in all maps, where urban land has spread in a similar fashion than that of the BAU scenario from Chapter 1. Forest is maintained for all scenarios except for the BAU-Corr scenario, which only differs from the BAU scenario of chapter 1 in that links and areas identified by stakeholders are protected from urbanization and agricultural expansion. In this scenario, agriculture is also allowed to maintain its area. This is not the case for the other scenarios, which include reforestation, which, in the face of spread urban land, must be compensated by a loss in agricultural land (similarly to what was observed under the reforestation scenario in Chapter 1).

Although the two conservation scenarios that include reforestation do not differ in their changes to the different land use area, they display different spatial patterns. Under the BAU-R-Corr scenario we observe a similar pattern of growing forest patches than under Chapter 1's Reforestation scenario. But we see the most striking difference with the last conservation scenario, BAU-R(T)-Corr, for which reforestation is targeted to the corridors identified by the stakeholders. Under this scenario, we still assume that forest grow from existing patches. The consequence of those two constraints is that where forest can grow easily in a linear fashion, it is encourages to do so. Therefore we see that actual corridors of forest are delineated, within the 5km buffer around identified links that was given to the model. The patterns is especially striking in the north of the region. In the south and the west of the region, we see a different pattern under which the model encouraged the growth of the identified forested cores. \\

\subsubsection{Connectivity Modelling}

\vspace{1em}

\subparagraph*{\textit{Current flow}} 

\vspace{1em}

\subparagraph*{\textit{Flow distributions}}

\vspace{1em}

\subparagraph*{\textit{Feature detection}} 

\section{Discussion}

%TBC

\section{Conclusion}

% TBC

%---------------------------------------------------------------------------------------------------------------------------------------------------

\newpage
\begin{center}
\section*{Figures \& Tables}
\end{center}

%---------------------------------------------------------------------------------------------------------------------------------------------------
% tables (moved to appendix)

%---------------------------------------------------------------------------------------------------------------------------------------------------
% figures

% AC

\begin{figure}[h!]
\makebox[\textwidth]{
\includegraphics[width=1.3\textwidth]{figures/Regional_AC.png}
}
\caption{Cartography of opportunity and challenges  with the highest importance and diversity metric (Regional tables).}
\label{fig:reg_AC}
\end{figure}
\clearpage

\begin{figure}[h!]
\makebox[\textwidth]{
\includegraphics[width=1.3\textwidth]{figures/Transversal_AC.png}
}
\caption{Cartography of opportunity and challenges  with the highest importance and diversity metric (Transversal tables).}
\label{fig:trans_AC}
\end{figure}
\clearpage

%---------------------------------------------------------------------------------------------------------------------------------------------------
% Links and areas

\begin{figure}[h!]
\makebox[\textwidth]{
\includegraphics[width=1.3\textwidth]{figures/Regional.png}
}
\caption{Cartography of priority areas and links (Regional tables).}
\label{fig:reg}
\end{figure}
\clearpage

\begin{figure}[h!]
\makebox[\textwidth]{
\includegraphics[width=1.3\textwidth]{figures/Transversal.png}
}
\caption{Cartography of priority areas and links (Transversal tables).}
\label{fig:trans}
\end{figure}
\clearpage

%---------------------------------------------------------------------------------------------------------------------------------------------------
% Land use composition

\begin{figure}[h!]
\makebox[\textwidth]{
\includegraphics[width=\textwidth]{figures/bar_chap2.png}
}
\caption{Land use composition.}
\label{fig:bar_chap2}
\end{figure}
\clearpage

%---------------------------------------------------------------------------------------------------------------------------------------------------
% Flow

% Linear
\begin{figure}[h!]
\makebox[\textwidth]{
  \includegraphics[width=\textwidth]{figures/connectivity_decrease_x5species_chap2.png}
}
 \caption{Change in mean flow (in \% of the the 2010 flow) between 2010 and 2100, contrasting BAU scenario (solid line) with conservation scenarios.}
 \label{fig:flow_linear_2}
\end{figure}

% Radar
\begin{figure}[h!]
\makebox[\textwidth]{
  \includegraphics[width=\textwidth]{figures/radar_ggradar_chap2.png}
}
 \caption{Change in mean flow (in \% of the the 2010 flow) between 2010 and 2100, contrasting BAU scenario (solid line) with conservation scenarios.}
 \label{fig:flow_radar_2}
\end{figure}

% Radar All
\begin{figure}[h!]
\makebox[\textwidth]{
  \includegraphics[width=\textwidth]{figures/radar_ggradar_both.png}
}
 \caption{Change in mean flow (in \% of the the 2010 flow) between 2010 and 2100, contrasting BAU scenario (solid line) with both other land use change and conservation scenarios.}
 \label{fig:flow_radar_both}
\end{figure}

% Histograms
\begin{figure}[h!]
\makebox[\textwidth]{
  \includegraphics[width=\textwidth]{figures/hist_chap2.png}
}
 \caption{Histograms of flow values change.}
 \label{fig:hist_2}
\end{figure}

%---------------------------------------------------------------------------------------------------------------------------------------------------
% SURF

% Linear
\begin{figure}[h!]
\makebox[\textwidth]{
  \includegraphics[width=\textwidth]{figures/surf_chap2.png}
}
 \caption{Change in identified features (partly pinch-points.}
 \label{fig:surf_linear_2}
\end{figure}

% Radar
\begin{figure}[h!]
\makebox[\textwidth]{
  \includegraphics[width=\textwidth]{figures/surf_radar_chap2.png}
}
 \caption{Change in identified features (partly pinch-points.}
 \label{fig:surf_radar_2}
\end{figure}
