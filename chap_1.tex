% !TEX root = ./thesis.tex
\chapter{Integrating Land Use Change Modelling with Connectivity Modelling}
\begin{center}
{Valentin Lucet$^{1}$, Andrew Gonzalez$^{1}$}\\
\end{center}
\textit{Author Affiliations:}\\
\normalsize{$^{1}$Department of Biology, McGill University}\\
\section{Abstract}

Space is a finite resource. How we manage and govern space is a reflection of the trade-offs and choices made by people and organizations at different spatial and temporal scales. Those choices primarily determine and regulate land use: if and how space is organized and whether resources are exploited, transformed or conserved. Ecological connectivity, defined as the extent to which the landscape supports the movements of organisms, can be strongly affected by land use. It is an important component of the resilience of populations in heterogeneous and fragmented landscapes. Land use changes such as urban sprawl and agricultural intensification intensify habitat fragmentation and landscape homogenization, leading to the erosion of ecological connectivity. The Monteregie region in southern Quebec, where this work takes place, is experiencing urban growth and sprawl. We present a framework that integrates land-use change and connectivity modeling to forecast future changes in connectivity, using a combination of statistical modeling, MCMC-based simulations, and circuit theory. Models trained on past land use data were used to project future land use changes using different scenarios, and estimate future changes in functional connectivity for 5 different umbrella species. We contrast the past and future impacts of trends in land use (i.e.urbanization, agricultural expansion, and deforestation) and derive conservation priorities for the design of a local network of connected protected areas resilient to future landscape change. We demonstrate the flexibility of a scenario approach in forecasting the range of possible futures for ecological connectivity in the region and show that taking probable landscape changes into account lead to different conservation priorities. We conclude on the importance of considering such changes to produce a resilient network of protected areas and highlight the need for a multi stakeholder approach in the definition of scenarios and conservation priorities.\\

\section{Introduction}
\textit{Problem Statement}: Connectivity conservation planning methods does not account for risks associated with future land use change. This can potentially lead to ill-informed conservation plans with low chances of success.\\

\textit{Research question}: \textbf{How can we explain past changes in land use and connectivity in Montérégie and better predict future changes?}\\

In this first chapter, we build on past work of connectivity modelling using circuit theory in the region and complement it with land use change modelling that uses a combination of statistical modeling and MCMC-based simulations. Models trained on past land use data were used to project future land use changes and estimate future changes in potential functional connectivity for 5 different umbrella species. We derive conservation priorities for the design of a local network of connected protected areas resilient to future landscape change.

This work is the continuation of two important contributions on the connectivity of the region: Albert et al. (2017) and Rayfield et al. (2019, unpublished). In a seminal paper, Albert laid out the first methodological steps we follow: umbrella species selection and connectivity modelling. They also included a simple land use change model that was parameterized to replicate plausible change in the region. Rayfield improved on Albert’s work by increasing the scale of the analysis and reducing the number of focal species. They showed that because species had redundant connectivity needs, modelling the needs for 5 species resulted in qualitatively similar results than when modeling the needs of all 14 species, like in Albert et al. (2017). They demonstrated that we could exploit this redundancy to reduce the computing time needed for analysis. This is welcomed as land use change simulation is computationally intensive.

In this chapter, we utilise Albert and Rayfield’s framework, modelling connectivity for the 5 focal species identified by Rayfield, and using their workflow for habitat suitability and connectivity analysis. We complement this framework with a land use model that combines statistical modeling and MCMC-based simulations. It is important to note that the primary goal of this chapter is not to draw strong inference with regards to land use change drivers in the region, but to provide enough predictive power to replicate the trends in land use change that have been observed and project those trends forward into the future.

We predict that our land use change model will simulate an overall decrease in connectivity change for our focal species, given the fact that we will be simulating a “business as usual” scenario for the change in the region. We do not generate specific predictions for each of our focal species.\\

\section{Methods}
In this section, we explain in detail the workflow for chapter 1. The workflow is divided into three steps: land use change modelling, connectivity modelling and prioritisation. The data and code necessary to reproduce this analysis is available on a github repository, under https://github.com/VLucet/TBD.\\

\subsection{Land Usec Change Modelling}
To be completed.
\\

\subsubsection{Background}
To be completed.
\\

\section{Results}
To be completed.
(Fig. \ref{fig:map}).
\\
\section{Discussion}
\lipsum[66]
\\

\newpage
\section*{Figures \& Tables}

\begin{figure}[!ht]
  \centering
    \includegraphics[width=0.5\textwidth]{../land_con_monteregie/outputs/figures/one_to_one_targets.png}
  \caption{Figure 1}
  \label{fig:map}
\end{figure}

\begin{table}[h!]
\centering
\begin{tabular}{llllll|l}
  \hline
  \hline
  Site & Weeks & S & XS & Daily & A/B & Total\\
  \hline
  Nhlanguleni (NHL) 	& 3 & 0 & 0 & 0 & Yes & 6 \\
  Nwaswitshaka (NWA) 	& 3 & 1 & 1 & 4 & Yes & 18 \\
  De LaPorte (DLP) 		& 1 & 1 & 1 & 0 & Yes & 6 \\
  Kwaggas Pan (KWA) 	& 2 & 1 & 1 & 0 & Yes & 8\\
  Girivana (GIR) 		& 3 & 0 & 0 & 0 & Yes & 6 \\
  Witpens (WIT) 		& 3 & 0 & 0 & 0 & Yes & 6 \\
  Imbali (IMB) 			& 3 & 0 & 0 & 0 & Yes & 6 \\
  Hoyo Hoyo (HOY) 		& 3 & 1 & 1 & 0 & Yes & 10 \\
  Nyamarhi (NYA) 		& 3 & 1 & 1 & 0 & Yes & 10 \\
  Ngosto North (NGO) 	& 3 & 1 & 1 & 0 & Yes &10 \\
  BLANK 				& 2 & 0 & 0 & 0 & No & 2 \\
  \hline
   						& 29 & 6 & 6 & 4 & & 88 \\
  \hline
  \hline
\end{tabular}
\caption{Table 1.}
\label{tab:samples}
\end{table}
