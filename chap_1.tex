% !TEX root = ./thesis.tex
\chapter{Integrating Land Use Change Modelling with Connectivity Modelling}
\begin{center}
{Valentin Lucet$^{1}$, Andrew Gonzalez$^{1}$}\\
\end{center}
\textit{Author Affiliations:}\\
\normalsize{$^{1}$Department of Biology, McGill University}\\
\section{Abstract}

Ecological connectivity, defined as the extent to which the landscape supports the movements of organisms, can be strongly affected by land use. It is an important component of the resilience of populations in heterogeneous and fragmented landscapes. Land use changes such as urban sprawl and agricultural intensification intensify habitat fragmentation and landscape homogenization, leading to the erosion of ecological connectivity. The Montérégie region in southern Quebec, where this work takes place, is experiencing urban growth and sprawl. We present a framework that integrates land-use change and connectivity modelling to forecast future changes in connectivity, using a combination of statistical modelling, MCMC-based simulations, and circuit theory. We used a hybrid modelling approach to project future land use changes using different scenarios, and estimate future changes in functional connectivity for 5 different umbrella species. We contrast the past and future impacts of trends in land use (i.e.urbanization, agricultural expansion, and deforestation) and derive some insights for the hypothetical design of a local network of connected protected areas resilient to future landscape change. We explore the flexibility of a scenario approach in forecasting the range of possible futures for ecological connectivity in the region. In conclusion, we highlight the need for a multi stakeholder approach in the definition of scenarios and conservation priorities.\\

\section{Introduction}
\textit{Problem Statement}: Connectivity conservation planning methods does not account for risks associated with future land use change. This can potentially lead to ill-informed conservation plans with low chances of success.

\textit{Research question}: \textbf{How can we explain past changes in land use and connectivity in Montérégie and better predict future changes?}\\

In this first chapter, we build on past work of connectivity modelling using circuit theory in the region and complement it with land use change modelling that uses a combination of statistical modelling and MCMC-based simulations. Models trained on past land use data were used to project future land use changes and estimate future changes in potential functional connectivity for 5 different umbrella species, under different climate change scenarios. We derive conservation insights for the design of an hypothetical local network of connected protected areas resilient to future landscape change.

This work is the continuation of two important contributions on the connectivity of the region: \cite{albert_applying_2017} and Rayfield et al. (2019, unpublished). In a seminal paper, Albert laid out the some of thr methodological steps we follow: umbrella species selection and connectivity modelling. They also included a simple land use change model that was parameterized to replicate plausible change in the region. Rayfield improved on Albert’s work by increasing the scale of the analysis and reducing the number of focal species. They showed that because species had redundant connectivity needs, modelling the needs for 5 species resulted in qualitatively similar results than when modelling the needs of all 14 species, like in Albert et al. (\citeyear{albert_applying_2017}). They demonstrated that we could exploit this redundancy to reduce the computing time needed for analysis. This is welcomed as land use change simulation is computationally intensive.

We make use of Albert and Rayfield’s framework, modelling connectivity for the 5 focal species identified by Rayfield, and using their workflow for habitat suitability and connectivity analysis. We complement this framework with a land use model that combines statistical modelling and MCMC-based simulations. It is important to note that the primary goal of this chapter is not to draw strong inference with regards to land use change drivers in the region, but to provide enough predictive power to replicate the trends in land use change that have been observed and project those trends forward into the future.

We predict that our land use change model will simulate an overall decrease in connectivity change for our focal species, given the fact that we will be simulating a “business as usual” scenario for the change in the region. In addition, we predict that species who prefers coniferous forest will be more strongly impacted that others.\\ %TODO review this paragraphs on predictions

\section{Methods}
In this section, we explain in detail the workflow for chapter 1. The workflow is divided into two major steps: land use change modelling and connectivity modelling. The data and code for this analysis is available on \href{https://github.com/VLucet/landchange-connectivity-Montérégie}{this GitHub repository}. \\

\subsubsection*{\textit{Software tools and reproducibility}}
All work was conducted in the R statistical software version 3.6.2 "Dark and Stormy Night" \citep[see][]{R}). We used ST-Sim version 2.2.10, scripted in R with the help of the rsyncrosim package version 1.2.0. ST-Sim ran on a Linux (Ubuntu 18.04) via mono 6.4.0.198. Many geospatial analysis were conducted in GRASS GIS 7.8. 

A note should be added on data availability and reproducibility. A few datasets were not obtained from open sources. The Canadian census data was obtained through the University of Toronto's CHASS (Computing in the Humanities and Social Sciences) Data Centre, which makes the Rando Foret analysis not reproducible. In addition, data on protected areas in Monteregie was obtained  via the RMN (Réseaux des milieux naturels protégés du Québec) under a limited license, which makes the Land use change modelling via ST-Sim not fully reproducible. However, the results of those steps are made open, and the subsequent steps (habitat modelling, connectivity analysis) are made fully reproducible (see appendix on reproducibility for the steps to take to reproduce those results). \\ %TODO write appendix 

\subsection{Land Use Change Modelling}

\subsubsection{Background}
Land use change modelling is a prolific subfield of land systems science which has spurred a very large diversity of approaches \citep{dang_review_2016, noszczyk_review_2018}. Beyond the large amount of methods published, the number of applications and software tools that have been built and are available for research purposes is also consequent - although a note must be made on their openness and accessibility, which can vary tremendously \citep{moulds_open_2015}. It is easy to get confused when having to choose the appropriate methods, and papers that compare frameworks and results across tools and methods are rare \citep{pontius_comparing_2008, pontius_comparison_2005}, even if there has been a noticeable effort in the past decades, and  more papers are attempting methodological comparisons \citep{sun_comparison_2018}. Although a full review of land use change modelling methods is beyond the scope of this chapter and this thesis, a few important concepts should be introduced.

A classic, spatially defined land use change model is what can be called a “state change model”, where the landscape is divided in a grid and each grid element (pixel) is assigned a state \citep{daniel_state-and-transition_2016}. Most land use change models rely on some form of remotely sensed land use data in order to be calibrated. Therefore, the list of possible states is derived from the remotely sensed data on which the model will be calibrated. The art of land use change modelling resides in teaching the model how likely each pixel is to transition into another state (or to remain as is), given its current state and given a set of conditions both intrinsic and extrinsic to that pixel. A simple land use change model will assume that the rules of state changes obey markovian laws: i.e. that the next state will always only depend on the current timestep and not previous timesteps. This is a simplification of real landscapes in which the history of the pixel (beyond its present state) can play a large role in the rules governing state change. Probably due to their simplicity, markovian models such as cellular automata have a long history in land use change models and have been shown to deliver accurate results, notably in urban spread modelling \citep{soares-filho_dinamicastochastic_2002, jokar_arsanjani_integration_2013, iacono_markov_2015}. They are yet another application of markov chains, and are fairly easy to set up and to run. Markov chain models represent a phenomenological approach to land use change, because the model learns the rules of transition without requiring an understanding of the mechanisms behind change. The model learns that the state “forest” has a probability of 0.23 to transition into urban if it contains at least 2 urban pixels in a 4 pixel radius (this is an adjacency rule), but it is oblivious to the drivers of land use change.

In comparison to a phenomenological approach, a mechanistic approach to land use change modelling would fully consider that land change happens because of processes of multi-scale decision making, and will attempt to model those processes directly. Agent-Based Models (ABMs) is probably the best example of a mechanistic approach to land use change modelling. ABMs simulate the behavior of individuals in a network of decision processes whose effects are designed to sum up to the observed change \citep{parker_agent-based_2002, filatova_spatial_2013}. Those models are extremely promising for the future of land use change modelling, but tend to be applied at smaller spatial scales as they are more data intensive. In addition, the modelling process is much more involved: it requires a much deeper understanding of the specific drivers of change in the region.

Therefore, land use change models can be classified according to their level of focus on land use change drivers, from phenomenological (or “top down” if you will) to more mechanistics approaches (“bottom up”). Given the simplicity and ease of use of phenomenological models, and the requirements of mechanistics ones, most approaches to land use change modelling fall somewhat in between the two extremes, and a certain number of “hybrid” approaches has been developed \citep{sun_comparison_2018, jokar_arsanjani_integration_2013}. The approach we take in this chapter is, like most land use change models nowadays, not strictly phenomenological and lean on some mechanistic understanding of land use change.

Such an “in-between” land use change model is set up in two steps: a land use suitability analysis and an allocation algorithm. In a nutshell, the suitability analysis step determines how likely each pixel is to acquire a given state by producing a probability surface, predicted based on spatial data, and the allocation step simulates change over that surface. The simulation (allocation) is often a markov process too, but this time informed by the results of the first step. This markov process can be complexified to include things such as time lags and conditions, allowing to distill some mechanistic flavor into the model.

We choose here to present those two steps as separate because they are methodologically distinct in our model. But it is important to note that recent efforts in land use change modelling have managed to better integrate those steps. Those recent methods can integrate suitability and allocation because the model is capable of learning both processes at the same time. We can cite important work on modelling land use change with machine learning techniques such as Neural Networks (NNs) \citep{tayyebi_simulating_2013} or exploiting the potential of non frequentist statistics in the case of Belief Networks (BBNs). BBNs, like BBNs are promising methods but tend to be applied at smaller spatial scales \citep{celio_modeling_2014}.

The two steps framework we just outlined allows to combine the power of markov chains with the flexibility of statistics. Any method can be used to determine the probability surface (although linear and logistic regression techniques are most common), and complex algorithms can be written to determine rules of allocation. The simplicity and flexibility of this framework explains the popularity of common land use change modelling frameworks like the CLUE family of models \citep{verburg_modeling_2002, verburg_combining_2009}. CLUE (Conversion of Land Use and its Effects) models are some of the most commonly used models in land system science. In CLUE, the allocation is determined by the local and regional “demand” for each land use. This demand is a latent variable for the complex and multi-scalar social-environmental processes that determine land use change, and which more mechanistic models attempt to simulate.

This approach still requires a lot more information and in some cases a lot more thinking about drivers of change and processes than simple markov processes. It is also a much more difficult modelling task. Because land use change drivers are spatially structured, variables are more often than not spatially autocorrelated, which further complicates the matter. Researchers have been dealing with those issues with varying levels of rigour. The quick and easy way of selecting variables and dealing with autocorrelation is to select variables based on significance in a full linear model, and training the model in a random subset instead of on the fully spatially auto-correlated dataset. The forward selection of variables is now seen as bad practice. In addition, papers rarely show proof that their training subset is fully rid of spatial autocorrelation.

A better way to tackle spatial autocorrelation, would be to use a more involved data partitioning method. CCV (Conventional Cross-Validation) and SCV (Spatial Cross-Validation) have been successfully used in a recent paper comparing land use change modelling techniques, but for vector-based data \citep{sun_comparison_2018}. However, to our knowledge, there are no clear guidelines on how to conduct such partitioning and fully resolve spatial autocorrelation in land use change models that use suitability analysis for raster-based data. In this chapter, although we conduct basic CCV, we do not provide an improvement in this domain, and remain aware of this limitation.\\

\subsubsection{The RF-CA modelling framework}

In the absence of a solid body of methodological comparisons between statistical frameworks in land use change modelling, it is difficult to decide on the best framework for raster-based models. Our choice of statistical framework was done considering at least three factors: computing time, prediction power and capacity to deal with spatial processes. Computing time is an important factor because fitting models to large datasets can be time consuming. Capacity for dealing with spatial processes varies across methods. For instance, GAMs (generalized additive models) allow the fitting of spatial smoothing functions which can capture spatial structure in the data. Finally, statistical frameworks sit along a continuum that emphasises more or less inference and prediction. A bayesian model will be more easily thought of as a tool for statistical inference, whereas a neural network will be used more with predictive power in mind.

For our land use suitability analysis, we chose Random Forests (RF) as a statistical framework. RFs are a “tree-based machine learning algorithm that generates a “forest” of randomized independent to each other and identically distributed decision trees”. RFs can be easily “grown’ in parallel, which reduces computation time. RFs possess a strong predictive power, but can still provide metrics of variable importance, providing an interesting balance between inference and predictive power.

For the allocation step, we use an advanced cellular automata (CA) modelling tool (markov-chain based) called ST-Sim, a package of the free software SyncroSim. ST-Sim allows for more complex simulation than a classic CA as it allows for complex neighbouring rules with state and age cell tracking. It is also built to handle large datasets and can be scripted to run in parallel. It is important to note that although it is the first time RF is used with ST-Sim (to our knowledge), this hybrid modelling approach (coined “RF-CA”) has been used successfully in other land use change modelling projects \citep{kamusoko_simulating_2015, gounaridis_random_2019}. ST-Sim can be customized to function under multiple different types of inputs. The two main inputs are transition probability and transition targets. Readers should refer to Daniel el al. (\citeyear{daniel_state-and-transition_2016}) for a more in-depth overview of how ST-Sim and SyncroSim functions. \\

\subsubsection{Data sources and preparation}

\subsubsection*{\textit{Land use change data}}
For most modelling frameworks, the raw source of land use change data is remote sensing data. Because producing land use change data from remotely sensed imagery was beyond the scope of this thesis, we looked for a curated dataset. Agriculture and Agri-Food Canada (AAFC) produces a dataset of land use in Canada at a 10 years interval and at the resolution of 30 meters (the resolution is the dimension of a pixel in a spatial grid, or raster data, and is an important property of land use datasets). This dataset provides land use data for the years of 1990, 2000, 2010. We found this data product, referred to hereafter as the AAFC data  to be the right fit for our approach: the data production method of the AAFC  data is consistent, meaning that similar methods have been used to produce all three maps at different time points. Consistency in data production methods is key as it makes the computation of change between time points more reliable.

This data is not without limitations: compared to other data products - such as the Quebec Ministry of the Environment land use dataset or the Statistique Quebec land accounting dataset - the AAFC dataset is not very detailed: only a few land use categories are identified, and important information for suitability analysis such as forest age and forest density are missing. 

Another limitation is the coordinate reference system (CRS) in which this dataset is available: it is projected in a UTM projection system, which is not the standard CRS used by other data products that usually covers Quebec and emanate from open data portals in the province. This makes direct comparison (pixel by pixel) to those others datasets difficult. Finally, the AAFC data does not differentiate between road types, which has implications for habitat suitability analysis (see the habitat suitability analysis section).\\

\subsubsection*{\textit{Forest types and dynamics under climate change}}

The AAFC data does not differentiate between different forest type. However, our connectivity analysis relies on an analysis of habitat suitability, which is reliant on knowing the preferences of different species for different forest types (see section of habitat suitability analysis for more details). In order to integrate dynamics of forest types into our simulations, we relied on the results of a separate set of simulations that were run in LANDIS, a free and open source source designed to model forest growth. Those simulations were ran by Larocque and Rayfield (unpublished report to the Quebec Ministry of the Environment) for the extent of the Saint-Lawrence lowlands, in the context of a  similar analysis of connectivity under land use change. The LANDIS simulations do not include land use change and therefore can not be directly merged into our simulations. Instead, we parameterized the changes in forest types from the results of the LANDIS simulations. A more detailed methodological description of the LANDIS simulations is to be found in the appendix ([add ref to appendix]). %TODO write this appendix with the help of Yan and Guillaume

We source two different set of "parameters" from these simulations: first, forest type data in 2010 was used as a starting point for the forecasts scenarios. Second, a set of transition multipliers were extracted from the simulations and was used to reflect different forest dynamics under different climate change scenarios. These are described in more details in the "Model Execution" section.\\

\subsubsection*{\textit{Explanatory variables}}
A diversity of drivers of land-use change can be found in the literature. Some general group of variables can be identified: an important distinction can be made between physical and social-economical variables. Proxy variable for land use change drivers used in this chapter were selected among commonly recognized predictors of land use change.

A digital elevation model was used to derive elevation data (continuous variable). We used the data product “SRTM 30m” available throughout the Google Earth Engine data portal to generate this variable. This data was already available at a 30 m resolution. 

We used the Canadian Census for the years of 1991, 2001 and 2011, provided by Statistics Canada, to extract two variables: population change (1991 -2001) and average income. The data was not available in a spatial format and had to be turned into a spatial object in R before being rasterized. The data was collected at the lowest aggregation level available: the denomination area (DAs) or the Enumeration area (EAs) depending on the census year (see table \ref{tab:variables} for the full description of variables and data sources).\\

\subsubsection*{\textit{Data preparation}}
Prior to running the model, the data was pre-processed. The first step consisted in reclassifying the AAFC land use dataset into only 5 categories: the 3 categories of our land use change model and supplemental categories of roads and wetlands. Then, all the pixels classified as forest were reclassified as null, and the resulting maps were patched with the forest cover from the LANDIS simulations' starting conditions (state of forest types in 2010). At the end of this process, some cells remained null as the forest cover in 1990 was larger than that of 2010. We filled those NULL values using modal imputation under two different neighbourhood window sizes (13 and 25 pixels).

A second step consisted in the preparation of the explanatory variables. Vector data was rasterized when needed. A non negligeable effort had to be put into wrangling the Canadian Census data, as Statistics Canada does not make available a spatial dataset for their census data. Therefore, the geometries of DAs and EAs had to be matched with the corresponding line in the Census dataset, based on the unique EA and DA IDs. Multiple of these have remained unmatched due to discrepancies between the (spatial) boundaries datasets of EAs/DAs available and the EAs/DAs listed in the StatsCan dataset. This led to areas with NAs values. These areas could not be ignored in our random forest model and had to be imputed. We used mean imputation for lack of a better method but are aware of the large number of issues associated with this practice \citep[see][]{lodder_impute_2014}.

In order to reduce run time as well as memory requirements, we ran the entire simulation at a 90m resolution, which required to aggregate (resample) all of the data in the rasters by a factor of 3, using a majority rule. Finally, all variables were extracted and turn into tabular data. Only relevant pixels (i.e. rows in the dataset) were kept to model each transition. For instance, only forest and agricultural land pixels were kept to model urbanization. The variables were standardized prior to running the model. We used the \verb|tidymodels| framework (version 0.1.0) in R to pre-process our tabular data and run our random forest models. \\

\subsubsection{Preliminary analysis of past land use change}
In an effort to better understand land use patterns in the region, we ran some preliminary analysis on land use change data at the municipality level. Montérégie is made up of 177 different municipalities. We generated land use matrices (amount of each land use category) and land cover transition matrices (1990 to 2010) for each municipality and for the land use categories corresponding to our model. We then performed multivariate analyses by running a first ordination using PCA, and then using Ward clustering % add ref to legendre. 
Because municipalities in Montérégie have variable size, the amount of change in each municipality needs to be standardized. The data was normalized prior to the ordination to reflect relative amounts of change in each municipality. We used the vegan package in R to produce such plots.\\

\subsubsection{Model Execution}

\subsubsection*{Random forest}

\subsubsection*{\textit{Calibration \& validation}}

Land use change data has a tendency to be highly unbalanced, because over a certain time period, the amount of pixels that have transitioned into a new state is usually small compared to the amount that did not transition into a new state. In the case of Random forest models, there are a few ways to deal with such imbalance. In this work, we simply down-sampled our dataset to reach a ratio of 2:1 (i.e. 2 pixels that did not transition for each pixel that did transition). 

The RF model was calibrated on a training partition (70\% of the down-sampled dataset) for the timestep of 1990 to 2000. The model was then spatially validated on a test partition (30\% of the dataset), and then temporally validated for the timestep of 2000 to 2010 (100\% of the dataset for this timestep). \\

\subsubsection*{\textit{Performance evaluation}}

There are a number of ways to evaluate model performance in the context of land use change predictions. We used three different approaches:
\begin{itemize}
  \item We report the $R^{2}$ resulting from the random forest model, as well as variable importance (evaluated as gini impurity score).
  \item We used a ROC curve with Area Under the Curve (AUC) measurement to compare true observed change with the predicted probabilities of transition. This method is widely used in land use change modelling.
  \item We used a Conventional Cross Validation (CCV) method for the reporting of AUC values, with 10 folds, for which we also produce a measure of AUC with ROC curves.\\
\end{itemize}

\subsubsection{\textit{ST-Sim model}}

In the subsequent description, we make the distinction between:
\begin{itemize}
\item{\textit{historic runs}}, parameterized to reproduce the changes that happened between 1990 and 2010 in the region, with a different set of transition targets for 1990-2000 and 2000-2010. 
\item{\textit{forecast runs}}, parameterized to project forward an average of the trends observed in the two decades, with a single set of transition targets (average of the two 1990-2000 and 2000-2010 targets) and for which only certain set of spatial multipliers change depending on climatic scenarios. 
\end{itemize}

ST-Sim is a State and Transition model (STM), and can therefore be provided with straight probabilities of change. However in order to integrate the results of our random forest models, we parameterize ST-Sim differently than a classic STM. All probabilities are in fact set to one, and we leverage the spatial multiplier feature of the software. We provide our fitted probability surfaces (outputs of the RF models) as spatial multipliers. Therefore, each pixel's probability of change is provided by that surface (multiplied by 1).  This is not enough to reproduce past trends in land use change: we need to provide transition targets to the allocation algorithm. These targets are directly based on the amount of land change between 1990 and 2010 for time steps 1990 - 2010. + spatial multipliers derived from the RF previously mentioned

In all scenarios described, ST-Sim was run from 1990 to 2100 with decadal time steps, for 10 iterations (10 MCMC realization).\\

\subsubsection*{\textit{Model definition}}

The ST-Sim model uses three states: Forest, Urban and Agriculture. Forest was further subdivided into classes of age and type, derived from the LANDIS outputs: three forest types (\textbf{decididuous}, \textbf{mixt}, and \textbf{coniferous}) and three age classes (\textbf{young}, \textbf{medium}, and \textbf{old}) for a total of 9 classes.

The model allows for 4 groups of transitions:
\begin{itemize}
\item{\textbf{Urbanisation}}: groups \textbf{deforestation} (Forest $\Rightarrow$ Urban) and \textbf{agricultural loss} (Agriculture $\Rightarrow$ Urban). 
\item{\textbf{Agricultural expansion}} (Forest $\Rightarrow$ Urban).
\item{\textbf{Forest Internals}}: groups the 72 combinations of forest change between the 9 forest internal states.\\
\end{itemize} 

The model takes in 3 different strata. A stratum in ST-Sim refers to a subset of the entire landscape being considered for analysis. Strata are typically used to divide the overall landscape into regions or zones, each of which can have different model parameters specified.


\subsubsection*{\textit{Spatial dependency}}

Under the parameters defined so far, ST-Sim takes care of transitioning the amount of land corresponding to the targets it is fed, and will do it probabilistically across the landscape according to the probability surface (spatial multipliers). There is an additional way to restrict this change via Adjacency parameters, which are essentially neighbourhood rules. Adjacency rules were  defined from basic assumptions about transition spread. The list of neighbourhood rules given for each transition type (or group) is provided in \ref{tab:neigh_rules}.

We added two more conditions for the Forest Internals group: the Time since Transition parameter which sets a minimum time pf 20 years (2 timesteps) in between transitions between Forest  types states. In addition, pathway autocorrelation settings allows to compartmentalize all parameters to a given set of regions.  

\subsubsection*{\textit{Scenario definition}}

We defined 2 land use change scenarios, crossed with 2 different climate change scenarios, plus a control climate scenario. those 6 scenarios in total are designed to  encompass the realm of possible future habitat changes for our 5 focal species.
The two land use change scenarios are:
\begin{itemize}
\item{\textbf{Business As Usual (BAU)}}: the land use change trends are projected forward into the future with no alteration to those trends.
\item {\textbf{Business As Usual + Reforestation (BAU-R)}}: the land use change trends are projected forward into the future, and reforestation is randomly occurring, converting to forest as much land as is lost through urbanization, for every timestep.
\end{itemize} 
The climate scenarios influence the probability of transitions between forest types (i.e. succession and disturbances) they subsequently influence habitat suitability and connectivity. The three climate scenarios are:
\begin{itemize}
\item{\textbf{Historic (HIST)}}: No forest change: forest remains at it was in 2010. This is equivalent to a control treatment.
\item{\textbf{Baseline (BASE)}}: Forest change continues as it does in today's climate. 
\item{\textbf{RCP 8.5 (RCP85)}}: The forest changes according to the emissions scenario RCP 8.5, characterised by increasing green-house gas emissions, high rates of population growth, modest GDP growth and low rates of technological development and uptake. It is the most severe of the future scenarios and you can cite the Fifth Assessment Report of the IPCC \citep{ipcc_summary_2013}.\\
\end{itemize} 

\subsubsection*{\textit{Details of run parameters}}
We ran a total of 7 scenarios:
\begin{itemize}
\item \textbf{Historic run - 1990 to 2010} %TODO explain that forest is 2010 but not changing, just used for comparison.
\item \textbf{Forecast Land use BAU - Historic climate - 2010 to 2100}
\item \textbf{Forecast BAU - Baseline climate - 2010 to 2100}
\item \textbf{Forecast BAU - RCP 8.5 climate - 2010 to 2100}
\end{itemize}
%TODO add reference to a table that summarises all the scenarios (cobiens with those of chapter 2 in the appendix)

\subsection{Connectivity modelling}

\subsubsection{Background}
The simplest way to describe connectivity is as a property of the landscape. Connectivity is the extent to which the landscape facilitates or impedes the movement of organisms. It is a dynamic property: a full definition of connectivity takes into account how it changes across time and space. Connectivity is also a multi-dimensional property. It will manifest itself differently depending on the needs of different species. Because these needs will also vary depending on the time of the year and on their habitat range and preferences, connectivity science is concerned with scale. The scale of a connectivity analysis will often depend on the needs of the species considered. Therefore, from being a dynamic and multi-dimensional property of the landscape, ecological connectivity is also a multi-scalar property. These properties make connectivity modelling interesting, but challenging.

Connectivity modelling is similar to land use change modelling inasmuch as the diversity of methods at the researcher’s disposal is large.  Methods of connectivity modelling are grounded in important concepts, the most important of which is graph theory. Graph theory is the study of graphs, and describes the properties of the elements (edges and vertices) that make up a network. Applying graph theory to a landscape comes down to collapsing that landscape into a network, from which metrics can be computed, and models can be fitted - for instance dispersal models, which makes metapopulation theory another important concept in connectivity. Metapopulation theory describes how populations of a given species in a landscape are connected through dispersal and gene flow, which makes landscape genetics a prolific field for connectivity modelling.

It is important to distinguish structural from functional connectivity. While structural connectivity strictly refers to how animal movement is mediated by features of the landscape in a general way, functional connectivity is based on species preferences for such features. Functional connectivity modelling therefore attempts to model to which extent the landscape meets the need of a specific species or set of species.

There has been a significant push in the recent literature for focusing onto functional connectivity modelling, as it is regarded as a more accurate representation of  the landscape’s capacity to facilitate species movement than simple structural connectivity. However, it is important to recognize that true functional connectivity modelling requires individual animal movement data for fitting and for validation. This type of data is very costly and difficult to acquire, especially when lots of data is needed for a species with a diversity of profile (which is needed to get a full picture of connectivity in a given region). For this reason, much of current functional connectivity modelling, including the methods of this chapter is in fact modelling “potential functional connectivity”, or PFC. PFC models do not necessarily rely on actual movement data like actual functional connectivity models, but are based on others kinds of knowledge of species habitat preferences (for instance expert knowledge of literature reviews). A limitation of PFC modelling is that what the model produces is an informed prediction about potential movement, which is still in need to be verified. Therefore, the quality of the data used for parameterization is extremely important and will determine how much the results can be trusted.

A typical PFC modelling workflow can be broken down to 3 steps: species selection, Habitat suitability modelling and connectivity analysis. The three steps are described below, and the methodology follows closely the methodology of Rayfield et al.\\

\subsubsection{Species Selection}
Species selection is unchanged from Rayfield et al. The 5 species chosen are presented in table \ref{tab:species}.\\

\subsubsection{Habitat Suitability}

\subsubsection*{\textit{}} %TODO complete this
Habitat suitability analysis consists in reclassifying land use data into a resistance surface which is then used to model connectivity at the next step.
Habitat suitability remains mostly unchanged from Rayfield et al. The main difference is that the AAFC data does not contain information on the forest density and age. To remedy this simulation are ongoing in collaboration with NRC (Natural Resources Canada, Boulanger and Larocque 2020), to obtain future changes in Forest types and age.\\

\subsubsection{Connectivity analysis}
The method of choice for our connectivity analysis is a circuit-theory based software called Circuitscape, a free and open software under MIT license developed originally in Python, and now in its 5th version, in Julia \citep{circuitjulia}.  Circuitscape “borrows algorithms from electronic circuit theory to predict connectivity in heterogeneous landscapes”.
For each of the maps produced in the first step, Circuitscape was run in two directions (“wall to wall” run): east to west and north to south. The results for each map were added and an average is taken for all iterations at each timestep.\\

%\subsubsection{Priorisation}
%\textbf{TBD}
%\\

\subsubsection*{\textit{Software tools}}

\section{Results}

\subsection{Analysis of past land use changes in Montérégie}
The clustering and subsequent ordination of the land use change matrices of the 177 municipalities revealed that Montérégie has 5 profiles and land compositions (see figures \ref{fig:clustervals}, \ref{fig:PCAvals}, and \ref{fig:mapvals}):
\renewcommand{\labelitemi}{$\textendash$}
\begin{itemize}[leftmargin=0.5cm]
  \item \textbf{Forest - Dominant}: have the lowest level of fragmentation and are dominated by forest
  \item \textbf{Forest - Agriculture}: still have a healthy amount of forested areas but fragmentation is much more pronounced.
  \item \textbf{Agriculture - Dominant}: forested habitat is scarce and most of the remaining forest is classified as “Trees” in the AAFC dataset (forest fragment of less than 1 hectare)
  \item \textbf{Urban - Medium density}: correspond to the front of the wave of urban sprawling
  \item \textbf{Urban - High density}: urban cores make up most of the municipality
\end{itemize} 

The clustering and subsequent ordination of land use change profiles showed that Montérégie has 4 different profiles (see figures \ref{fig:clustertrans}, \ref{fig:PCAtrans}, and \ref{fig:maptrans}):
\begin{itemize}[leftmargin=0.5cm]
  \item \textbf{Urban Spread / Deforestation}: forest fragmentation is progressing mainly via the growth of urban land (in the west) or villegiatives pressures (in the east)
  \item \textbf{Urban Spread / Agricultural loss}: agriculture is losing ground to urban land
  \item \textbf{Agricultural Expansion / Fragmentation}: forest is losing ground to agriculture in those municipalities where forest is still quite present
  \item \textbf{Agricultural Expansion / Deforestation}: forest is already scare and is being replaced by agricultural lands
These results show interesting regional trends with a front line of fragmentation and deforestation on each side of the region and along the Richelieu river. \\ % (Fig. \ref{fig:map}).
\end{itemize}

\subsection{Random forest Models}

\subsection{Land Use Change Model}

\subsubsection{Model performance}

\subsubsection{Model predictions}

\subsection{Connectivity Modelling}

\subsection{Scenario comparison}

\section{Discussion}

\section{Conclusion}

%---------------------------------------------------------------------------------------------------------------------------------------------------

\newpage
\begin{center}
\section*{Figures \& Tables}
\end{center}

% Figures: values
% Clustering
\begin{figure}[h!]
  \centering
    \includegraphics[width=\textwidth]{figures/clustering_values.png}
  \caption{Results of Ward clustering for land use for municipalities (cut at 5 groups)}
  \label{fig:clustervals}
\end{figure}

% PCA
\begin{figure}[h!]
  \centering
    \includegraphics[width=0.9\textwidth]{figures/PCA_data_profiles.png}
  \caption{Ordination of land use data (proportions) for municipalities. Groups are derived from clustering in \ref{fig:clustervals}}
  \label{fig:PCAvals}
\end{figure}

% MAP
\begin{figure}[h!]
  \centering
    \includegraphics[width=\textwidth]{figures/profiles_land_use.png}
  \caption{Geographical distribution of the 5 profiles identified in \ref{fig:clustervals} and \ref{fig:PCAvals}.}
  \label{fig:mapvals}
\end{figure}

% Figures: Transitions
% Clustering
\begin{figure}[h!]
  \centering
    \includegraphics[width=\textwidth]{figures/clustering_trans.png}
  \caption{Results of Ward clustering for transition data for municipalities (cut at 4 groups)}
  \label{fig:clustertrans}
\end{figure}

% PCA
\begin{figure}[h!]
  \centering
    \includegraphics[width=0.9\textwidth]{figures/PCA_trans_profiles.png}
  \caption{Ordination of land use transition data for municipalities. Groups are derived from clustering \ref{fig:clustertrans}}
  \label{fig:PCAtrans}
\end{figure}

%MAP
\begin{figure}[h!]
  \centering
    \includegraphics[width=\textwidth]{figures/transition_prof_map.png}
  \caption{Geographical distribution of the 4 change profiles identified in \ref{fig:clustertrans} and \ref{fig:PCAtrans}.}
  \label{fig:maptrans}
\end{figure}

%---------------------------------------------------------------------------------------------------------------------------------------------------

% RF variables
\begin{table}[h!]
\centering
\caption{Description and data sources for all variables used in the RF-CA model}
\label{tab:variables}
\begin{tabular}{|l|l|l|}
\hline
\textbf{Variable} & \textbf{Format} & \textbf{Source}  \\ 
\hline
Distance from urban land & \multirow{3}{*}{Raster} & \multirow{2}{*}{Generated from land cover data} \\ \cline{1-1}
Size of forest patch &  &  \\ \cline{1-1} \cline{3-3} 
Elevation &  & \begin{tabular}[c]{@{}l@{}}SRTM 30m from \\ Google Earth Engine \\ data library\end{tabular} \\ \hline
Population change & \multirow{2}{*}{\begin{tabular}[c]{@{}l@{}}Tabular data joined to vector \\ data and rasterized\end{tabular}} & \multirow{2}{*}{\begin{tabular}[c]{@{}l@{}}Canadian Census for \\ 1991, 2001 and 2011\end{tabular}} \\ \cline{1-1}
Income &  &  \\ 
\hline
\end{tabular}
\end{table}

% Neighborhood rules

\begin{table}[h!]
\centering
\caption{Neighborhood rules for ST-Sim transitions}
\label{tab:neigh_rules}
\begin{tabular}{llcc}
\hline
\textbf{Transition} & \textbf{State counted} & \textbf{Neighborhood (m)} & \textbf{Minimum Proportion} \\ \hline
Urbanization & Urban & 500 & 0.5 \\
Agricultural Expansion & Agriculture & 250 & 0.75 \\
Reforestation & Forest (any type) & 225 & 0.5 \\
Forest Internals & Forest (specific type) & 225 & 0.05 \\ \hline
\end{tabular}
\end{table}

%---------------------------------------------------------------------------------------------------------------------------------------------------
% Results tables

% RF Variable importance
\begin{table}[h!]
\centering
\caption{Variable importance (gini impurity index) for both models - non-categorical variables only}
\label{tab:varimp}
\begin{tabular}{lcc}
\hline
\multicolumn{1}{c}{\multirow{2}{*}{\textbf{Variable}}} & \multicolumn{2}{c}{\textbf{Model}} \\ \cline{2-3} 
\multicolumn{1}{c}{} & \multicolumn{1}{l}{\textbf{Urbanisation}} & \multicolumn{1}{l}{\textbf{Agricultural expansion}} \\ \hline
Distance from urban land & 756.8292 & 1787.357 \\
Size of forest patch & 548.5826 & 5373.270 \\
Elevation & 580.9726 & 1823.918 \\
Population change & 447.1120 & 1138.319 \\
Income & 390.5668 & 1108.612 \\ 
\hline
\end{tabular}
\end{table}