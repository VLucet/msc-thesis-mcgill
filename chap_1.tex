\chapter{My first chapter}
\begin{center}
{Valentin Lucet$^{1}$, Andrew Gonzalez$^{1}$}\\
\end{center}
\textit{Author Affiliations:}\\
\normalsize{$^{1}$Department of Biology, McGill University}\\
\section{Abstract}
To be completed.
\\
\section{Introduction}

Space is a finite resource. How we manage and govern space is a reflection of the trade-offs and choices made by people and organizations at different spatial and temporal scales. Those choices primarily determine and regulate land use: if and how space is organized and whether resources are exploited, transformed or conserved. Ecological connectivity, defined as the extent to which the landscape supports the movements of organisms, can be strongly affected by land use. It is an important component of the resilience of populations in heterogeneous and fragmented landscapes. Land use changes such as urban sprawl and agricultural intensification intensify habitat fragmentation and landscape homogenization, leading to the erosion of ecological connectivity. The Monteregie region in southern Quebec, where this work takes place, is experiencing urban growth and sprawl. We present a framework that integrates land-use change and connectivity modeling to forecast future changes in connectivity, using a combination of statistical modeling, MCMC-based simulations, and circuit theory. Models trained on past land use data were used to project future land use changes using different scenarios, and estimate future changes in functional connectivity for 5 different umbrella species. We contrast the past and future impacts of trends in land use (i.e.urbanization, agricultural expansion, and deforestation) and derive conservation priorities for the design of a local network of connected protected areas resilient to future landscape change. We demonstrate the flexibility of a scenario approach in forecasting the range of possible futures for ecological connectivity in the region and show that taking probable landscape changes into account lead to different conservation priorities. We conclude on the importance of considering such changes to produce a resilient network of protected areas and highlight the need for a multi stakeholder approach in the definition of scenarios and conservation priorities

\section{Methods}

\subsection*{Study Site}
To be completed.
\\
\subsection*{Sampling}
To be completed.
\\
\subsection*{Analyses}
To be completed.
\\
\section{Results}
To be completed.
(Fig. \ref{fig:map}).
\\
\section{Discussion}
\lipsum[66]
\\

\newpage
\section*{Figures \& Tables}

\begin{figure}[!ht]
  \centering
    \includegraphics[width=0.5\textwidth]{../land_con_monteregie/outputs/figures/one_to_one_targets.png}
  \caption{Figure 1}
  \label{fig:map}
\end{figure}

\begin{table}[h!]
\centering
\begin{tabular}{llllll|l}
  \hline
  \hline
  Site & Weeks & S & XS & Daily & A/B & Total\\
  \hline
  Nhlanguleni (NHL) 	& 3 & 0 & 0 & 0 & Yes & 6 \\
  Nwaswitshaka (NWA) 	& 3 & 1 & 1 & 4 & Yes & 18 \\
  De LaPorte (DLP) 		& 1 & 1 & 1 & 0 & Yes & 6 \\
  Kwaggas Pan (KWA) 	& 2 & 1 & 1 & 0 & Yes & 8\\
  Girivana (GIR) 		& 3 & 0 & 0 & 0 & Yes & 6 \\
  Witpens (WIT) 		& 3 & 0 & 0 & 0 & Yes & 6 \\
  Imbali (IMB) 			& 3 & 0 & 0 & 0 & Yes & 6 \\
  Hoyo Hoyo (HOY) 		& 3 & 1 & 1 & 0 & Yes & 10 \\
  Nyamarhi (NYA) 		& 3 & 1 & 1 & 0 & Yes & 10 \\
  Ngosto North (NGO) 	& 3 & 1 & 1 & 0 & Yes &10 \\
  BLANK 				& 2 & 0 & 0 & 0 & No & 2 \\
  \hline
   						& 29 & 6 & 6 & 4 & & 88 \\
  \hline
  \hline
\end{tabular}
\caption{Table 1.}
\label{tab:samples}
\end{table}
