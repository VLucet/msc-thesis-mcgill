\chapter{My first chapter}
\begin{center}
{Valentin Lucet$^{1}$, Andrew Gonzalez$^{1}$}\\
\end{center}
\textit{Author Affiliations:}\\
\normalsize{$^{1}$Department of Biology, McGill University}\\
\section{Abstract}
Space is a finite resource. How we manage and govern space is a reflection of the trade-offs and choices made by people and organizations at different spatial and temporal scales. Those choices primarily determine and regulate land use: if and how space is organized and whether resources are exploited, transformed or conserved. Ecological connectivity, defined as the extent to which the landscape supports the movements of organisms, can be strongly affected by land use. It is an important component of the resilience of populations in heterogeneous and fragmented landscapes. Land use changes such as urban sprawl and agricultural intensification intensify habitat fragmentation and landscape homogenization, leading to the erosion of ecological connectivity. The Monteregie region in southern Quebec, where this work takes place, is experiencing urban growth and sprawl. We present a framework that integrates land-use change and connectivity modeling to forecast future changes in connectivity, using a combination of statistical modeling, MCMC-based simulations, and circuit theory. Models trained on past land use data were used to project future land use changes using different scenarios, and estimate future changes in functional connectivity for 5 different umbrella species. We contrast the past and future impacts of trends in land use (i.e.urbanization, agricultural expansion, and deforestation) and derive conservation priorities for the design of a local network of connected protected areas resilient to future landscape change. We demonstrate the flexibility of a scenario approach in forecasting the range of possible futures for ecological connectivity in the region and show that taking probable landscape changes into account lead to different conservation priorities. We conclude on the importance of considering such changes to produce a resilient network of protected areas and highlight the need for a multi stakeholder approach in the definition of scenarios and conservation priorities.
\\
\section{Introduction}
Space is a finite resource. How we, as a community, manage and govern space is a reflection of the trade-offs and choices made by different people and organizations at different spatial and temporal scales. Those choices determine and regulate land use: if and how the resources held on the land are exploited, transformed or conserved.  The results of those choices, referred to as land use and land cover change, is an important threat to the biodiversity and ecosystem function.
One example of ecosystem function affected by land use that is of crucial importance to biodiversity is ecological connectivity. Ecological connectivity is the extent to which the landscape supports the movements of organisms (Gonzalez et al. 2017), and is paramount for the resilience of both populations and ecosystem services (Mitchell et al. 2015) in heterogeneous and fragmented landscapes. Land use changes such as urban sprawl can cause deforestation, fragmenting habitats, and slowly eroding ecological connectivity. Many urban landscapes are experiencing uncontrolled urban sprawl and have suffered losses in connectivity and ecosystem services in consequence. Examples include cities like Barcelona (Marulli and Mallarach 2005), New York City (McPhearson et al. 2014), and also Montreal (Dupras and Alam 2015).  The forces behind those land use changes are complex and understanding them is an obstacle to conservation planning  (Worboys et al. 2010).  Because land use change is a social process with consequences of both social and ecological nature, it is best understood within the concept of social-ecological system (Ostrom 2009). A social-ecological system (SES) can be understood as the set of human and non-human actors, the set of natural habitats they inhabit and resources and use, and the set of interactions that are maintained between all the components of the system. SESs thus form complex and integrated aggregates of interactions (Hinkel et al. 2014). Those interactions also impact governance, the process by which actors in power establish rules and laws (Bissonnette et al. 2018).
Connectivity conservation planning refers to the enterprise that engages multiple actors such as academics, NGOs, governmental bodies at different scales and in a common goal to conserve the ecological connectivity of the landscape. Connectivity conservation methods usually entails modelling connectivity of the landscape of interest and using a prioritisation method to determine conservation priorities. However, current connectivity conservation planning methods have at least two major limitations: their prioritisation process fails to take into account risks associated with future land use change, and also fails to confront the results of the prioritisation with the priorities perceived by stakeholders. Not taking into account risks of land use change would in theory lead to ill-informed conservation planned that would be over-optimistic with regard to their probability of success. In addition, failing to integrate the perceptions of stakeholders is detrimental to conservation for two reasons: first, it most likely mean that conservation will fail to gather enough local momentum to lead to actual policy change, and second, it means that the only tool for decision making will be the modelisations, whereas these are incomplete representation of the landscape and would benefit from inputs from stakeholders. This is especially true in landscapes where a considerable effort of connectivity modelling has already been conducted, like in the landscape of interest in this thesis, the southern Quebec region of Montérégie.
Montérégie is situated southeast of the city of Montreal, and contains parts of the Greater Montreal Area (GMA). The ecological connectivity of the GMA and its benefits as a provider of ecosystem services has recently been assessed in a report to the Quebec ministry of the environment (Rayfield 2018, unpublished). This study focused on identifying regions of highest connectivity, and therefore of highest priority for the conservation of biodiversity and ecosystem services. Other work by Rayfield et al. (2019, unpublished) has extended the analysis of connectivity to the whole of the Saint Lawrence Lowlands.
Although the map produced by this analysis is a snapshot of the current state of connectivity in the region, methods are available for including future land use and climate change impacts (Albert et al. 2017). Those methods rely on established the use of land use and land cover change models whose complexity has increased from simple probabilistic state transition models to more advanced approaches using targets, discrete events and accounting for the time elapsed since the last transition  (Verburg and Overmars 2009, Daniel et al. 2016).
Methods are also available to include stakehodler’s input in connectivity conservation planning. Some of those methods have been developed through the methodology of participatory modeling, which can be defined as a modelling framework that can integrate knowledge from multiple sources, even if this knowledge is generated by different processes. For instance, it is possible landscape perceptions by different actors and quantitative modelling. Those methods often rely on collecting data through a community-driven process during workshops. Those methods are time consuming and require a long term engagement with a given community over many years. Other workshop-based methods are less involved, and allow researchers to simply collect data to be confronted with the results of traditional modeling techniques.
In this thesis, we show an attempt to remedy the two issues we identified above,  in an ongoing effort of connectivity conservation planning for the region of Montérégie in Southern Quebec, Canada. In the first chapter, we built on past work of connectivity modelling using circuit theory in the region and complemented it with land use change modelling that uses a combination of statistical modeling and MCMC-based simulations. In the second chapter, we compare those results with the perceived conservation priorities in the landscape, using data collected during a day-long workshop with stakeholders. The Montérégie is relevant for our questions given the recent political momentum gained by connectivity conservation. There is a strong political will in the region for the conservation of ecological connectivity.
\\
\section{Methods}

\subsection*{Study Site}
To be completed.
\\
\subsection*{Sampling}
To be completed.
\\
\subsection*{Analyses}
To be completed.
\\
\section{Results}
To be completed.
(Fig. \ref{fig:map}).
\\
\section{Discussion}
\lipsum[66]
\\

\newpage
\section*{Figures \& Tables}

\begin{figure}[!ht]
  \centering
    \includegraphics[width=0.5\textwidth]{../land_con_monteregie/outputs/figures/one_to_one_targets.png}
  \caption{Figure 1}
  \label{fig:map}
\end{figure}

\begin{table}[h!]
\centering
\begin{tabular}{llllll|l}
  \hline
  \hline
  Site & Weeks & S & XS & Daily & A/B & Total\\
  \hline
  Nhlanguleni (NHL) 	& 3 & 0 & 0 & 0 & Yes & 6 \\
  Nwaswitshaka (NWA) 	& 3 & 1 & 1 & 4 & Yes & 18 \\
  De LaPorte (DLP) 		& 1 & 1 & 1 & 0 & Yes & 6 \\
  Kwaggas Pan (KWA) 	& 2 & 1 & 1 & 0 & Yes & 8\\
  Girivana (GIR) 		& 3 & 0 & 0 & 0 & Yes & 6 \\
  Witpens (WIT) 		& 3 & 0 & 0 & 0 & Yes & 6 \\
  Imbali (IMB) 			& 3 & 0 & 0 & 0 & Yes & 6 \\
  Hoyo Hoyo (HOY) 		& 3 & 1 & 1 & 0 & Yes & 10 \\
  Nyamarhi (NYA) 		& 3 & 1 & 1 & 0 & Yes & 10 \\
  Ngosto North (NGO) 	& 3 & 1 & 1 & 0 & Yes &10 \\
  BLANK 				& 2 & 0 & 0 & 0 & No & 2 \\
  \hline
   						& 29 & 6 & 6 & 4 & & 88 \\
  \hline
  \hline
\end{tabular}
\caption{Table 1.}
\label{tab:samples}
\end{table}
