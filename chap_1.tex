% !TEX root = ./thesis.tex
\chapter{Integrating Land Use Change Modelling with Connectivity Modelling}
\begin{center}
{Valentin Lucet$^{1}$, Andrew Gonzalez$^{1}$}\\
\end{center}
\textit{Author Affiliations:}\\
\normalsize{$^{1}$Department of Biology, McGill University}\\
\section{Abstract}

Ecological connectivity, defined as the extent to which the landscape supports the movements of organisms, can be strongly affected by land use. It is an important component of the resilience of populations in heterogeneous and fragmented landscapes. Land use changes such as urban sprawl and agricultural intensification intensify habitat fragmentation and landscape homogenization, leading to the erosion of ecological connectivity. The Montérégie region in southern Quebec, where this work takes place, is experiencing urban growth and sprawl. We present a framework that integrates land-use change and connectivity modelling to forecast future changes in connectivity, using a combination of statistical modelling, MCMC-based simulations, and circuit theory. We used a hybrid modelling approach to project future land use changes using different climate scenarios, and estimate future changes in functional connectivity flow for 5 different umbrella species. We explore the flexibility of a scenario approach in forecasting the range of possible futures for ecological connectivity in the region, derive insights for the hypothetical design of a local network of connected protected areas resilient to future landscape change. In conclusion, we highlight the need for a multi stakeholder approach in the definition of scenarios and conservation priorities.\\

\section{Introduction}
\vspace{2em}

%[intro paragraph, andy said it is good]
The goal of connectivity conservation planning is to preserve the continuity of habitat throughout a given landscape, by identifying priority areas and corridors important for the preservation of ecological connectivity \citep{keeley_thirty_2019}. Ecological connectivity, defined as the extent to which the landscape facilitates or impedes the movement of organisms \citep{crooks_landscape_2006}, is a critical component of the resilience of populations in heterogeneous and fragmented landscapes \citep{gonzalez_spatial_2017}. Connectivity conservation planning methods do not typically account for risks associated with future land use and climate change, two main drivers of the erosion of ecological connectivity. Here, we ask whether those methods can be improved by taking into account those risks. To do so, we integrate a connectivity model with a land use change model including climate scenarios. 

% [CC as an adaptation measure, but CC designs are vulnerable]
Connectivity conservation has often been pointed out as a method of adaptation against both climate and land use change \citep{costanza_landscape_2019}. One recurring argument is that preserving landscape connectivity could be critical for ensuring the adequate protection of shifting species ranges \citep{krosby_ecological_2010, keeley_making_2018} due to climate change. Another is that it is more important than ever to connect habitats in landscapes increasingly hostile to species movement \citep{ellis_anthropogenic_2010}. These arguments make the case for the design of Connected Protected Area Networks (CPAN, \cite{stewart_corridors_2019}). Those networks are meant to allow the movement of species across multiple scales, which may require them to include both natural and semi-natural habitat, with different levels of protections \citep{daloia_coupled_2019}. This heterogeneity is likely to be reinforced by the multiplicity of urban and land use planning practices that exist on the landscape, making CPAN designs particularly vulnerable to changes in land use. It is therefore important to test which specific CPAN design would be most resilient to future  change.  One possible approach to exploring this resilience is to use landscape-level forecast models.

%[Land use change and climate models often  used in conservation]
Land use change models have often used in conservation planning (for example, see \cite{echeverria_spatially_2008} or \cite{lesschen_identification_2007}). They allow researchers to test the efficiency of a given conservation plan in the face of expected land use change. Similarly, climate models are often used in conservation, for example to assess whether protected areas will be able to track the current changes in species ranges due to climate change \citep{araujo_would_2004, heller_targeting_2015}. They are central in species distribution models \citep{porfirio_improving_2014}, a classic tool in conservation research. 

% [what about those models in CC context?]
Both types of models are increasingly common In approaches that focus on ecological connectivity. Connectivity researchers have long been interested how past changes in land use can explain current connectivity patterns \citep{dupras_urban_2015, henareh_khalyani_spatial_2013, vergara_deforestation_2013, patru-stupariu_using_2013} and multiple connectivity measures have been developed, one specifically in this optic \citep{saura_new_2007}. But beyond these historical analyses, land use forecast models are becoming more common \citep{correa_ayram_habitat_2015}. For example, \cite{rubio_sustaining_2012}  aimed at identifying a network of forest patches resilient to land use change under multiple scenarios in Spain. Similarly, \cite{piquer-rodriguez_future_2012} integrated land use change and connectivity models to identify under-protected areas by projecting land use change trends. \cite{huang_simulating_2018} also used land use change models to assess potential impacts on functional connectivity in China. From these examples, we can see that combining land use change modelling and connectivity modelling does not yet have a fully established methodology. In its emerging form, it simply consists of running a land use change model on a given landscape, and comparing changes in the connectivity measure of choice between time steps \citep{perkl_urban_2018}. 

% [back to what we want to do]
Climate change can interact with land use change in destructive ways for biodiversity \citep{oliver_interactions_2014} It is therefore likely that connectivity models integrating these two drivers would improve connectivity conservation methods such as CPAN design. Yet, in a recent review, \cite{costanza_landscape_2019} have found that connectivity models frameworks rarely combine  land use change and climate change. Here, we build on recent connectivity modelling work using circuit theory, by pairing it with a hybrid land use and climate change model. The model we develop uses a combination of statistical modelling and simulations based on a Markov Chain Monte Carlo (MCMC) method, combining Random forests (RF) models with a Cellular automaton (CA) model. We use this RF-CA model to project future land use changes and estimate the changes in potential functional connectivity for five different umbrella species, under different climate change scenarios. We also derive conservation insights for the design of a hypothetical regional network of connected protected areas resilient to future landscape change. In addition, we provide a cursory analysis of past land use change trends in our study region, the Monteregie region in southern Quebec. 

%[we are building on past work, this work determined that we could do 5 species instead of 14]
This work is the continuation of recent research efforts on the connectivity of the region: \cite{albert_applying_2017} and \citep{rayfield_priorisation_2018}. In a seminal paper, \cite{albert_applying_2017} laid out some of the methodological steps we follow such as umbrella species selection and habitat suitability analysis. They also included a simple land use change model that was parameterized to replicate plausible change in the region. \cite{meurant_selecting_2018} developed the criteria for reducing the number of focal species from fourteen to five. They showed that because species had redundant connectivity needs, modelling the needs for 5 species resulted in qualitatively similar results than when modelling the needs of all 14 species, such as in Albert et al. (\citeyear{albert_applying_2017}). While Rayfield et al. (\citeyear{rayfield_priorisation_2018}) extended the spatial scale of the analysis to include the Saint Lawrence lowlands ecoregion, they demonstrated that we could exploit this redundancy to reduce the computing time needed for the connectivity analysis. This is an important development because land use change simulations are also computationally intensive.

% \subsubsection*{\textit{Modelling potential functional connectivity}}
%[description of connectivity]
Connectivity is a dynamic property of the landscape: it changes across time and space \citep{beyer_functional_2013}. It is also scale and species specific, varying according to the needs of each species and because those needs also vary depending on the time of the year, life history, and on the species’ habitat range and preference \citep{anderson_scale-dependent_2005}. It is important to distinguish structural connectivity from functional connectivity \citep{kindlmann_connectivity_2008}. While structural connectivity strictly refers to how animal movement is mediated by the configuration of habitat features in the landscape, functional connectivity is based on species’ preferences for such features and how they move among them. Functional connectivity modelling therefore attempts to model the extent to which the landscape meets the need of a specific species or set of species.

%[functional versus PFC]
There has been a trend in the recent literature to focus on functional connectivity, as it is regarded as a more realistic representation of  the landscape’s capacity to facilitate species movement than structural connectivity. However, functional connectivity analyses require direct animal movement data or genetic data for model fitting and for validation (for example, see \cite{beyer_functional_2013, milanesi_three-dimensional_2017}). This type of data is costly to acquire, especially when needed for many species. For this reason, much of current functional connectivity modelling, including the methods of this chapter are in fact modelling “potential functional connectivity”, or PFC. PFC models do not necessarily rely on actual movement data, but are based on other kinds of knowledge of species habitat preferences and movement behaviors, for instance, from expert knowledge and literature reviews. This makes PFC well suited to the task of this paper. 

%[we combine the PFC and redundancy concepts]
Here we model  PFC for the 5 focal species identified by  \cite{meurant_selecting_2018} and use the workflow of \citep{rayfield_priorisation_2018} for habitat suitability and connectivity analysis. We complement this framework with the RF-CA land use model. It is important to note that the primary goal of this chapter is not to explain the drivers of land use change in the region, but to provide a sufficiently realistic simulation to replicate the trends in land use change that have been observed historically, and to project those trends forward into the future.

%\subsubsection*{\textit{A hybrid land use change model}}
%[what is land use change modelling]
A classic spatially defined land use change model is what can be called a “state change model”, where the landscape is divided in a grid and each grid element (pixel) is assigned a state that changes in discrete time \citep{daniel_state-and-transition_2016}. The difficulty of land use change modelling resides in setting the transition probabilities that define the probability of a pixel transitioning to another state (or to remain as is), given its current state and given a set of conditions both intrinsic and extrinsic to that pixel.

%[phenomenological models]
A simple land use change model will assume that the rules of state changes obey markovian laws: i.e. that the next state will always only depend on the current timestep and not previous timesteps. This is a simplification: in real landscapes, the history of the pixel (beyond its present state) can play a large role in the rules governing state change. Probably due to their simplicity, markovian models such as cellular automata have a long history in land use change models \citep{agarwal_review_2002}. They have been shown to deliver accurate results, notably in urban spread modelling \citep{soares-filho_dinamicastochastic_2002, jokar_arsanjani_integration_2013, iacono_markov_2015}. They are yet another application of markov chains, and are fairly easy to set up and to run. Markov chain models represent a phenomenological approach to land use change, because the model applies the rules of transition without encoding the mechanisms underlying the  change. For example, the model defines that the state “forest” has a probability of 0.23 to transition into urban if it contains at least 2 urban pixels in a 4 pixel radius (this is an adjacency rule), but it is oblivious to the drivers of land use change.

%[mechanistic models]
In comparison to the phenomenological approach of cellular automata, a mechanistic approach to land use change modelling would fully consider that land change happens because of processes of multi-scale decision making, and will attempt to model those processes directly. Agent-Based Models (ABMs) are probably the best example of a mechanistic approach to land use change modelling. ABMs simulate the behavior of individuals in a network of decision processes whose effects are designed to sum up to the observed change \citep{parker_agent-based_2002, filatova_spatial_2013}. Those models are promising for the future of land use change modelling, but tend to be applied at smaller spatial scales, and tend to rely on expert knowledge or on a rather involved data collection procedure (such as interviews with actors, \cite{taylor_agent-based_2016}). This makes the the modelling process more difficult (albeit rewarding) as it requires a deeper understanding of the specific drivers of change in the region.

%[no guidelines, choice of model]
In the absence of a solid body of methodological comparisons between statistical frameworks in land use change modelling, it is difficult to decide on the best framework for raster-based (i.e. based on grids) models. The RF-CA model presented in this chapter is a  hybrid approach in which a statistical approach is taken (statistical modelling through random forest) and combined with a more mechanistic approach (a cellular automaton with transition targets and spatial dependency rules, CA). The RF-CA model is similar to other popular approaches like the CLUE family of models (Conversion of Land Use and its Effects \cite{verburg_modeling_2002, verburg_combining_2009}), which are some of the most commonly used models in land system science. However In CLUE, the allocation is determined by the local and regional “demand” for each land use. This demand is a latent variable for the complex and multi-scalar social-environmental processes that determine land use change, and which more mechanistic models attempt to simulate \citep{verburg_combining_2009}.

% [2 steps but there are other better integrated alternatives]
We choose here to present the two steps of probability modelling and allocation as separate because they are methodologically distinct in our model. But it is worthwhile to note that recent efforts in land use change modelling have managed to integrate those steps. Those recent methods can integrate suitability and allocation because the model is capable of encoding both processes at the same time. We can mention important work on modelling land use change with machine learning techniques such as Neural Networks (NNs) \citep{tayyebi_simulating_2013} or exploiting the potential of non frequentist statistics in the case of Belief Networks (BBNs, see \cite{celio_modeling_2014} for a great example). BBNs, like ABMs are promising methods but tend to be applied at smaller spatial scales.

%\subsubsection*{\textit{A scenario-based approach}}
% [scenarios]
 Scenarios are a flexible approach to forecasting the range of possible futures in landscapes \citep{peterson_scenario_2003}. We integrate land use change and connectivity modelling under two “warming” scenarios (Baseline and RCP 8.5) and one “control” scenario (Historic). 
Depending on the scenario, the model generates different forest change dynamics. The Historic scenario assumes no change in forest composition and stands as our control. While the baseline scenario assumes no further warming and status quo forest change dynamics, the RCP 8.5 scenario is one of the most severe climate change scenarios, and the trends observed the in the baseline scenario are exacerbated \cite{IPCC}. We also derive two simple land use change scenarios: Business as usual (BAU) and Reforestation. Those two scenarios are meant to represent two extremes in land use change, and it is likely that the most probable trajectories for our landscape will be within those two boundaries. In combination with the climate scenarios, this brings our total scenario number to 6.
What predictions can be made as to how connectivity will be affected under our model? There has been a significant research effort to study and predict forest dynamics under climate change in Quebec. It has been shown that the distributions of many of the most abundant trees occurring in the region are out of equilibrium with climate at the margin of their range \citep{talluto_extinction_2017}. Some papers points to existing and expected lags in the response of forest to climate change \citep{savage_elevational_2015,}, and others have commented on how disturbances and forest management practices are likely to fasten and facilitate range shifts \citep{leithead_northward_2010, boulanger_climate_2019, vieira_paying_2020}. A major prediction that can be made is that although boreal type (coniferous) forest is likely to respond more slowly to climate range than deciduous type, both coniferous and mixed population types in southern Quebec are likely to become more rarer, and be replaced by deciduous stands. 
We can therefore derive different predictions as to how connectivity will be affected for each species under our model. Three main predictions can be made. First, under the BAU scenario, we can expect that our land use change model will simulate an decrease in connectivity change for our focal species. We can also expect connectivity to maintain itself under the Reforestation scenario. Finally. we can expect climate change to impact more strongly species that prefer coniferous  and mixed forest types. \\

\section{Methods}

Our workflow is divided into two major steps: land use change modelling and connectivity modelling. Two workflow figures describe the steps and methods involved (figure \ref{fig:workflow1} and \ref{fig:workflow2}). \\

\subsubsection*{\textit{Software tools and reproducibility}}

All work was conducted in the R statistical software version 3.6.2 "Dark and Stormy Night" \citep[see][]{R}). The data and code for this analysis are available at \href{https://github.com/VLucet/landchange-connectivity-Montérégie}{this GitHub repository}.

We used ST-Sim version 2.2.10, scripted in R with the help of the rsyncrosim package version 1.2.0. ST-Sim ran on Linux (Ubuntu 18.04) via Mono 6.4. Many steps of data preparation were conducted in GRASS GIS 7.8.

A note should be added on data availability and reproducibility. A few datasets were not obtained from open sources. These datasets are marked in red in figure \ref{fig:workflow1}. The Canadian census data was obtained through the University of Toronto's CHASS (Computing in the Humanities and Social Sciences) Data Centre, which requires institutional access. This makes the Random Forest analysis not reproducible. In addition, data on protected areas in Montérégie was obtained  via the RMN (Réseaux des milieux naturels protégés du Québec) under a limited license, which makes the Land use change modelling via ST-Sim not fully reproducible. However, the results of those steps are made open, and the subsequent steps (habitat modelling, connectivity analysis) are reproducible via the use of various software ( i.e. docker, renv, and more). Please consult the appendix on reproducibility for the steps to take to reproduce those results. \\ %TODO write appendix on reproducibility

\subsection{Land Use Change Modelling}
Land use change modelling is a prolific subfield of land systems science which has spurred a broad diversity of approaches \citep{dang_review_2016, noszczyk_review_2018}. Beyond the many methods published, there is a large number of applications and software tools available for research - although their openness and accessibility varies \citep{moulds_open_2015}. Papers that compare frameworks and results across tools and methods are rare \citep{pontius_comparing_2008, pontius_comparison_2005, sun_comparison_2018}. Although a full review of land use change modelling methods is beyond the scope of this chapter and this thesis, I will introduce the key concepts.

Land use change models can be classified according to their level of focus on land use change drivers, from phenomenological (“top down”) to more mechanistic approaches (“bottom up”). Given the simplicity and ease of use of phenomenological models, and the extra demands of mechanistic models, most approaches to land use change modelling fall between these two extremes, and a certain number of “hybrid” approaches have been developed \citep{sun_comparison_2018, jokar_arsanjani_integration_2013}. The approach we take in this chapter is not strictly phenomenological and introduces some mechanistic understanding of land use change, and can therefore be qualified as a hybrid method.

Our land use change model is set up in two steps: a land use suitability analysis and an allocation algorithm. In a nutshell, the suitability analysis step determines how likely each pixel is to acquire a given state by producing a probability surface, predicted based on spatial data, and the allocation step simulates change over that surface. The allocation is a Markov process informed by the results of the first step. This Markov process is embellished to include conditions such as neighboring rules and transition size distributions.

The two steps framework we just outlined allows us to combine the power of markov chains with the flexibility of statistics. Any method can in theory be used to determine the probability surface (although linear and logistic regression techniques are most common), and complex algorithms can be written to determine rules of allocation. The simplicity and flexibility of this framework explains the popularity of common land use change modelling frameworks like CLUE  (Conversion of Land Use and its Effects \cite{verburg_modeling_2002, verburg_combining_2009}). \\

\subsubsection{The RF-CA modelling framework}

Our choice of statistical framework was done considering at least three factors: computing time, prediction power and capacity to deal with spatial processes. Computing time is an important factor because fitting models to large datasets can be time consuming. Capacity for dealing with spatial processes varies across methods. Finally, statistical frameworks sit along a continuum that emphasises more or less inference and prediction. A Bayesian model will be more easily thought of as a tool for statistical inference, whereas a neural network will be used more with predictive power in mind. 

We chose Random Forests (RF) as a statistical framework. RFs are a “tree-based machine learning algorithm that generates a “forest” of randomized independent to each other and identically distributed decision trees”. RFs can be easily “grown’ in parallel, which reduces computation time. RFs possess a strong predictive power, but can still provide metrics of variable importance, providing an interesting balance between inference and predictive power.

For the allocation step, we use an advanced cellular automata (CA) modelling tool (markov-chain based) called ST-Sim, a package of the free software SyncroSim. ST-Sim allows for more complex simulation than a classic CA as it allows for complex neighbouring rules with state and age cell tracking. It is also built to handle large datasets and can be scripted to run in parallel. It is important to note that although it is the first time RF is used with ST-Sim (to our knowledge), this hybrid modelling approach (coined “RF-CA”) has been used successfully in other land use change modelling projects \citep{kamusoko_simulating_2015, gounaridis_random_2019}. ST-Sim can be customized to function under multiple different types of inputs. The two main inputs are transition probability and transition targets. Readers should refer to Daniel el al. (\citeyear{daniel_state-and-transition_2016}) for a more in-depth overview of how ST-Sim and SyncroSim functions. \\

\subsubsection{Data sources and preparation}

\subsubsection*{\textit{Land use change data}}

For most modelling frameworks, the raw source of land use change data is remote sensing data. Because producing land use change data from remotely sensed imagery was beyond the scope of this thesis, we looked for a curated dataset. Agriculture and Agri-Food Canada (AAFC) produces a dataset of land use in Canada at a 10 years interval and at the resolution of 30 meters (the resolution is the dimension of a pixel in a spatial grid, or raster data, and is an important property of land use datasets). This dataset provides land use data for the years of 1990, 2000, 2010. We found this data product, referred to hereafter as the AAFC data  to be the right fit for our approach: the data production method of the AAFC  data is consistent, meaning that similar methods have been used to produce all three maps at different time points. Consistency in data production methods is key as it makes the computation of change between time points more reliable.

This data is not without limitations: compared to other data products - such as the Quebec Ministry of the Environment land use dataset or the Statistique Quebec land accounting dataset - the AAFC dataset is not very detailed: only a few land use categories are identified, and important information for suitability analysis such as forest age and forest density are missing.

Another limitation is the coordinate reference system (CRS) in which this dataset is available: it is projected in a UTM projection system, which is not the standard CRS used by other data products that usually covers Quebec and emanates from open data portals in the province. This makes direct comparison (pixel by pixel) to those others datasets difficult. Finally, the AAFC data does not differentiate between road types, which has implications for habitat suitability analysis (see the habitat suitability analysis section).\\

\subsubsection*{\textit{Forest types and dynamics under climate change}}

The AAFC data does not differentiate between different forest types. However, our connectivity analysis relies on an analysis of habitat suitability, which is reliant on knowing the preferences of different species for different forest types (see section of habitat suitability analysis for more details). In order to integrate dynamics of forest types into our simulations, we relied on the results of a separate set of simulations that were run in LANDIS, a free and open source source designed to model forest growth. Those simulations were run by Larocque and Rayfield (unpublished report to the Quebec Ministry of the Environment) for the extent of the Saint-Lawrence lowlands ecoregion, in the context of a research contract with the MELCC (Ministère de l'Environnement et de la Lutte contre les changements climatiques). The LANDIS simulations do not include land use change and therefore can not be directly merged into our simulations. Instead, we parameterized the changes in forest types from the results of the LANDIS simulations. A more detailed methodological description of the LANDIS simulations is to be found in the appendix ([add ref to appendix]). %TODO write this appendix with the help of Yan and Guillaume

We source two different sets of "parameters" from these simulations: first, forest type data in 2010 was used as a starting point for the forecast scenarios. Second, a set of transition multipliers were extracted from the simulations and was used to reflect different forest dynamics under different climate change scenarios. These are described in more details in the "Model Execution" section.\\

\subsubsection*{\textit{Explanatory variables}}

A diversity of drivers of land-use change can be found in the literature. Some general group of variables can be identified: an important distinction can be made between physical and social-economical variables. Proxy variables for land use change drivers used in this chapter were selected among commonly recognized predictors of land use change.

A digital elevation model was used to derive elevation data (continuous variable). We used the data product “SRTM 30m” available throughout the Google Earth Engine data portal to generate this variable. This data was already available at a 30 m resolution.

We used the Canadian Census for the years of 1991, 2001 and 2011, provided by Statistics Canada, to extract two variables: population change (1991 -2001) and average income. The data was not available in a spatial format and had to be turned into a spatial object in R before being rasterized. The data was collected at the lowest aggregation level available: the denomination area (DAs) or the Enumeration area (EAs) depending on the census year (see table \ref{tab:variables} for the full description of variables and data sources).\\

\subsubsection*{\textit{Data preparation}}

The first step of preparing the data from the model consisted in reclassifying the AAFC land use dataset into only 5 categories: the 3 categories of our land use change model and supplemental categories of roads and wetlands. Then, all the pixels classified as forest were reclassified as null, and the resulting maps were patched with the forest cover from the LANDIS simulations' starting conditions (state of forest types in 2010). At the end of this process, some cells remained null as the forest cover in 1990 was larger than that of 2010. We filled those NULL values using modal imputation under two different neighbourhood window sizes (13 and 25 pixels). Next, a 5km buffer was added around Montérégie so as to improve the Circuitscape results later down the pipeline (see the Connectivity Analysis section). Not all NA cells were due to the difference in forest cover between 1990 and 2010, some were due to the fact that the LANDIS run did not cover some parts of the 5km buffer. Those cells were filled with the value corresponding to the forest type Mixed Forest of Medium Age. We decided on this imputed value because it is among the most suitable for all species and therefore unlikely to cause significant changes in the Circuitscape runs, even if it does not correspond to the mode or the mean for the region. In addition, because no change takes place in the buffer, those values can be equivalent to a "padding" around the landscape.

A second step was the preparation of the explanatory variables. Vector data was rasterized when needed. A significant effort was invested in working with the Canadian Census data, as Statistics Canada does not make available a spatial dataset for their census data. Therefore, the geometries of DAs and EAs had to be matched with the corresponding line in the Census dataset, based on the unique EA and DA IDs. Some of these (between 2 and 5 \% of the total number of DAs or EAs, depending of the time steps) have remained unmatched due to discrepancies between the (spatial) boundaries datasets of EAs/DAs available and the EAs/DAs listed in the StatsCan dataset. This led to areas with NAs values. These areas could not be ignored in our random forest model and had to be imputed. We used mean imputation for lack of a better method but are aware of the large number of issues associated with this practice \citep[see][]{lodder_impute_2014}.

In order to reduce run time as well as memory requirements, we ran the entire simulation at a 90m resolution, which is an aggregation (resample) of all the data in the rasters by a factor of 3, using a majority rule (changing the resolution of our data to 90m). Finally, all variables were extracted and turned into tabular data. Only relevant pixels (i.e. rows in the dataset) were kept to model each transition. For instance, only forest and agricultural land pixels were kept to model urbanization. The variables were standardized prior to running the model. We used the \verb|tidymodels| framework (version 0.1.0) in R to pre-process our tabular data and run our random forest models. \\

\subsubsection{Preliminary analysis of past land use change}

To better understand land use patterns in the region, we ran some preliminary analysis on land use change data at the municipality level. Montérégie is made up of 177 different municipalities. We generated land use matrices (amount of each land use category) and land cover transition matrices (1990 to 2010) for each municipality and for the land use categories corresponding to our model. We then performed multivariate analyses by running a first ordination using PCA (Principal Component Analysis), and then using Ward clustering % TODO add ref to legendre.
Because municipalities in Montérégie have variable size, the amount of change in each municipality needs to be standardized. The data was normalized prior to the ordination to reflect relative amounts of change in each municipality. We used the vegan package in R to produce such plots.\\

\subsubsection{Model Execution}

\subsubsection*{Random forest}

\subsubsection*{\textit{Calibration \& validation}}

Land use change data has a tendency to be highly unbalanced, because over a certain time period, the amount of pixels that have transitioned into a new state is usually small compared to the amount that did not transition into a new state. In the case of Random forest models, there are a few ways to deal with such imbalance. In this work, we simply down-sampled our dataset to reach a ratio of 2:1 (i.e. 2 pixels that did not transition for each pixel that did transition).

The RF model was calibrated on a training partition (70\% of the down-sampled dataset) for the timestep of 1990 to 2000. The model was then spatially validated on a test partition (30\% of the dataset), and then temporally validated for the timestep of 2000 to 2010 (100\% of the dataset for this timestep). \\

\subsubsection*{\textit{Performance evaluation}}

There are a number of ways to evaluate model performance in the context of land use change predictions. We used three different approaches:
\begin{itemize}
 \item We report the $R^{2}$ resulting from the random forest model, as well as variable importance (evaluated as gini impurity score).
 \item We used a ROC curve with Area Under the Curve (AUC) to compare true observed change with the predicted probabilities of transition. This method is widely used in land use change modelling. %TODO add ref (ANDY)
 \item We used a Conventional Cross Validation (CCV) method for the reporting of AUC values, with 10 folds, for which we also produce a measure of AUC with ROC curves.\\
\end{itemize}

\subsubsection*{ST-Sim}

ST-Sim \citep{daniel_state-and-transition_2016}  is a State and Transition model (STM) and can therefore be provided with probabilities of land use change. However in order to integrate the results of our random forest models, we parameterized ST-Sim differently than a classic STM. All probabilities are set to 1, and we leverage the spatial multiplier feature of the software. We provide our fitted probability surfaces (outputs of the RF models) as spatial multipliers. Therefore, each pixel's probability of change is provided by that surface (multiplied by 1).  This is not enough to reproduce past trends in land use change: we need to provide transition targets to the allocation algorithm. These targets are directly based on the amount of land change between 1990 and 2010 for time steps 1990 - 2010.\\

\subsubsection*{\textit{Model definition}}

The ST-Sim model uses three states: Forest, Urban and Agriculture. Forest was further subdivided into classes of age and type, derived from the LANDIS outputs: three forest types (\textbf{decididuous}, \textbf{mixt}, and \textbf{coniferous}) and three age classes (\textbf{young}, \textbf{medium}, and \textbf{old}) for a total of 9 classes.
The model allows for 4 groups of transitions:
\begin{itemize}
\item{\textbf{Urbanisation}}: groups \textbf{deforestation} (Forest $\Rightarrow$ Urban) and \textbf{agricultural loss} (Agriculture $\Rightarrow$ Urban).
\item{\textbf{Agricultural expansion}} (Forest $\Rightarrow$ Urban).
\item{\textbf{Forest Internals}}: groups the 72 combinations of forest change between the 9 forest internal states.
\end{itemize}
The model takes in 3 different strata. A stratum in ST-Sim refers to a subset of the entire landscape being considered for analysis. Strata are typically used to divide the overall landscape into regions or zones, each of which can have different model parameters specified. The three stratum are:
\begin{itemize}
\item{Primary stratum:} delineates Montérégie from the  surrounding buffer and highlights protected areas.
\item{Secondary stratum:} delineates municipalities of the region, to allow to set targets specific to each municipality. 
\item{Tertiary stratum:} delineates LANDIS land types, to allow to set transition multipliers for forest types that match the LANDIS simulations. \\
\end{itemize}

\subsubsection*{\textit{Spatial dependency}}

Under the parameters defined so far, ST-Sim takes care of transitioning the amount of land corresponding to the targets it is fed, and will do it probabilistically across the landscape according to the probability surface (spatial multipliers). There is an additional way to restrict this change via Adjacency parameters, which are essentially neighbourhood rules. Adjacency rules were defined from basic assumptions about transition spread. The list of neighbourhood rules given for each transition type (or group) is provided in the appendix table \ref{tab:neigh_rules}.

We added two more constraints for the Forest Internals group: the Time since Transition parameter which sets a minimum time of 20 years (2 timesteps) in between transitions between Forest  types states. In addition, pathway autocorrelation settings allows to further isolate the characterization of transition rules to the specific land type to which a pixel belongs.\\

\subsubsection*{\textit{Scenario definition}}

In the subsequent description, we make the distinction between:
\begin{itemize}
\item{\textit{historic runs}}, parameterized to reproduce the changes that happened between 1990 and 2010 in the region, with a different set of transition targets for 1990-2000 and 2000-2010.
\item{\textit{forecast runs}}, parameterized to project forward an average of the trends observed in the two decades, with a single set of transition targets (average of the two 1990-2000 and 2000-2010 targets) and for which only certain set of spatial multipliers change depending on climatic scenarios.
\end{itemize}
We defined two land use change scenarios, crossed with two different climate change scenarios, plus a control climate scenario. Those six scenarios in total are designed to  encompass the realm of possible future habitat changes for our 5 focal species.
The two land use change scenarios are:
\begin{itemize}
\item{\textbf{Business As Usual (BAU)}}: the land use change trends are projected forward into the future with no alteration to those trends.
\item {\textbf{Business As Usual + Reforestation (BAU-R)}}: the land use change trends are projected forward into the future, and reforestation is randomly occurring, converting to forest as much land as is lost through urbanization, for every timestep.
\end{itemize}
The climate scenarios influence the probability of transitions between forest types (i.e. succession and disturbances) they subsequently influence habitat suitability and connectivity. The three climate scenarios are:
\begin{itemize}
\item{\textbf{Historic (HIST)}}: No forest change: forest remains as it was in 2010. This is equivalent to a control treatment.
\item{\textbf{Baseline (BASE)}}: Forest change continues as it does in today's climate.
\item{\textbf{RCP 8.5 (RCP8)}}: The forest changes according to the emissions scenario RCP 8.5, characterised by increasing green-house gas emissions, high rates of population growth, modest GDP growth and low rates of technological development and uptake. It is the most severe of the future scenarios \citep{ipcc_summary_2013}.\\
\end{itemize}

\subsubsection*{\textit{Details of scenario parameters}}

We ran a total of seven scenarios (one historic run and six forecast runs). Two behaviours remain constant in all seven scenarios: no land use change  takes place in the 5km buffer. In addition, new forest laid by the reforestation transition does not change state for the remainder of the simulation (this is because LANDIS does not provide land type data for new pixels). Therefore, reforestation always produces a forest of the class "Medium Deciduous". In all scenarios described, ST-Sim with decadal time steps, for 10 (5?) iterations (10 MCMC realization).

The final list of scenarios is the following:
\begin{itemize}
\item Historic run - 1990 to 2010 \textbf{(HIST)}. In this historic run, the forest type for all three time steps was taken from the LANDIS inputs (forest type data 2010). This is of course not accurate, but because no change is allowed to take place, this is not a problem and it is only there to allow consistency and comparison with the other scenarios.
\item Forecast land use BAU - Historic climate - 2010 to 2100 \textbf{(BAU-HIST)}
\item Forecast land use BAU - Baseline climate - 2010 to 2100 \textbf{(BAU-BASE)}
\item Forecast land use BAU - RCP 8.5 climate - 2010 to 2100 \textbf{(BAU-RCP8)}
\item Forecast land use reforestation - Historic climate - 2010 to 2100 \textbf{(BAU-R-HIST)}
\item Forecast land use reforestation - Baseline climate - 2010 to 2100 \textbf{(BAU-R-BASE)}
\item Forecast land use reforestation - RCP 8.5 climate - 2010 to 2100 \textbf{(BAU-R-RCP8)}\\
\end{itemize}

\subsection{Habitat suitability and Connectivity modelling}

Connectivity modelling is similar to land use change modelling inasmuch as the diversity of methods at the researcher’s disposal is large \citep{calabrese_comparison_2004}. As mentioned in the intro, we implemented an analysis of Potential Functional Connectivity, or PFC. A typical PFC modelling workflow can be broken down to 3 steps: species selection, Habitat suitability modelling and connectivity analysis. The three steps are described below, and the methodology partially follows the methodology of Rayfield et al. This part of the methods is summarized in figure \ref{fig:workflow2}.\\

\subsubsection{Species Selection}

Species selection is unchanged from Rayfield et al. The five species chosen are presented in the appendix table \ref{tab:species}. It was demonstrated that modelling the connectivity needs of those 5 species was equivalent to modelling the needs of 14 species in the study region \citep{meurant_selecting_2018, albert_applying_2017}\\

\subsubsection{Habitat Suitability}

Habitat suitability analysis consists in reclassifying land use data into a resistance surface which is then used to model connectivity. For each of the maps obtained from the land use change analysis (for each of the timestep in every iteration and for each of the 5 species), we followed the following steps to reclass land use into resistance:
\begin{enumerate}
\item Classify the landscape as \textbf{forest} and \textbf{non forest}.
\item Reclassify non-forest pixels into resistance following the supplementary tables in  \citeauthor{rayfield_priorisation_2018}, with only a few alterations. See table \ref{tab:key_non_forest} for the reclassification key.
\item Clump the forest landscape into contiguous \textbf{patches} using queen case neighbourhood (includes diagonal).
\item Reclassify pixels in all patches as suitable (value = 1) unsuitable (value = 0) or moderately suitable (value = 0.5), based on forest age and type, and following guidelines in the supplementary tables in  \citeauthor{rayfield_priorisation_2018}.  The key for such reclassification is recorded in table \ref{tab:suit_pixls}.
\item  Compute the mean patch value based on the clumped outputs.
\item Classify each patch as \textbf{possibly suitable} or \textbf{unsuitable}, with a cut-off value of 0.5.
\item Among the suitable patches, calculate their size.
\item Classify \textbf{possibly suitable} patches into \textbf{fully suitable} patches or \textbf{too small} patches based on their size, using the values provided by \citeauthor{rayfield_priorisation_2018} (see table \ref{tab:patch_size}).
\item Reclassify patches into resistance, once again following the guidelines in \citeauthor{rayfield_priorisation_2018}'s supplementary tables. See table \ref{tab:hab_or_not} for the reclassification key.
\item Patch the re-classified non-forest map with the re-classified patches map.  
\end{enumerate}

Although these steps build on the work of \cite{albert_applying_2017} and \cite{rayfield_priorisation_2018}, we apply a simpler habitat suitability modelling workflow. There are a number of reasons for why we were not able to apply the exact same workflow as  in \citeauthor{rayfield_priorisation_2018}. First, the resolution of the land use change data differs: we used a 90m resolution versus a 30m resolution for \citeauthor{rayfield_priorisation_2018}. This has implications for the minimum patch size required for the two species (\textit{Blarina} and \textit{Plethodon}) which at the resolution of 90m becomes impossible to differentiate. In addition, we did not include a range of variables that were used in \citeauthor{rayfield_priorisation_2018}: forest density, soil drainage, distance from minor versus major roads,   \\

\subsubsection{Connectivity analysis}

Methods of connectivity modelling are grounded in important concepts, the most important of which is graph theory \citep{dale_graphs_2010}. Graph theory is the study of networks, and describes the properties of the elements (edges and vertices) that make up a network . Applying graph theory to a landscape comes down to collapsing that landscape into a network, from which metrics can be computed, and models can be fitted - for instance dispersal models, which makes metapopulation theory another important concept in connectivity\citep{hanski_habitat_1999}. Metapopulation theory describes how populations of a given species in a landscape persist due to dispersal and gene flow.

In this paper, we analyse connectivity using circuit theory-based software called Circuitscape. Circuit theory is used to simulate the movement of propagules. The software “borrows algorithms from electronic circuit theory to predict connectivity in heterogeneous landscapes''. It is a free and open software under MIT license developed originally in Python, and now in its 5th version, in Julia \citep{circuitjulia}.

For each of the maps produced in the suitability analysis, Circuitscape was run in two directions (“wall to wall” run, \cite{mcrae_conserving_2016}): east to west and north to south. This method allowed us to model omnidirectional animal movements, and required that we added the resulting North/South and East/West flow map. The results for each map were therefore added to produce a final set of flow maps for analysis. From these flow maps, we extracted the mean flow for each of the municipalities and for the entire flow map. \\

\section{Results}

We first describe the results of the preliminary analysis on land use change dynamics, then go on to describe the results of the hybrid land use change and connectivity model.

\subsection{Analysis of past land use changes in Montérégie}

The clustering and subsequent ordination of the land use change matrices of the 177 municipalities revealed that Montérégie has 5 profiles and land compositions (see figures \ref{fig:clustervals}, \ref{fig:PCAvals}, and \ref{fig:mapvals}):
\renewcommand{\labelitemi}{$\textendash$}
\begin{itemize}[leftmargin=0.5cm]
  \item \textbf{Forest - Dominant}: have the lowest level of fragmentation and are dominated by forest
  \item \textbf{Forest - Agriculture}: still have a healthy amount of forested areas but fragmentation is much more pronounced.
  \item \textbf{Agriculture - Dominant}: forested habitat is scarce and most of the remaining forest is classified as “Trees” in the AAFC dataset (forest fragment of less than 1 hectare)
  \item \textbf{Urban - Medium density}: correspond to the front of the wave of urban sprawling
  \item \textbf{Urban - High density}: urban cores make up most of the municipality
\end{itemize} 

The clustering and subsequent ordination of land use change profiles showed that Montérégie has 4 different profiles (see figures \ref{fig:clustertrans}, \ref{fig:PCAtrans}, and \ref{fig:maptrans}):
\begin{itemize}[leftmargin=0.5cm]
  \item \textbf{Urban Spread / Deforestation}: forest fragmentation is progressing mainly via the growth of urban land (in the west) or villegiatives pressures (in the east)
  \item \textbf{Urban Spread / Agricultural loss}: agriculture is losing ground to urban land
  \item \textbf{Agricultural Expansion / Fragmentation}: forest is losing ground to agriculture in those municipalities where forest is still quite present
  \item \textbf{Agricultural Expansion / Deforestation}: forest is already scare and is being replaced by agricultural lands
These results show interesting regional trends with a front line of fragmentation and deforestation on each side of the region and along the Richelieu river. \\ % (Fig. \ref{fig:map}).
\end{itemize}

\subsection{Random forest Models}

\subsubsection{Variable importance}

As previously mentioned, the primary goal of this chapter is not to draw strong inference with regards to land use change drivers in the region. It is still worth noting that  the most important variables to predict urbanization and agricultural expansion are the size of forest patches, respectively (see \ref{tab:varimp}). This demonstrates that small patches close to urban areas are the most under threat in the region.  \\

\subsubsection{Model performance}

The reported adjusted  $R^{2}$ for our urbanization and agricultural expansion random forest models are $0.57$ and $0.56$ respectively (see table \ref{tab:R_squares}). The values for AUC are comparatively higher: varying between $0.841$ and $0.931$ (depending of the validation set) for Agricultural Expansion and between $0.921$ and $0.939$ for Urbanisation. The resampled AUC (10-folds CCV) is equally as high with means of $0.929 \pm .002$ and $0.938 \pm 0.002$ (see table \ref{tab:R_squares_resample}).\\ 

\subsection{Land Use Change Model}

%% Descrive results of the model with figures that shows the changes in forest compositions and land use composition.  

\subsection{Connectivity Modelling}

%% Show results of historic versus scenarios (make different figure for only this chapter)

\section{Discussion}

% discuss 1) what methodological integration was achieved 2) the limitations of the method and 3) compare and contrast scenarios

\section{Conclusion}

%---------------------------------------------------------------------------------------------------------------------------------------------------

\newpage
\begin{center}
\section*{Figures \& Tables}
\end{center}

%---------------------------------------------------------------------------------------------------------------------------------------------------
% methods figures %TODO better caption describing colors

\begin{figure}[h]
\makebox[\textwidth]{
\includegraphics[width=1.3\textwidth]{figures/Chapter1_flowchart.png}
}
\caption{Workflow for data preparation, statistical modelling and land use change modelling.}
\label{fig:workflow1}
\end{figure}
\clearpage

\begin{figure}[h!]
\makebox[\textwidth]{
\includegraphics[width=1.3\textwidth]{figures/Chapter1_flowchart2.png}
}
\caption{Workflow for the habitat suitability and connectivity analyses.} 
\label{fig:workflow2}
\end{figure}
\clearpage

%---------------------------------------------------------------------------------------------------------------------------------------------------
% Results figures

% Figures: values
% Clustering
\begin{figure}[h!]
\centering
 \includegraphics[width=\textwidth]{figures/clustering_values.png}
 \caption{Results of Ward clustering for land use for municipalities (cut at 5 groups)}
 \label{fig:clustervals}
\end{figure}

% PCA
\begin{figure}[h!]
  \centering
    \includegraphics[width=0.9\textwidth]{figures/PCA_data_profiles.png}
  \caption{Ordination of land use data (proportions) for municipalities. Groups are derived from clustering in \ref{fig:clustervals}}
  \label{fig:PCAvals}
\end{figure}

% MAP
\begin{figure}[h!]
  \centering
    \includegraphics[width=\textwidth]{figures/profiles_land_use.png}
  \caption{Geographical distribution of the 5 profiles identified in \ref{fig:clustervals} and \ref{fig:PCAvals}.}
  \label{fig:mapvals}
\end{figure}

% Figures: Transitions
% Clustering
\begin{figure}[h!]
  \centering
    \includegraphics[width=\textwidth]{figures/clustering_trans.png}
  \caption{Results of Ward clustering for transition data for municipalities (cut at 4 groups)}
  \label{fig:clustertrans}
\end{figure}

% PCA
\begin{figure}[h!]
  \centering
    \includegraphics[width=0.9\textwidth]{figures/PCA_trans_profiles.png}
  \caption{Ordination of land use transition data for municipalities. Groups are derived from clustering \ref{fig:clustertrans}}
  \label{fig:PCAtrans}
\end{figure}

%MAP
\begin{figure}[h!]
  \centering
    \includegraphics[width=\textwidth]{figures/transition_prof_map.png}
  \caption{Geographical distribution of the 4 change profiles identified in \ref{fig:clustertrans} and \ref{fig:PCAtrans}.}
  \label{fig:maptrans}
\end{figure}

\clearpage

%---------------------------------------------------------------------------------------------------------------------------------------------------
% Results tables

% RF Variable importance
\begin{table}[h!]
\centering
\caption{Variable importance (gini impurity index) for both models - non-categorical variables only}
\label{tab:varimp}
\begin{tabular}{lcc}
\hline
\multicolumn{1}{c}{\multirow{2}{*}{\textbf{Variable}}} & \multicolumn{2}{c}{\textbf{Model}} \\ \cline{2-3} 
\multicolumn{1}{c}{} & \multicolumn{1}{l}{\textbf{Urbanisation}} & \multicolumn{1}{l}{\textbf{Agricultural expansion}} \\ \hline
Distance from urban land & 756.8292 & 1787.357 \\
Size of forest patch & 548.5826 & 5373.270 \\
Elevation & 580.9726 & 1823.918 \\
Population change & 447.1120 & 1138.319 \\
Income & 390.5668 & 1108.612 \\ 
\hline
\end{tabular}
\end{table}

\begin{table}[!htbp] \centering 
  \caption{$R^{2}$ values for both models and both validation sets.} 
  \label{tab:R_squares} 
\begin{tabular}{@{\extracolsep{5pt}} ccccc} 
\\[-1.8ex]\hline 
\hline \\[-1.8ex] 
Model & Validation set & $R^{2}$ & AUC & Average Precision \\ 
\hline \\[-1.8ex] 
Agricultural Expansion & Spatial & $0.566$ & $0.931$ & $0.652$ \\ 
Agricultural Expansion &  Temporal & $0.566$ & $0.841$ & $0.023$ \\ 
Urbanisation & Spatial & $0.573$ & $0.939$ & $0.234$ \\ 
Urbanisation & Temporal & $0.573$ & $0.921$ & $0.250$ \\ 
\hline \\[-1.8ex] 
\end{tabular} 
\end{table} 

\begin{table}[!htbp] \centering 
  \caption{$R^{2}$ values for both models in resampled 10 folds CCV.} 
  \label{tab:R_squares_resample} 
\begin{tabular}{@{\extracolsep{5pt}} cccccc} 
\\[-1.8ex]\hline 
\hline \\[-1.8ex] 
Model & Mean AUC & Std AUC  \\ 
\hline \\[-1.8ex] 
Agricultural Expansion & $0.929$ & $0.002$ \\ 
Urbanisation & $0.938$ & $0.002$ \\ 
\hline \\[-1.8ex] 
\end{tabular} 
\end{table} 
