% !TEX root = ./thesis.tex

\chapter*{\textbf{Chapter I Supplementary Material \\ \hspace{1em}}}
\addcontentsline{toc}{chapter}{Chapter I Supplementary Material}

\setcounter{chapter}{3}
\setcounter{table}{0}
\setcounter{figure}{0}

%---------------------------------------------------------------------------------------------------------------------------------------------------
% METHODS CHAP1

%\begin{landscape}
\begin{longtable}[c]{|p{5cm}|p{11cm}|}
%\centering
\caption{Description of the 5 species used in the connectivity model (taken from Rayfield et al. 2018)}
\label{tab:species} \\
%\begin{tabular}{|p{5cm}|p{10cm}|}
\hline
\hline
\textbf{Species} & \textbf{Description} \\ \hline
Northern short-tailed Shrew \newline \textit{Blarina brevicauda} & Abundant small fossorial mammal. This highly active species can live in a diversity of habitats (grasslands, old fields, marshy areas, gardens, and some developed areas) but is mainly found in deciduous and mixed old forests with thick understories that provide good cover for hiding from predators. It feeds primarily on earthworms found in areas with moist soils. It has a high reproductive rate and is generally a poor disperser although it can cross gaps of 50-100m. \\ \hline
American Marten \newline \textit{Martes americana} & Small vagile carnivorous predator. Found in core areas of dense (\textgreater{}60\% cover) and old (\textgreater{}70 yr) coniferous or mixed forests with complex vertical and horizontal structure. It requires large home ranges (above hundreds of ha). It generally avoids large openings and clearings (above few hundred meters wide) but crosses roads and frozen rivers easily. Deep persistent snow pack is a habitat critical element as it excludes predators (Canis latrans) and competitors (Martes pennanti) and provides good hunting conditions. This forest specialist is particularly sensitive to human activities. Juveniles are able to cover tens of kilometers when dispersing, more than what would be expected from body mass-based estimates. Trapped for its fur, this species has patrimonial and economical importance. \\ \hline
Red-Backed Salamander \newline \textit{Plethodon cinereus} & Terrestrial salamander. This sedentary and territorial forest-dwelling species lives under the leaf litter or coarse woody debris in mature and moist deciduous and mixed forests. It is a poor disperser that uses tens of square meters as a home range and rarely ventures more than 50 m in open fields. Roads and edges (up to 20-30 m) have a negative effect on populations’ densities and reduce individual movements. \\ \hline
Wood Frog \newline \textit{Rana sylvatica} & Forest-specialist amphibian. This species prefers mixed and coniferous stands with closed canopy (\textgreater 40\%) and moist soil covered with woody debris (for egg deposition) but can adapt to other closed habitats. Both aquatic (palustrine, fish-free wetlands, not open and permanent ones) and terrestrial habitats are essential, and distance between both should not exceed ca. 600 m. It is sensitive to forest edge (25-35 m), gaps, high intensity agriculture, human developments and recent clearcuts that can act as barriers to movement. This poor disperser is particularly sensitive to roads and habitat loss (fidelity to first breeding pond). \\ \hline
Black Bear \newline \textit{Ursus americanus} & Large opportunistic omnivorous mammal. This species likes deciduous and mixed mature forest with dense cover interspersed with small clearings and early-successional stages of forest that are rich in berry production (depends on main soil surface deposit). It uses broad territories (tens to hundreds of square km) to follow fruiting season by going upslope with the season. It is an effective seed disperser because it can travel long distances (up to 390 km), in particular male juveniles. It clearly avoids human activity (up to 5 km) and roads (up to 800 m), in particular highways, and is likely to take a detour instead of crossing a 60 m gap. \\
\hline
\hline
%\end{tabular}
\end{longtable}
%\end{landscape}

%---------------------------------------------------------------------------------------------------------------------------------------------------
% RF

% RF variables

\begin{table}[h!]
\centering
\caption{Description and data sources for all variables used in the RF-CA model}
\label{tab:variables}
\begin{tabular}{lll}
\hline
\textbf{Variable} & \textbf{Format} & \textbf{Source}  \\ 
\hline
Distance from urban land & \multirow{3}{*}{Raster} & \multirow{2}{*}{Generated from land cover data} \\ \cline{1-1}
Size of forest patch &  &  \\ \cline{1-1} \cline{3-3} 
Elevation &  & \begin{tabular}[c]{@{}l@{}}SRTM 30m from \\ Google Earth Engine \\ data library\end{tabular} \\ \hline
Population change & \multirow{2}{*}{\begin{tabular}[c]{@{}l@{}}Tabular data joined to vector \\ data and rasterized\end{tabular}} & \multirow{2}{*}{\begin{tabular}[c]{@{}l@{}}Canadian Census for \\ 1991, 2001 and 2011\end{tabular}} \\ \cline{1-1}
Income &  &  \\ 
\hline
\end{tabular}
\end{table}

%---------------------------------------------------------------------------------------------------------------------------------------------------
% STSIM

% Neighborhood rules

\begin{table}[h!]
\centering
\caption{Neighborhood rules for ST-Sim transitions}
\label{tab:neigh_rules}
\begin{tabular}{llcc}
\hline
\textbf{Transition} & \textbf{State counted} & \textbf{Neighborhood (m)} & \textbf{Minimum Proportion} \\ \hline
Urbanization & Urban & 500 & 0.7 \\
Agricultural Expansion & Agriculture & 250 & 0.9 \\
Reforestation & Forest (any type) & 250 & 0.9 \\
Forest Internals & Forest (specific type) & 225 & 0.15 \\ \hline
\end{tabular}
\end{table}

%---------------------------------------------------------------------------------------------------------------------------------------------------
% Connectivity analyses

% Habitat suitability 

\begin{table}[h!]
\centering
\caption{Pixel suitability based on forest types and age.}
\label{tab:suit_pixls}
\begin{tabular}{lccccccccc}
\cline{2-10}
 & \multicolumn{9}{c}{\textbf{Forest}} \\ \cline{2-10} 
\textbf{} & \multicolumn{3}{c}{\textbf{Deciduous}} & \multicolumn{3}{c}{\textbf{Mixt}} & \multicolumn{3}{c}{\textbf{Coniferous}} \\ \hline
\textbf{Species} & \textbf{Young} & \textbf{Medium} & \textbf{Old} & \textbf{Young} & \textbf{Medium} & \textbf{Old} & \textbf{Young} & \textbf{Medium} & \textbf{Old} \\ \hline
\textit{\begin{tabular}[c]{@{}l@{}}Blarina \\ brevicauda\end{tabular}} & 0 & 0.5 & 1 & 0 & 0.5 & 1 & 0 & 0 & 0 \\ \hline
\textit{\begin{tabular}[c]{@{}l@{}}Martes \\ americana\end{tabular}} & 0 & 0 & 0 & 0 & 1 & 1 & 1 & 1 & 1 \\ \hline
\textit{\begin{tabular}[c]{@{}l@{}}Plethodon \\ cinereus\end{tabular}} & 1 & 1 & 1 & 1 & 0.5 & 1 & 0 & 0 & 0 \\ \hline
\textit{\begin{tabular}[c]{@{}l@{}}Rana \\ sylvatica\end{tabular}} & 1 & 1 & 1 & 1 & 1 & 1 & 1 & 1 & 1 \\ \hline
\textit{\begin{tabular}[c]{@{}l@{}}Ursus\\ americanus\end{tabular}} & 1 & 1 & 1 & 0.5 & 0.5 & 0.5 & 1 & 1 & 1 \\ \hline
\end{tabular}
\end{table}

% Reclassification into resistance

\begin{table}[h!]
\centering
\caption{Resistance reclassification key for non-forest pixels.}
\label{tab:key_non_forest}
\begin{tabular}{lcccccc}
\hline
\textbf{Species} & \textbf{Urban land} & \textbf{Roads} & \textbf{Agricultural land} & \textbf{Wetlands} & \textbf{Water} & \textbf{Other} \\ \hline 
\textit{\begin{tabular}[c]{@{}l@{}}Blarina \\ brevicauda\end{tabular}} 	&  32 & 32 &	8    & 8 & 16 & 8\\ \hline
\textit{\begin{tabular}[c]{@{}l@{}}Martes \\ americana\end{tabular}} 	&  32 & 32 &	16  & 8 &	 16 & 8\\ \hline
\textit{\begin{tabular}[c]{@{}l@{}}Plethodon \\ cinereus\end{tabular}} 	&  32 & 32 &	8    & 8 & 32 & 8	\\ \hline
\textit{\begin{tabular}[c]{@{}l@{}}Rana \\ sylvatica\end{tabular}} 			&  32 & 32 &	8    & 2 & 8   &	 8	\\ \hline
\textit{\begin{tabular}[c]{@{}l@{}}Ursus\\ americanus\end{tabular}} 		&  32 & 32 &	16  & 2 &	 16 & 8\\ \hline
\end{tabular}
\end{table}

\clearpage

% Minimum patch

\begin{table}[h!]
\centering
\caption{Minimum patch size for suitability of a patch in the habitat suitability analysis.}
\label{tab:patch_size}
\begin{tabular}{lc}
\hline
\textbf{Species} 																									& \textbf{Minimum patch size} (hectares) \\ \hline 
\textit{\begin{tabular}[c]{@{}l@{}}Blarina \\ brevicauda\end{tabular}} 	&  1		\\ \hline
\textit{\begin{tabular}[c]{@{}l@{}}Martes \\ americana\end{tabular}} 	&  150		\\ \hline
\textit{\begin{tabular}[c]{@{}l@{}}Plethodon \\ cinereus\end{tabular}} 	&  1		\\ \hline
\textit{\begin{tabular}[c]{@{}l@{}}Rana \\ sylvatica\end{tabular}} 			&  1		\\ \hline
\textit{\begin{tabular}[c]{@{}l@{}}Ursus\\ americanus\end{tabular}} 		&  1200		\\ \hline
\end{tabular}
\end{table}

% Key patches type

\begin{table}[h!]
\centering
\caption{Resistance reclassification key for different forest patches types.}
\label{tab:hab_or_not}
\begin{tabular}{lccc}
\hline
\textbf{Species} & \textbf{Habitat patch} & \textbf{Habitat patch - too small} & \textbf{Non-habitat patch} \\ \hline 
\textit{\begin{tabular}[c]{@{}l@{}}Blarina \\ brevicauda\end{tabular}} 	&	1	&	2	&	4	\\ \hline
\textit{\begin{tabular}[c]{@{}l@{}}Martes \\ americana\end{tabular}} 	&  1	&	4	&	8	\\ \hline
\textit{\begin{tabular}[c]{@{}l@{}}Plethodon \\ cinereus\end{tabular}} 	&  1	&	2	&	4	\\ \hline
\textit{\begin{tabular}[c]{@{}l@{}}Rana \\ sylvatica\end{tabular}} 			&  1	&	2	&	4	\\ \hline
\textit{\begin{tabular}[c]{@{}l@{}}Ursus\\ americanus\end{tabular}} 		&  1	&	4	&	16	\\ \hline
\end{tabular}
\end{table}

% SURF

\begin{table}[h!]
\centering
\caption{SURF analysis parameters}
\label{tab:surf}
\begin{tabular}{lc}
\hline
\hline
Parameter & Value \\ \hline
Hessian threshold & 7000 \\
Number of octaves & 1 \\
Number of octave layers & 2 \\ \hline
\end{tabular}
\end{table}

\newpage

%---------------------------------------------------------------------------------------------------------------------------------------------------
% RESULTS CHAP1

% VALUES

% Clustering

\begin{figure}[h!]
\centering
 \includegraphics[width=\textwidth]{figures/clustering_values.png}
 \caption{Results of Ward clustering for land use for municipalities (cut at 5 groups)}
 \label{fig:clustervals}
\end{figure}

% TRANSITIONS

% Clustering

\begin{figure}[h!]
  \centering
    \includegraphics[width=\textwidth]{figures/clustering_trans.png}
  \caption{Results of Ward clustering for transition data for municipalities (cut at 4 groups)}
  \label{fig:clustertrans}
\end{figure}

%---------------------------------------------------------------------------------------------------------------------------------------------------
% ROC Curves

% Agex

\begin{figure}[h!]
\makebox[\textwidth]{
  \includegraphics[width=\textwidth]{figures/rf_ratio_2_agex_roc.png}
}
 \caption{ROC Curve for Agricultural Expansion}
 \label{fig:roc_agex}
\end{figure}

% Urb

\begin{figure}[h!]
\makebox[\textwidth]{
  \includegraphics[width=\textwidth]{figures/rf_ratio_2_urb_roc.png}
}
 \caption{ROC Curve for urbanisation}
 \label{fig:roc_urb}
\end{figure}

%  Resample, figure and table

\begin{figure}[h!]
\makebox[\textwidth]{
  \includegraphics[width=1.3\textwidth]{figures/double_roc_resample.png}
}
 \caption{ROC Curves for re-samples}
 \label{fig:roc_rs}
\end{figure}

%---------------------------------------------------------------------------------------------------------------------------------------------------
% Maps compare

\begin{figure}[h!]
\makebox[\textwidth]{
  \includegraphics[width=\textwidth]{figures/BAU_compare.png}
}
 \caption{Comparison of Monteregie at the beginning of the BAU simulation in 2010, and at the end in 2100 (BAU-Baseline scenario).}
 \label{fig:BAU_compare}
\end{figure}

\begin{figure}[h!]
\makebox[\textwidth]{
  \includegraphics[width=\textwidth]{figures/Ref_compare.png}
}
 \caption{Comparison of Monteregie at the beginning of the Reforestation simulation in 2010, and at the end in 2100 (BAU-Reforestion-Baseline scenario).}
 \label{fig:Ref_compare}
\end{figure}

\newpage

%---------------------------------------------------------------------------------------------------------------------------------------------------

\chapter*{\textbf{Chapter II Supplementary Material \\ \hspace{1em}}}
\addcontentsline{toc}{chapter}{Chapter II Supplementary Material}

%---------------------------------------------------------------------------------------------------------------------------------------------------
% METHODS/results CHAP2

% Workshop tables

\begin{table}[h!]
\centering
\caption{Opportunity and challenge table (Non Spatial).}
\label{tab:opp_chall_ns}
\begin{tabular}{m{0.15\textwidth}lm{0.5\textwidth}l}
\hline
\textbf{Table} &
  \textbf{\begin{tabular}[c]{@{}l@{}}Atout/Contrainte \\ (Opportunity/Challenge)\end{tabular}} &
  \textbf{\begin{tabular}[c]{@{}l@{}}Contenu\\ (Content)\end{tabular}} &
  \textbf{Score} \\ \hline
\multirow{5}{*}{Centre} &
  \multirow{3}{*}{Atouts} &
  Présence d'organisme environnementaux &
  8.00 \\ \cline{3-4} 
                       &                              & Basses terres: lien prioritaire l’échelle nationale & 4.00  \\ \cline{3-4} 
                       &                              & Programme ALUS                                      & 1.00  \\ \cline{2-4} 
                       & \multirow{2}{*}{Contraintes} & CPTAQ                                               & 18.00 \\ \cline{3-4} 
                       &                              & Besoin de rentabilité des entreprises agricoles     & 5.00  \\ \hline
\multirow{3}{*}{Est}   & Atouts                       & Vocation forestiere existante                       & 8.00  \\ \cline{2-4} 
                       & \multirow{2}{*}{Contraintes} & Tenure privée des terres                            & 2.00  \\ \cline{3-4} 
                       &                              & Plusieurs territoires couverts                      & 1.00  \\ \hline
\multirow{5}{*}{Nord} &
  \multirow{2}{*}{Atouts} &
  Réglementation favorable maintien couvert bois et corridor métropolitain &
  16.20 \\ \cline{3-4} 
                       &                              & PRMHH                                               & 10.80 \\ \cline{2-4} 
                       & \multirow{3}{*}{Contraintes} & Usage agricole prédominant                          & 44.33 \\ \cline{3-4} 
                       &                              & Compréhension et participation citoyenne            & 14.40 \\ \cline{3-4} 
                       &                              & Coûts pour faire de la connectivité                 & 10.80 \\ \hline
\multirow{6}{*}{Ouest} & \multirow{2}{*}{Atouts}      & Routes cours d'eau                                  & 3.00  \\ \cline{3-4} 
                       &                              & Presence de sols pauvre                             & 3.00  \\ \cline{2-4} 
                       & \multirow{4}{*}{Contraintes} & Les réalités économiques                            & 50.00 \\ \cline{3-4} 
                       &                              & Terres privées                                      & 12.00 \\ \cline{3-4} 
                       &                              & LPTAA                                               & 7.20  \\ \cline{3-4} 
                       &                              & Le monde politique                                  & 4.00  \\ \hline
\multirow{6}{*}{Montérégie} &
  \multirow{4}{*}{Atouts} &
  Milieu agricole: potentiel de faire des corridors si un levier est trouvé &
  12.60 \\ \cline{3-4} 
                       &                              & Présence d'acteurs locaux et education              & 8.00  \\ \cline{3-4} 
                       &                              & PRMHH                                               & 3.00  \\ \cline{3-4} 
                       &                              & Permettre l'aménagement forestier                   & 3.00  \\ \cline{2-4} 
                       & \multirow{2}{*}{Contraintes} & Routes et autoroutes                                & 4.00  \\ \cline{3-4} 
                       &                              & Terres privées, présence prédominantes              & 3.00  \\ \hline
\end{tabular}
\end{table}

% Opportunity /challenges

\begin{table}[]
\centering
\caption{Opportunity and challenge table (Spatial).}
\label{tab:opp_chall_s}
\begin{tabular}{m{0.15\textwidth}lm{0.5\textwidth}l}
\hline
\textbf{Table} &
  \textbf{\begin{tabular}[c]{@{}l@{}}Atout/Contrainte \\ (Opportunity/Challenge)\end{tabular}} &
  \textbf{\begin{tabular}[c]{@{}l@{}}Contenu\\ (Content)\end{tabular}} &
  \textbf{Score} \\ \hline
\multirow{5}{*}{Centre} & Atouts                       & Leaders politiques positifs                                  & 33.33          \\ \cline{2-4} 
                        & Les deux                     & Propriétés protégées                                         & 26.67          \\ \cline{2-4} 
                        & \multirow{3}{*}{Contraintes} & Pressions urbaines                                           & 42.86          \\ \cline{3-4} 
                        &                              & Manque d'adhésion des agriculteurs                           & 39.29          \\ \cline{3-4} 
                        &                              & Pressions agricoles                                          & 16.20          \\ \hline
\multirow{5}{*}{Est}          & Atouts                       & Mobilisation projet corridor bleu vert fondation séthy                         & 18.00 \\ \cline{2-4} 
                        & \multirow{4}{*}{Contraintes} & Autoroute 10                                                 & 26.67          \\ \cline{3-4} 
                        &                              & Pressions villegiatives                                      & 22.2           \\ \cline{3-4} 
                        &                              & Fragmentation des habitats liées au dev                      & 18.00          \\ \cline{3-4} 
                        &                              & Activites agricoles intensives                               & 16.2           \\ \hline
\multirow{4}{*}{Nord}         & \multirow{2}{*}{Atouts}      & Municipalite pro-protection des monteregiennes ex. st bruno mont saint hilaire & 8.00  \\ \cline{3-4} 
                        &                              & Comité municipal travail MRC MDY                             & 1.00           \\ \cline{2-4} 
                        & \multirow{2}{*}{Contraintes} & Autoroute 10 20 30 et autres                                 & 1.00           \\ \cline{3-4} 
                        &                              & Gestion de l'application des bandes riveraines               & 1.00           \\ \hline
\multirow{6}{*}{Ouest}  & \multirow{4}{*}{Atouts}      & Proximité des milieux                                        & 24.00          \\ \cline{3-4} 
                        &                              & Article 50.3 du règlement des exploitations agricoles        & 19.20          \\ \cline{3-4} 
                        &                              & Mobilisation des acteurs du milieux                          & 12.00          \\ \cline{3-4} 
                        &                              & Bande riveraine potentielle                                  & 10.00          \\ \cline{2-4} 
                        & \multirow{2}{*}{Contraintes} & Canal beauharnois isole                                      & 24.00          \\ \cline{3-4} 
                        &                              & Réticences de certains producteurs                           & 10.00          \\ \hline
\multirow{5}{*}{Montérégie 1} & \multirow{3}{*}{Atouts}      & Réseaux de sites avec couvert forestier                                        & 14.4  \\ \cline{3-4} 
                        &                              & Grande volonté d'action locale pour créer de la connectivité & 12.60          \\ \cline{3-4} 
                        &                              & Rétrécissement du fleuve                                     & 7.20           \\ \cline{2-4} 
                              & \multirow{2}{*}{Contraintes} & Agriculture intensive compenser la production                                  & 16.20 \\ \cline{3-4} 
                        &                              & Développement urbain                                         & 12.00          \\ \hline
\multirow{5}{*}{Montérégie 2} & \multirow{2}{*}{Atouts}      & Mobilisation sociale organisme conservation sensibilisation                    & 12.00 \\ \cline{3-4} 
                        &                              & Usage des sols favorable                                     & 10.00          \\ \cline{2-4} 
                        & \multirow{3}{*}{Contraintes} & Prix des terres agricoles                                    & 12.00          \\ \cline{3-4} 
                        &                              & Étalement urbain deuxième couronne                           & 10.00          \\ \cline{3-4} 
                        &                              & Pole logistique de transport                                 & 10.00          \\ \hline
\end{tabular}
\end{table}

\begin{table}[h!]
\centering
\caption{Breakdown of the area covered by each table in the workshop}
\label{tab:workshoptables}
\begin{tabular}{ll}
\hline
\textbf{Table} & \textbf{MRCs} \\ \hline
Ouest (West) & Vaudreuil, Haut SL, Beauharnois \\
Centre (Center) & Jardins, Haut Richelieu, Rouville, Roussillon \\
Nord (North) & \begin{tabular}[c]{@{}l@{}}Longueuil, Marguerite d'Youville, Vallée du richelieu, \\ Pierre de Saurel, Les Maskoutains\end{tabular} \\
Est & Brome-Missisquoi, Haute Yamaska, Acton \\
Transversal (x2) & Toute la Montérégie (All of  Montérégie)\\ \hline
\end{tabular}
\end{table}

%---------------------------------------------------------------------------------------------------------------------------------------------------
% Maps compare

\begin{figure}[h!]
\makebox[\textwidth]{
  \includegraphics[width=\textwidth]{figures/Corr_compare.png}
}
 \caption{Corr).}
 \label{fig:Corr_compare}
\end{figure}

\begin{figure}[h!]
\makebox[\textwidth]{
  \includegraphics[width=\textwidth]{figures/CorrRef_compare.png}
}
 \caption{CorrRef).}
 \label{fig:	CorrRef_compare}
\end{figure}

\begin{figure}[h!]
\makebox[\textwidth]{
  \includegraphics[width=\textwidth]{figures/CorrRefTar_compare.png}
}
 \caption{CorrRefTar.}
 \label{fig:	CorrRefT_compare}
\end{figure}

%---------------------------------------------------------------------------------------------------------------------------------------------------
% Flow

% Radar All
\begin{figure}[h!]
\makebox[\textwidth]{
  \includegraphics[width=\textwidth]{figures/radar_ggradar_both.png}
}
 \caption{Change in mean flow (in \% of the the 2010 flow) between 2010 and 2100, contrasting BAU scenario (solid line) with both other land use change and conservation scenarios.}
 \label{fig:flow_radar_both}
\end{figure}