% !TEX root = ./thesis.tex

\chapter*{\textbf{Chapter I Supplementary Material \\ \hspace{1em}}}
\addcontentsline{toc}{chapter}{Chapter I Supplementary Material}

\setcounter{chapter}{3}
%\begin{landscape}
\begin{longtable}[c]{|p{5cm}|p{11cm}|}
%\centering
\caption{Description of the 5 species used in the connectivity model (taken from Rayfield et al. 2018)}
\label{tab:species} \\
%\begin{tabular}{|p{5cm}|p{10cm}|}
\hline
\hline
\textbf{Species} & \textbf{Description} \\ \hline
Northern short-tailed Shrew \newline \textit{Blarina brevicauda} & Abundant small fossorial mammal. This highly active species can live in a diversity of habitats (grasslands, old fields, marshy areas, gardens, and some developed areas) but is mainly found in deciduous and mixed old forests with thick understories that provide good cover for hiding from predators. It feeds primarily on earthworms found in areas with moist soils. It has a high reproductive rate and is generally a poor disperser although it can cross gaps of 50-100m. \\ \hline
American Marten \newline \textit{Martes americana} & Small vagile carnivorous predator. Found in core areas of dense (\textgreater{}60\% cover) and old (\textgreater{}70 yr) coniferous or mixed forests with complex vertical and horizontal structure. It requires large home ranges (above hundreds of ha). It generally avoids large openings and clearings (above few hundred meters wide) but crosses roads and frozen rivers easily. Deep persistent snow pack is a habitat critical element as it excludes predators (Canis latrans) and competitors (Martes pennanti) and provides good hunting conditions. This forest specialist is particularly sensitive to human activities. Juveniles are able to cover tens of kilometers when dispersing, more than what would be expected from body mass-based estimates. Trapped for its fur, this species has patrimonial and economical importance. \\ \hline
Red-Backed Salamander \newline \textit{Plethodon cinereus} & Terrestrial salamander. This sedentary and territorial forest-dwelling species lives under the leaf litter or coarse woody debris in mature and moist deciduous and mixed forests. It is a poor disperser that uses tens of square meters as a home range and rarely ventures more than 50 m in open fields. Roads and edges (up to 20-30 m) have a negative effect on populations’ densities and reduce individual movements. \\ \hline
Wood Frog \newline \textit{Rana sylvatica} & Forest-specialist amphibian. This species prefers mixed and coniferous stands with closed canopy (\textgreater 40\%) and moist soil covered with woody debris (for egg deposition) but can adapt to other closed habitats. Both aquatic (palustrine, fish-free wetlands, not open and permanent ones) and terrestrial habitats are essential, and distance between both should not exceed ca. 600 m. It is sensitive to forest edge (25-35 m), gaps, high intensity agriculture, human developments and recent clearcuts that can act as barriers to movement. This poor disperser is particularly sensitive to roads and habitat loss (fidelity to first breeding pond). \\ \hline
Black Bear \newline \textit{Ursus americanus} & Large opportunistic omnivorous mammal. This species likes deciduous and mixed mature forest with dense cover interspersed with small clearings and early-successional stages of forest that are rich in berry production (depends on main soil surface deposit). It uses broad territories (tens to hundreds of square km) to follow fruiting season by going upslope with the season. It is an effective seed disperser because it can travel long distances (up to 390 km), in particular male juveniles. It clearly avoids human activity (up to 5 km) and roads (up to 800 m), in particular highways, and is likely to take a detour instead of crossing a 60 m gap. \\
\hline
\hline
%\end{tabular}
\end{longtable}
%\end{landscape}
