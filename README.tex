\PassOptionsToPackage{unicode=true}{hyperref} % options for packages loaded elsewhere
\PassOptionsToPackage{hyphens}{url}
%
\documentclass[]{article}
\usepackage{lmodern}
\usepackage{amssymb,amsmath}
\usepackage{ifxetex,ifluatex}
\usepackage{fixltx2e} % provides \textsubscript
\ifnum 0\ifxetex 1\fi\ifluatex 1\fi=0 % if pdftex
  \usepackage[T1]{fontenc}
  \usepackage[utf8]{inputenc}
  \usepackage{textcomp} % provides euro and other symbols
\else % if luatex or xelatex
  \usepackage{unicode-math}
  \defaultfontfeatures{Ligatures=TeX,Scale=MatchLowercase}
\fi
% use upquote if available, for straight quotes in verbatim environments
\IfFileExists{upquote.sty}{\usepackage{upquote}}{}
% use microtype if available
\IfFileExists{microtype.sty}{%
\usepackage[]{microtype}
\UseMicrotypeSet[protrusion]{basicmath} % disable protrusion for tt fonts
}{}
\IfFileExists{parskip.sty}{%
\usepackage{parskip}
}{% else
\setlength{\parindent}{0pt}
\setlength{\parskip}{6pt plus 2pt minus 1pt}
}
\usepackage{hyperref}
\hypersetup{
            pdfborder={0 0 0},
            breaklinks=true}
\urlstyle{same}  % don't use monospace font for urls
\usepackage[margin=1in]{geometry}
\usepackage{graphicx,grffile}
\makeatletter
\def\maxwidth{\ifdim\Gin@nat@width>\linewidth\linewidth\else\Gin@nat@width\fi}
\def\maxheight{\ifdim\Gin@nat@height>\textheight\textheight\else\Gin@nat@height\fi}
\makeatother
% Scale images if necessary, so that they will not overflow the page
% margins by default, and it is still possible to overwrite the defaults
% using explicit options in \includegraphics[width, height, ...]{}
\setkeys{Gin}{width=\maxwidth,height=\maxheight,keepaspectratio}
\setlength{\emergencystretch}{3em}  % prevent overfull lines
\providecommand{\tightlist}{%
  \setlength{\itemsep}{0pt}\setlength{\parskip}{0pt}}
\setcounter{secnumdepth}{0}
% Redefines (sub)paragraphs to behave more like sections
\ifx\paragraph\undefined\else
\let\oldparagraph\paragraph
\renewcommand{\paragraph}[1]{\oldparagraph{#1}\mbox{}}
\fi
\ifx\subparagraph\undefined\else
\let\oldsubparagraph\subparagraph
\renewcommand{\subparagraph}[1]{\oldsubparagraph{#1}\mbox{}}
\fi

% set default figure placement to htbp
\makeatletter
\def\fps@figure{htbp}
\makeatother


\author{}
\date{\vspace{-2.5em}}

\begin{document}

\textless{}!DOCTYPE html\textgreater{}

thesis

{2}

{General Introduction}

Space is a finite resource. How we, as a community, manage and govern
space is a reflection of the trade-offs and choices made by different
people and organizations at different spatial and temporal scales. Those
choices determine and regulate land use: if and how the resources held
on the land are exploited, transformed or conserved. The results of
those choices, referred to as land use and land cover change, is an
important threat to the biodiversity and ecosystem function. One example
of ecosystem function affected by land use that is of crucial importance
to biodiversity is ecological connectivity. Ecological connectivity is
the extent to which the landscape supports the movements of organisms
(Gonzalez et al.~2017), and is paramount for the resilience of both
populations and ecosystem services (Mitchell et al.~2015) in
heterogeneous and fragmented landscapes. Land use changes such as urban
sprawl can cause deforestation, fragmenting habitats, and slowly eroding
ecological connectivity. Many urban landscapes are experiencing
uncontrolled urban sprawl and have suffered losses in connectivity and
ecosystem services in consequence. Examples include cities like
Barcelona (Marulli and Mallarach 2005), New York City (McPhearson et
al.~2014), and also Montreal (Dupras and Alam 2015). The forces behind
those land use changes are complex and understanding them is an obstacle
to conservation planning (Worboys et al.~2010). Because land use change
is a social process with consequences of both social and ecological
nature, it is best understood within the concept of social-ecological
system (Ostrom 2009). A social-ecological system (SES) can be understood
as the set of human and non-human actors, the set of natural habitats
they inhabit and resources and use, and the set of interactions that are
maintained between all the components of the system. SESs thus form
complex and integrated aggregates of interactions (Hinkel et al.~2014).
Those interactions also impact governance, the process by which actors
in power establish rules and laws (Bissonnette et al.~2018).
Connectivity conservation planning refers to the enterprise that engages
multiple actors such as academics, NGOs, governmental bodies at
different scales and in a common goal to conserve the ecological
connectivity of the landscape. Connectivity conservation methods usually
entails modelling connectivity of the landscape of interest and using a
prioritisation method to determine conservation priorities. However,
current connectivity conservation planning methods have at least two
major limitations: their prioritisation process fails to take into
account risks associated with future land use change, and also fails to
confront the results of the prioritisation with the priorities perceived
by stakeholders. Not taking into account risks of land use change would
in theory lead to ill-informed conservation planned that would be
over-optimistic with regard to their probability of success. In
addition, failing to integrate the perceptions of stakeholders is
detrimental to conservation for two reasons: first, it most likely mean
that conservation will fail to gather enough local momentum to lead to
actual policy change, and second, it means that the only tool for
decision making will be the modelisations, whereas these are incomplete
representation of the landscape and would benefit from inputs from
stakeholders. This is especially true in landscapes where a considerable
effort of connectivity modelling has already been conducted, like in the
landscape of interest in this thesis, the southern Quebec region of
Montérégie. Montérégie is situated southeast of the city of Montreal,
and contains parts of the Greater Montreal Area (GMA). The ecological
connectivity of the GMA and its benefits as a provider of ecosystem
services has recently been assessed in a report to the Quebec ministry
of the environment (Rayfield 2018, unpublished). This study focused on
identifying regions of highest connectivity, and therefore of highest
priority for the conservation of biodiversity and ecosystem services.
Other work by Rayfield et al.~(2019, unpublished) has extended the
analysis of connectivity to the whole of the Saint Lawrence Lowlands.
Although the map produced by this analysis is a snapshot of the current
state of connectivity in the region, methods are available for including
future land use and climate change impacts (Albert et al.~2017). Those
methods rely on established the use of land use and land cover change
models whose complexity has increased from simple probabilistic state
transition models to more advanced approaches using targets, discrete
events and accounting for the time elapsed since the last transition
(Verburg and Overmars 2009, Daniel et al.~2016). Methods are also
available to include stakehodler's input in connectivity conservation
planning. Some of those methods have been developed through the
methodology of participatory modeling, which can be defined as a
modelling framework that can integrate knowledge from multiple sources,
even if this knowledge is generated by different processes. For
instance, it is possible landscape perceptions by different actors and
quantitative modelling. Those methods often rely on collecting data
through a community-driven process during workshops. Those methods are
time consuming and require a long term engagement with a given community
over many years. Other workshop-based methods are less involved, and
allow researchers to simply collect data to be confronted with the
results of traditional modeling techniques. In this thesis, we show an
attempt to remedy the two issues we identified above, in an ongoing
effort of connectivity conservation planning for the region of
Montérégie in Southern Quebec, Canada. In the first chapter, we built on
past work of connectivity modelling using circuit theory in the region
and complemented it with land use change modelling that uses a
combination of statistical modeling and MCMC-based simulations. In the
second chapter, we compare those results with the perceived conservation
priorities in the landscape, using data collected during a day-long
workshop with stakeholders. The Montérégie is relevant for our questions
given the recent political momentum gained by connectivity conservation.
There is a strong political will in the region for the conservation of
ecological connectivity.

Integrating Land Use Change Modelling with Connectivity Modelling

{Valentin Lucet{1}, Andrew Gonzalez{1}}

Author Affiliations: {{1}Department of Biology, McGill University}

Abstract

Space is a finite resource. How we manage and govern space is a
reflection of the trade-offs and choices made by people and
organizations at different spatial and temporal scales. Those choices
primarily determine and regulate land use: if and how space is organized
and whether resources are exploited, transformed or conserved.
Ecological connectivity, defined as the extent to which the landscape
supports the movements of organisms, can be strongly affected by land
use. It is an important component of the resilience of populations in
heterogeneous and fragmented landscapes. Land use changes such as urban
sprawl and agricultural intensification intensify habitat fragmentation
and landscape homogenization, leading to the erosion of ecological
connectivity. The Monteregie region in southern Quebec, where this work
takes place, is experiencing urban growth and sprawl. We present a
framework that integrates land-use change and connectivity modeling to
forecast future changes in connectivity, using a combination of
statistical modeling, MCMC-based simulations, and circuit theory. Models
trained on past land use data were used to project future land use
changes using different scenarios, and estimate future changes in
functional connectivity for 5 different umbrella species. We contrast
the past and future impacts of trends in land use (i.e.urbanization,
agricultural expansion, and deforestation) and derive conservation
priorities for the design of a local network of connected protected
areas resilient to future landscape change. We demonstrate the
flexibility of a scenario approach in forecasting the range of possible
futures for ecological connectivity in the region and show that taking
probable landscape changes into account lead to different conservation
priorities. We conclude on the importance of considering such changes to
produce a resilient network of protected areas and highlight the need
for a multi stakeholder approach in the definition of scenarios and
conservation priorities.

Introduction

Problem Statement: Connectivity conservation planning methods does not
account for risks associated with future land use change. This can
potentially lead to ill-informed conservation plans with low chances of
success.

Research question: How can we explain past changes in land use and
connectivity in Montérégie and better predict future changes?

In this first chapter, we build on past work of connectivity modelling
using circuit theory in the region and complement it with land use
change modelling that uses a combination of statistical modeling and
MCMC-based simulations. Models trained on past land use data were used
to project future land use changes and estimate future changes in
potential functional connectivity for 5 different umbrella species. We
derive conservation priorities for the design of a local network of
connected protected areas resilient to future landscape change. This
work is the continuation of two important contributions on the
connectivity of the region: Albert et al.~(2017) and Rayfield et
al.~(2019, unpublished). In a seminal paper, Albert laid out the first
methodological steps we follow: umbrella species selection and
connectivity modelling. They also included a simple land use change
model that was parameterized to replicate plausible change in the
region. Rayfield improved on Albert's work by increasing the scale of
the analysis and reducing the number of focal species. They showed that
because species had redundant connectivity needs, modelling the needs
for 5 species resulted in qualitatively similar results than when
modeling the needs of all 14 species, like in Albert et al.~(2017). They
demonstrated that we could exploit this redundancy to reduce the
computing time needed for analysis. This is welcomed as land use change
simulation is computationally intensive. In this chapter, we utilise
Albert and Rayfield's framework, modelling connectivity for the 5 focal
species identified by Rayfield, and using their workflow for habitat
suitability and connectivity analysis. We complement this framework with
a land use model that combines statistical modeling and MCMC-based
simulations. It is important to note that the primary goal of this
chapter is not to draw strong inference with regards to land use change
drivers in the region, but to provide enough predictive power to
replicate the trends in land use change that have been observed and
project those trends forward into the future. We predict that our land
use change model will simulate an overall decrease in connectivity
change for our focal species, given the fact that we will be simulating
a ``business as usual'' scenario for the change in the region. We do not
generate specific predictions for each of our focal species.

Methods

Study Site

To be completed.

Sampling

To be completed.

Analyses

To be completed.

Results

To be completed. (Fig. 1.1).

Discussion

Figures \& Tables

Figure 1{}

Table 1.{}

Site

Weeks

S

XS

Daily

A/B

Total

Nhlanguleni (NHL)

3

0

0

0

Yes

6

Nwaswitshaka (NWA)

3

1

1

4

Yes

18

De LaPorte (DLP)

1

1

1

0

Yes

6

Kwaggas Pan (KWA)

2

1

1

0

Yes

8

Girivana (GIR)

3

0

0

0

Yes

6

Witpens (WIT)

3

0

0

0

Yes

6

Imbali (IMB)

3

0

0

0

Yes

6

Hoyo Hoyo (HOY)

3

1

1

0

Yes

10

Nyamarhi (NYA)

3

1

1

0

Yes

10

Ngosto North (NGO)

3

1

1

0

Yes

10

BLANK

2

0

0

0

No

2

29

6

6

4

88

My Second chapter

{A. Student{1}, B. Supervisor{1}}

Author Affiliations: {{1}Department of Biology, McGill University}

Abstract

Introduction

Methods

Study Site

Sampling

Analyses

Results

Discussion

Figures \& Tables

Appendix

Chapter I Supplementary Data and Results

Sup Table 2

\protect\hypertarget{tab:water_qual}{}{{[}tab:water\_qual{]}}

Sample code

Site

Date

Temp ({∘}C)

mS/cm

DO (\%)

DO (mg/L)

pH

DLP\_8

DLP

July 10

15.27

3.11

83.37

39.67

9.16

GIR\_1

GIR

June 24

18.58

1.95

50.83

42.00

9.27

GIR\_2

GIR

July 1

21.85

1.80

74.47

41.00

9.24

GIR\_3

GIR

July 8

20.72

1.90

88.47

39.00

9.35

HOY\_2

HOY

June 22

17.59

3.18

14.53

43.43

8.14

HOY\_3

HOY

June 29

17.84

3.01

42.53

40.00

8.25

HOY\_4

HOY

July 6

16.83

2.96

39.27

35.90

8.39

IMB\_2

IMB

June 22

15.17

2.46

74.80

46.77

8.19

\end{document}
